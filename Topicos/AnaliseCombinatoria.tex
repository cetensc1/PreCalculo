\chapter{Análise Combinatória}

Análise combinatória é a área da matemática dedicada aos processos de contagem. É nela que encontramos as ferramentas para contar o número de possibilidades de agrupar e organizar elementos de grupos finitos sobre determinadas circunstâncias.

Um exemplo de um problema de contagem é:
\begin{exem}\label{escnum}
 Quantos números diferentes, sem algarismos repetidos, podemos escrever usando os algarismos $\{4,5,6\}$?

 \underline{Resolução:}

 Para facilitar a compreensão da análise do número de possibilidades vamos usar os seguintes esquemas, chamados árvores de possibilidade:
 \begin{displaymath}
    \xymatrix{4 \ar[dd] \ar[r] \ar[dr] &  5 \\
                & 6 \ar[d] \\
                \text{1 possibilidade} & \text{2 possibilidades} \\
              5 \ar[dd] \ar[r] \ar[dr] & 4 \\
                & 6 \ar[d] \\
                \text{1 possibilidade} & \text{2 possibilidades}}
 \end{displaymath}

 \begin{displaymath}
    \xymatrix{ 6 \ar[dd] \ar[r] \ar[dr] &  4 \\
                & 5 \ar[d] \\
                \text{1 possibilidade} \ar[d] & \text{2 possibilidades} \ar[d] \\
                \text{3 possibilidades} & \text{2 possibilidades}}
\end{displaymath}
Portanto o total de possibilidades é $3 \cdot 2=6$ que como podemos observar é exatamente o número de linhas que temos na 2ª coluna.

Observe que as possibilidades são as seguintes $\{45, 46, 54, 56, 64, 65\}$.

\fim
\end{exem}

Este e outros problemas de contagem, por vezes com condições mais complexas, podem ser resolvidos por meio da análise combinatória, cujas principais ferramentas são princípio fundamental da contagem, fatorial, permutações, arranjos e combinações. Sendo que permutações, arranjos e combinações podem ser simples ou com elementos repetidos.

\section{Princípio fundamental da contagem}

 \vskip0.3cm
 \colorbox{azul}{
 \begin{minipage}{0.9\linewidth}
 \begin{center}
  Se um evento é composto por duas etapas sucessivas e independentes de tal maneira que o número de possibilidades na primeira etapa é $m$ e para cada possibilidade da primeira etapa o número de possibilidades na segunda etapa é $n$, então o número total de possibilidades de o evento ocorrer é dado pelo produto $m \cdot n$.
 \end{center}
 \end{minipage}}
 \vskip0.3cm

No exemplo \ref{escnum} acima queremos contar quantos números diferentes podemos escrever usando os algarismos $\{4,5,6\}$. Este é um exemplo que pode ser resolvido usando o princípio fundamental da contagem. Neste caso, a primeira etapa é o 1º algarismo utilizado para formar o número, para o qual temos $3$ possibilidades de escolha. A segunda etapa é o 2º algarismo utilizado para formar o número, que como não pode repetir algarismos, por exemplo não é permitido formar o número $44$, após cada escolha do 1º algarismo sobram então $2$ algarismos, como possibilidades de escolha para a 2º etapa, que é o 2º algarismo. Logo o número total de possibilidades de o evento ocorrer é dado pelo produto $3 \cdot 2= 6$, como vimos no exemplos \ref{escnum}.

\begin{obs}
 O produto do número de possibilidades vale para qualquer número de etapas independentes.
\end{obs}


\section{Fatorial}

\vskip0.3cm
 \colorbox{azul}{
 \begin{minipage}{0.9\linewidth}
 \begin{center}
  O fatorial de um número natural $n$, representado por $n!$, é o produto de todos os inteiros positivos menores ou iguais a $n$, ou seja,
 \[n! = n.(n-1).(n-2)...3.2.1 \ . \]
 \end{center}
 \end{minipage}}
 \vskip0.3cm

 Considera-se $\destaque{0!= 1}$ e $\destaque{1!= 1}$.


\section{Permutações}

Dado um conjunto com $n$ elementos, o ato de organizar estes elementos ordenando-os, de forma que cada nova ordem é considerada como um novo elemento é chamado de permutação. Existem dois tipos de permutação, a permutação simples quando os $n$ elementos são distintos, e a permutação com repetição quando no conjuntos de $n$ elementos temos elementos repetidos.

\subsection{Permutação simples}

Dados $n$ elementos distintos, vejamos como calcular quantos agrupamentos é possível formar usando os $n$ elementos.
\begin{exem}
 Quantos números de 3 algarismo (sem repeti-los num mesmo número) podemos formar com os algarismos $2, 4, 8$?

 \underline{Resolução:}
 \begin{displaymath}
    \xymatrix{2 \ar[dd] \ar[r] \ar[dr] &  4 \ar[r] & 8 \\
                & 8 \ar[d] \ar[r] & 4 \ar[d]\\
                \text{1 possibilidade} & \text{2 possibilidades} & \text{1 possibilidade} }
 \end{displaymath}

 \begin{displaymath}
    \xymatrix{4 \ar[dd] \ar[r] \ar[dr] & 2 \ar[r] & 8 \\
                & 8 \ar[d] \ar[r] & 2 \ar[d] \\
                \text{1 possibilidade} & \text{2 possibilidades} & \text{1 possibilidade}}
 \end{displaymath}

 \begin{displaymath}
    \xymatrix{ 8 \ar[dd] \ar[r] \ar[dr] &  2 \ar[r] & 4 \\
                & 4 \ar[d] \ar[r] & 2 \ar[d] \\
                \text{1 possibilidade} \ar[d] & \text{2 possibilidades} \ar[d] & \text{1 possibilidade} \ar[d] \\
                \text{3 possibilidades} & \text{2 possibilidades} & \text{1 possibilidade}}
\end{displaymath}
 Assim pelo princípio fundamental da contagem temos $3 \cdot 2 \cdot 1= 6$ possibilidades, observe que como neste caso nosso conjunto possui 3 elementos temos $n= 3$, logo a o número de possibilidade é igual a $n!= n \cdot (n-1) \cdot (n-2)= 3 \cdot 2 \cdot 1= 6$.
\end{exem}

De maneira geral o número de permutações simples de $n$ elementos é dado por:
\[n!= n \cdot (n-1) \cdot (n-2) \cdots 3 \cdot 2 \cdot 1 \ .\]



\subsection{Permutação com repetição}

Dado um conjunto de $n$ elementos no qual existem elementos repetidos, vejamos através de um exemplo como calcular quantos agrupamentos é possível formar usando os $n$ elementos.

\begin{exem}
 Quantos anagramas existem com as letras da palavra ABACAXI?

 \underline{Resolução:}

 A palavra ABACAXI possui 7 letras, logo $n=7$, e uma vogal, a vogal A, que se repete três vezes. Portanto, o número de anagramas desta palavras é dado por:
 \[P_{7}^{3}= \frac{7!}{3!}= \frac{7.6.5.4.3!}{3!}= 7.6.5.4= 840 \text{ anagramas.}\]
 Dentre eles temos por exemplo: IXACABA, ABACAIX ....
\end{exem}

\begin{exem}
 Quantos anagramas existem com as letras da palavra MATEMÁTICA, desconsiderando o acento?

 \underline{Resolução:}

 A palavra MATEMÁTICA, possui 10 letras, logo $n=10$, além disso a vogal A se repete 3 vezes, a consoante T se repete 2 vezes, e a consoante M se repete 2 vezes. Portanto, o número de anagramas desta palavras é dado por:
 \[P_{10}^{3;2;2}= \frac{10!}{3!2!2!}= \frac{10.9.8.7.6.5.4.3!}{3!.2.1.2.1}= 10.9.8.7.3.5.2= 151200 \text{ anagramas.} \]

\end{exem}

\begin{obs}
 No caso em que a palavra da qual estamos calculando o número de anagramas possíveis não possuir letras repetidas então se trata de uma permutação simples.
\end{obs}

\section{Arranjos e combinações simples}

Acabamos de ver como calcular a quantidade de possibilidades de ordenar $n$ elementos distintos, o que é chamado de permutação simples.

Agora consideremos um conjunto com $n$ elementos distintos vamos estudar os agrupamentos de 1 elemento, 2 elementos, 3 elementos, $\cdots$, $p$ elementos com $p \leqslant n$, deste conjunto.

\subsection{Arranjos simples}

Dado um conjunto com $n$ elementos distintos, a ferramenta que nos permite contar de quantas formas diferentes podemos tomar $p$ destes $n$ elementos sendo $p \leq n$, \textit{importando a ordem} na qual estes elementos são escolhidos é chamado de arranjo simples. Antes de vermos a definição formal deste conceito vejamos um exemplo no qual o utilizamos.

\begin{exem}
 Usando os algarismo 2, 3, 5, 7 e 9, quantos números naturais de 3 algarismos distintos podemos formar?

 \underline{Resolução:}
 \[\overline{\text{centena}} \ \ \ \overline{\text{dezena}} \ \ \  \overline{\text{unidade}} \]
  Há 5 possibilidades para a centena, 4 possibilidades para a dezena, 3 possibilidades para a unidade. No total podemos então formar: $5 \cdot 4 \cdot 3= 60$ números.

  Neste exemplo fizemos arranjos de 5 elementos 3 a 3, e o número de possíveis arranjos é 60.

  Indicamos isso por: $A_{5, 3}= 5 \cdot 4 \cdot 3= 60$.
\end{exem}

Para o caso geral temos a seguinte definição que generaliza a conta que acabamos de fazer:

\vskip0.3cm
 \colorbox{azul}{
 \begin{minipage}{0.9\linewidth}
 \begin{center}
  Arranjos simples de $n$ elementos tomados $p$ a $p$ $(p \leqslant n)$ são agrupamentos ordenados diferentes que se podem formar com $p$ dos $n$ elementos dados. (A ordem em que os elementos são escolhidos IMPORTA!).

 Indica-se por $A_{n, p}$ ou $A_{n}^{p}$ o total de possíveis agrupamentos deste tipo, que calculamos por:
 \[A_{n, p}= \frac{n!}{(n-p)!}\]
 \end{center}
 \end{minipage}}
 \vskip0.3cm


\subsection{Combinações simples}

Dado um conjunto com $n$ elementos distintos, a ferramenta que nos permite contar de quantas formas diferentes podemos tomar $p$ destes $n$ elementos sendo $p \leq n$, \textit{não importando a ordem} na qual estes elementos são escolhidos é chamado de combinação simples. Antes de vermos a definição formal deste conceito vejamos um exemplo no qual o utilizamos.

\begin{exem}
 De quantas maneiras diferentes podemos escolher 2 sabores de sorvete em uma sorveteria que dispõe dos seguintes sabores: morango (M), chocolate (C), flocos (F)?

 \underline{Resolução:}

 As possibilidades são: $\{M, C\}, \{M, F\}, \{C, F\}= 3$ possibilidades.

 Neste caso $n= 3$ e $p= 2$. Desafio refazer o cálculo usando a fórmula.
\end{exem}

\vskip0.3cm
 \colorbox{azul}{
 \begin{minipage}{0.9\linewidth}
 \begin{center}
  Combinações simples de $n$ elementos tomados $p$ e $p$ $(p \leq n)$ são os subconjuntos com exatamente $p$ elementos que se podem formar com os $n$ elementos dados. (A ordem em que os elementos são escolhidos NÃO IMPORTA!).

 Indica-se por $C_{n, p}$, $C_{n}^{p}$ ou ${n \choose p} $ o número total de combinações de $n$ elementos tomados $p$ a $p$ e calcula-se por:
 \[C_{n, p}= \frac{n!}{p!(n-p)!}\]
 \end{center}
 \end{minipage}}
 \vskip0.3cm


\begin{exem}
 De quantas maneiras diferentes podemos escolher 3 sabores de sorvete em uma sorveteria que dispõe dos seguintes sabores: morango (M), chocolate (C), flocos (F), doce de leite (D), açaí (A), baunilha (B), passas (P)?

 \underline{Resolução:}

 Neste caso $n= 7$ e $p= 3$, portanto o número de combinações possíveis é:
 \[C_{7, 3}= \frac{7!}{3!(7-3)!}= \frac{7.6.5.4!}{3!4!}= 7.5= 35 \text{ combinações.}\]
\end{exem}

\newpage
