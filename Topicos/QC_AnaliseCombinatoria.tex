\section{Questões}

\begin{enumerate}
 \item (Lógica - Fundatec - 2018) Quantas senhas de 4 caracteres distintos podem ser formadas quando são permitidas somente vogais maiúsculas e minúsculas e os algarismos 1, 2, 3, 4, 5, 6, 7, 8 e 9?
\begin{multicols}{2} 
\begin{enumerate}[a)]
\item 982.080
\item 116.280
\item 130.321
\item 93.024
\item 24.024
\end{enumerate}
\end{multicols}

 \item (UDESC/Fepese - 2009) Um restaurante oferece 20 tipos de pizza, 10 tipos de salada e 5 tipos de sobremesa.
  
  Considere que uma pessoa pretende se servir de:
  \begin{itemize}
   \item 1 tipos de pizza
   \item 1 tipos de salada
   \item 2 tipos de sobremesa
  \end{itemize}
  
  Quantas opções tem essa pessoa?

  \begin{enumerate}
  \item 1000
  \item 1200
  \item 2400
  \item 3600
  \item 4800
 \end{enumerate}
 
 \newpage
 \item (UDESC/Fundatec - 2015) Quantos números naturais ímpares de quatro algarismos distintos existem?
 \begin{enumerate}
 \item 10.000
 \item 4.032
 \item 2.520
 \item 3.024
 \item 2.240
 \end{enumerate}
 
 \item (UDESC/Fundatec - 2015) Gabriel e Anita são funcionários de uma empresa com um total de 16 funcionários, dos quais 7 são homens e 9 são mulheres. Todos os funcionários se reunira, para formar uma Brigada de Combate ao Incêndio que deve ter 4 funcionários. O número de brigadas distintas onde o Gabriel, obrigatóriamente, participa e Anita não participa é:
 \begin{enumerate}
 \item 2.184
 \item 364
 \item 2.730
 \item 1.365
 \item 460
 \end{enumerate}
 
 \item (UDESC/IBC - 2009) De um grupo de 5 homens e 5 mulheres, quantas comissões de 4 elementos podem ser formadas de modo que 3 elementos dessas comissões tenham o mesmo sexo?
 \begin{enumerate}
 \item 25
 \item 50
 \item 100 
 \item 200 
 \item 400
 \end{enumerate}
 
 \item (UDESC/IBC - 2009) A abertura de certo tipo de mala depende de dois cadeados. Para abrir o primeiro, é preciso digitar sua senha, que consiste num número de três algarismos distintos escolhidos de 1 a 9. Aberto o primeiro cadeado, deve-se abrir o segundo, cuja senha obedece às mesmas condições da primeira. Nessas condições, o número máximo de tentativas necessárias para abrir a mala é:
 \begin{enumerate}
 \item 10.024
 \item 5.040
 \item 2.880
 \item 1.440
 \item 1.008
 \end{enumerate}
 

\end{enumerate}
Gabarito: 1 d); 2 e); 3 e); 4 b); 5 c); 6 e).  

