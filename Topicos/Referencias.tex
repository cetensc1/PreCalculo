 \phantomsection
\chapter*{Referências}
\addcontentsline{toc}{chapter}{Referências}

 \begin{enumerate}[(1)]
  \item \textbf{Regras de divisibilidade.} Fonte: \href{http://mundoeducacao.bol.uol.com.br/matematica/regras-divisibilidade.htm}{Mundo Educação}. Acesso em: 18 de janeiro de 2018.

  \item Guidorizzi, Hamilton Luiz. \textbf{Um curso de cálculo}. vol 1. 5 ed. Rio de Janeiro: LTC, 2013.

  \item Kuhlkamp, Nilo. \textbf{Cálculo 1}. 5 ed. rev. Florianópolis: Ed. da UFSC, 2015.
  
  \item Domingues, Hygino Hugueros. \textbf{Fundamentos da Aritmética}. 2 ed. rev. Florianópolis: Ed. da UFSC, 2017.

  \item Spiegel, Murray R. \textbf{Álgebra}. 3 ed. Porto Alegre: Bookman, 2015. (Coleção Sachaum).

  \item Martins, Márcia da Silva. \textbf{Lógica - Uma abordagem introdutória}. Rio de Janeiro: Editora Ciência Moderna Ltda., 2012.

  \item Souza, João Nunes de. \textbf{Lógica para ciência da computação: uma introdução concisa}. 2 ed. Rio de Janeiro: Elsevier, 2008.

  \item Alencar Filho, Edgard. \textbf{Iniciação à Lógica Matemática}. São Paulo: Nobel, 2002.

  \item Dutra, Maurici José. \textbf{Matemática financeira: livro didático}. 8 ed. rev. e atual. Palhoça: UnisulVirtual, 2010.

  \item Dante, Luiz Roberto. \textbf{Matemática: Ensino médio, volume único - livro do professor}. 1 ed. São Paulo: Editora ática, 2008.

  \end{enumerate}
