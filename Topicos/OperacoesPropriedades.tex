 \chapter{Operações Numéricas e Algébricas}
 
 Antes de tratar das operações numéricas e algébricas, vale ressaltar que quando estamos resolvendo uma expressão numérica ou uma expressão algébrica temos vários cálculos para serem feitos sucessivamente, e para tal precisamos obedecer uma ordem de prioridades que é a seguinte:
 
 \begin{multicols}{2}
 Resolva em: \\
 1º lugar: raízes e potências;\\
 2º lugar: multiplicação e divisão;\\
 3º lugar: adição e subtração.\\
 
 Priorize cálculos em:\\
 1º lugar: parênteses $($ $)$;\\
 2º lugar: colchetes $[$ $]$;\\
 3º lugar: chaves $\{$ $\}$.
 \end{multicols}
 
 \section{Operações em \texorpdfstring{$\Z$}{Z}}
 
 Ao operar neste conjunto numérico precisamos lidar com os números negativos e para isso precisamos dominar os jogos de sinais envolvidos nestas operações, então vamos ver alguns exemplos de operações neste conjunto para entender como lidar com os números negativos.
 
 \textbf{Adição de números inteiros}
 
 \begin{itemize}
  \item Na adição de números inteiros com o mesmo sinal, some os números e conserve o sinal;
  \item Na adição de números inteiros com sinais diferentes, subtraia os números e conserve o sinal do maior.
 \end{itemize}

  \begin{enumerate}[1)]
   \item $123 + 7= 130$
   \item $123 - 7= 116$
   \item $-123 + 7 = -116$
   \item $-123 - 7 = -130$
 \end{enumerate}

 \textbf{Multiplicação e divisão de números inteiros}
 
  \begin{itemize}
   \item Na multiplicação e divisão de números inteiros com o mesmo sinal o resultado é sempre positivo.
   \item Na multiplicação e divisão de números inteiros com o sinais diferentes o resultado é sempre negativo.
  \end{itemize}

  \begin{multicols}{2}
  \begin{enumerate}[1)]
   \item $8 \cdot 20= 160$
   \item $8 \cdot (-7)= -160$
   \item $-8 \cdot 7= -160$
   \item $(-8) \cdot (-7)= 160$
   \item $45 \div 5= 9$
   \item $45 \div (-5)= -9$
   \item $(-45) \div 5= -9$
   \item $(-45) \div (-5)= 9$
  \end{enumerate}
  \end{multicols}


 
 \section{Operações em \texorpdfstring{$\Q$}{Q}}
 
 As operações no conjunto dos números Racionais envolvem em particular as operações com frações que possuem algumas particularidades por isso façamos uma rápida retomada destas operações.
 
 \vskip0.3cm
 
 \colorbox{azul}{
 \begin{minipage}{14.5cm}
 \begin{center}
  \textbf{Soma:} Dados $x, y, a, b \in \Z$ com $a, b \neq 0$ temos:
 \[\frac{x}{a} + \frac{y}{a}= \frac{x+y}{a} \ \ , \text{ ou}, \ \ 
  \frac{x}{a} + \frac{y}{b}= \frac{xb + ya}{ab} \]
 \end{center}
 \end{minipage}}
 
 \vskip0.3cm
  
 \colorbox{azul}{
 \begin{minipage}{14.5cm}
 \begin{center}
  \textbf{Subtração:} Dados $x, y, a, b \in \Z$ com $a, b \neq 0$ temos:
 \[\frac{x}{a} - \frac{y}{a}= \frac{x-y}{a} \ \ , \text{ ou}, \ \
 \frac{x}{a} - \frac{y}{b}= \frac{xb - ya}{ab} \]
 \end{center}
 \end{minipage}}
 
 \vskip0.3cm
 
 \begin{exem}
  \textbf{Soma e subtração de frações com mesmo denominador:}
   
   Quando os denominadores das frações são iguais, mantemos o denominador e operamos os numeradores.
    \vskip0.3cm
   \[\frac{3}{5} + \frac{1}{5}= \frac{3+1}{5}= \frac{4}{5} \ .\]
    \vskip0.3cm
   \[\frac{3}{5} - \frac{1}{5}= \frac{3-1}{5}= \frac{2}{5} \ .\]
 \end{exem}
 
 \begin{exem}
 \textbf{Soma e subtração de frações com denominadores diferentes:}
   
   Quando os denominadores das frações são diferentes podemos simplesmente multiplicar os denominadores ou calcular o mínimo múltiplo comum entre eles (MMC), a vantagem da segunda opção é que o MMC é menor ou igual ao produto, como podemos ver no exemplo:
    \vskip0.3cm
   \[\frac{2}{4} + \frac{3}{10}= \frac{10 \cdot 2 + 4 \cdot 3}{4 \cdot 10}= \frac{20 + 12}{40}= \frac{32}{40}= \frac{4}{5} \ .\]
    \vskip0.3cm
   \[\frac{2}{4} - \frac{3}{10}= \frac{10 \cdot 2 - 4 \cdot 3}{4 \cdot 10}= \frac{20 - 12}{40}= \frac{8}{40}= \frac{1}{5} \ .\]
    \vskip0.3cm
   Observamos que o $MMC(4, 10)= 20$, assim,
    \vskip0.3cm
   \[\frac{2}{4} + \frac{3}{10}= \frac{5 \cdot 2 + 2 \cdot 3}{20}= \frac{10+6}{20}= \frac{16}{20}=\frac{4}{5} \ .\]
    \vskip0.3cm
   \[\frac{2}{4} - \frac{3}{10}= \frac{5 \cdot 2 - 2 \cdot 3}{20}= \frac{10 - 6}{20}= \frac{4}{20}=\frac{1}{5} \ .\]
 \end{exem}

 
 \vskip0.5cm
 
 \colorbox{azul}{
 \begin{minipage}{14.5cm}
 \begin{center}
  \textbf{Multiplicação:} Dados $a, b, c, d \in \Z$ com $b, d \neq 0$ temos:
 \[\frac{a}{b} \cdot \frac{c}{d}= \frac{a \cdot c}{b \cdot d} \]
 \end{center}
 \end{minipage}}
 
 \vskip0.3cm
 \begin{exem}
  \textbf{Multiplicação de fração:} na multiplicação devemos multiplicar numerador por numerador e denominador por denominador.
   \[\frac{2}{3} \cdot \frac{6}{4}= \frac{2 \cdot 6}{3 \cdot 4}= \frac{12}{12}= 1 \]
   \[2 \cdot \frac{5}{3}= \frac{2 \cdot 5}{3}= \frac{10}{3}\]
 \end{exem}

 \vskip0.3cm
 
 \colorbox{azul}{
 \begin{minipage}{14.5cm}
 \begin{center}
  \textbf{Divisão:} Dados $a, b, c, d \in \Z$ com $b, c, d \neq 0$ temos:
 \[\frac{a}{b} \div \frac{c}{d}= \frac{a}{b} \cdot \frac{d}{c} \]
 \end{center}
 \end{minipage}}
 
 \vskip0.3cm
 \begin{exem}
  \textbf{Divisão de fração:} na divisão conservamos a primeira fração e multiplicamos pelo inverso da segunda.
   \[\frac{2}{3} \div \frac{1}{6}= \frac{2}{3} \cdot \frac{6}{1}= \frac{2 \cdot 6}{3 \cdot 1}= \frac{12}{3}= 4 \]
   \[\frac{4}{\left(\frac{2}{3}\right)}= \frac{4}{1} \cdot \frac{3}{2}= \frac{12}{2}=6\]
 \end{exem}

 
 \vskip0.3cm

 
 \section{Potenciação}
 
  \vskip0.3cm

 \textbf{Potência}
 
 \vskip0.3cm
 
 \colorbox{azul}{
 \begin{minipage}{14.5cm}
 \begin{center}
  Dados dois números $a \in \R$ e $b \in \N$ definimos:
 \[a^b= \underbrace{a \cdot a \cdots a}_{b - vezes} .\]
  Dizemos que $a$ é a base da potência e $b$ o expoente. Lê-se: $a$ elevado a $b$.
 \end{center}
 \end{minipage}}
 
 \vskip0.3cm
 
 \begin{exem}
 Observe que neste caso o expoente é um número natural, e portanto positivo, como por exemplo:
 
  $2^3= 2 \cdot 2 \cdot 2= 8$;
  
  $2^4=2 \cdot 2 \cdot 2 \cdot  2= 16$;
  
  $3^2= 3 \cdot 3= 9$;
  
  $5^3= 5 \cdot 5 \cdot 5= 125$.
  
  Este é o único caso em que temos a potência definida, se tivermos qualquer outro número no expoente precisamos fazer recair nesta situação.
 \end{exem}

 
 \vskip0.3cm
 
 \colorbox{azul}{
 \begin{minipage}{14.5cm}
 \begin{center}
   Dados dois números $a \in \R$ e $b \in \Z$ definimos:
 \[a^b= \underbrace{a \cdot a \cdots a}_{b - vezes} \ , \text{ para } b\geq0 \ , \text{ esta situação está inclusa no caso anterior};\]
 \[a^{-b}= \frac{1}{a^b}= \underbrace{\frac{1}{a} \cdot \frac{1}{a} \cdots \frac{1}{a}}_{b - vezes} \ , \text{ para } b>0 ;\]
 \[\frac{1}{a^{-b}}= a^b \ , \text{ para } b>0;\]
 \end{center}
 \end{minipage}}
 
 \vskip0.3cm
 
 \begin{exem}
 Vejamos agora alguns exemplos em que o expoente é um número negativo:
 \begin{eqnarray*}
  2^{-1}= \frac{1}{2^{1}}= \frac{1}{2} \ ; \\
  2^{-3}= \frac{1}{2^3}= \frac{1}{8} \ ; \\
  \frac{1}{3^{-2}}= 3^2= 3 \cdot 3= 9 \ ; \\
  \left( \frac{8}{22} \right)^{-2}= \left( \frac{22}{8} \right)^{2}= \frac{22}{8} \cdot \frac{22}{8}= \frac{484}{64} \ .
 \end{eqnarray*}
 Note que em todos os exemplos acima o que fizemos foi "inverter" a fração, e com isso deixamos os expoentes positivos, e então basta aplicar a definição de potência.

 \end{exem}
 
 \vskip0.3cm

 \textbf{Raiz}
 
 \vskip0.3cm
 
 \colorbox{azul}{
 \begin{minipage}{14.5cm}
 \begin{center}
  Dados dois números $a \in \R$ e $b \in \Q$, logo $b= \frac{m}{n}$ para $n \neq 0$, definimos:
 \[a^b= a^{\frac{m}{n}}= \sqrt[n]{a^m} \ , \text{ para } b>0 ;\]
 \[a^{-b}= \frac{1}{a^{\frac{m}{n}}}= \frac{1}{\sqrt[n]{a^m}} \ ,  \text{ para } b>0.\]
 \end{center}
 \end{minipage}}
 
 \vskip0.3cm
 
 \begin{exem}
  Vejamos agora alguns exemplos de potência com expoente sendo um número racional ($b \in \mathbb{Q}$):
  \begin{eqnarray*}
   4^{\frac{1}{2}}= \sqrt{4}= 2 \\
   8^{\frac{1}{3}}= \sqrt[3]{8^1}= \sqrt[3]{2^{3}}= 2 \\
   27^{\frac{2}{6}}= \sqrt[6]{27^2}= \sqrt[6]{729}= \sqrt[6]{3^6}= 3\\
   9^{-\frac{1}{2}}= \frac{1}{9^{\frac{1}{2}}}= \frac{1}{\sqrt{9}}= \frac{1}{3} \\
   \left(\frac{4}{9}\right)^{-\frac{1}{2}}= \left(\frac{9}{4}\right)^{\frac{1}{2}}= \sqrt{\left(\frac{9}{4}\right)}=\frac{\sqrt{9}}{\sqrt{4}}= \frac{3}{2} \\
   \frac{2}{3^{-2}}= 2 \cdot \frac{1}{3^{-2}}= 2 \cdot 3^{2}= 2 \cdot 9= 18
  \end{eqnarray*}

 \end{exem}

 
 \vskip0.3cm
 
 \colorbox{amarelo}{
 \begin{minipage}{14.5cm}
 \begin{center}
 Aqui é importante observar que:
 \begin{align*} 
 & \nexists 0^0 & & a^1= a, \forall a \in \R & & a^0= 1, \forall a \in \R & & 1^a= 1, \forall a \in \R
 \end{align*}
 \end{center}
 \end{minipage}}
 
 Além disso a operação de potenciação satisfaz as seguintes propriedades:
 
  \begin{table}[H]
 \centering
 \begin{tabular}{|c|c|} \hline
 $a^m \cdot a^n= a^{m + n}$ & $a^m \div a^n= a^{m - n}$ \\ \hline
 $(a^m)^n= a^{m \cdot n}$ & $(a \cdot b)^n= a^n \cdot b^n$ \\ \hline
 $\left(\frac{a}{b}\right)^n= \frac{a^n}{b^n}$ & $a^{-n}= \frac{1}{a^n}$ $(a \neq 0)$ \\ \hline
 $\left(\frac{a}{b} \right)^{-n}= \left(\frac{b}{a} \right)^{n}$ & $a^{\frac{m}{n}}= \sqrt[n]{a^m}$ \\\hline
 \end{tabular}
 \end{table}
 
 \section{Operações Algébricas}
 
 Expressões algébricas são expressões matemáticas que envolvem números, letras e operações. 
 
 Como por exemplo: 
 
 \begin{eqnarray*}
  2x=4 \ ,\\
  x^2+1=0 \ ,\\
  x(x+3)=5 \ ,\\
  2x+3y=17 \ ,\\
  x^2 + 2y + 3z -4= 52 \ , \\
  \frac{14x + 8y}{2x}= 3 \ , \\
  \frac{2}{5}x^3 + 3\sqrt{x^4}= 67 \ , \\
  5x(x+3)-4x(2-x)=7 \ .
 \end{eqnarray*}
 
 Nestas expressões as letras que aparecem são chamadas de \textbf{variáveis}, e os números que aparecem multiplicando uma letra são chamados de \textbf{coeficientes}.
 
 As expressões algébricas são utilizadas dentre outras coisas, para descrever uma situação problema na qual não conhecemos todos os valores envolvidos, representar uma fórmula, ou expressar uma equação. Devido a sua importância nas exatas precisamos compreender como se comportam as operações presentes nas expressões algébricas, em outras palavras, como fazer contas com letras.
 
 \vskip0.3cm 
 
 \textbf{Adição e subtração}
 
 Podemos somar somente letras iguais e com mesmo expoente. Como por exemplo:

 \begin{itemize}
  \item $2x + x= (2+1)x= 3x$
  \item $x^2 - 3x^2= (1-3)x^2= -2x^2$
  \item $2x + y + 5x^2 + 7y - 3x= 5x^2 + (2-3)x + (1+7)y= 5x^2 - 1x + 8y$
  \item $3(x+ 4y-2)= 3x + 3.4y - 3.2= 3x + 12y - 6$
 \end{itemize}
 
  \vskip0.3cm 
 
 \textbf{Multiplicação}
 
 Na multiplicação devemos sempre multiplicar coeficiente por coeficiente e letra por letra. Sendo que no caso das letras serem iguais, devemos manter a letra e somar seus expoentes, e no caso das letras serem diferentes apenas fazemos a associação das duas letras. Como mostram os seguintes exemplos:
 
  \begin{itemize}
   \item $x \cdot x = x^{1+1}= x^2$
   \item $x \cdot x^2= x^{1+2}= x^3$
   \item $x \cdot 2y= (1 \cdot 2)xy= 2xy$
   \item $3x \cdot 2x^2y= (3 \cdot 2)x^{1+2}y= 6x^3y$
   \item $4x^4 \cdot \frac{1}{2}x^{-2}= (4 \cdot \frac{1}{2})x^{4-2}= 2x^2$
  \end{itemize}

  \vskip0.3cm 
  
   \textbf{Divisão}
   
   Na divisão devemos sempre dividir coeficiente por coeficiente e letra por letra. Sendo que no caso das letras serem iguais, devemos manter a letra e subtrair seus expoentes, e no caso das letras serem diferentes apenas fazemos a associação das duas letras. Como mostram os seguintes exemplos:
   
  \begin{itemize}
   \item $x \div x= x^{1-1}= x^0= 1$
   \item $x \div x^2= x^{1-2}= x^{-1}= \frac{1}{x}$
   \item $2y \div x= 2\frac{y}{x}$
   \item $4y^3 \div 2y^2= \frac{4}{2} \cdot \frac{y^3}{y^2}= 2y^{3-2}= 2y$
   \item $\frac{x^2yz^3}{x^2y^3z^2}= x^{2-2}y^{1-3}z^{3-2}= x^0 y^{-2}z^{1}= \frac{z}{y^2}$  
  \end{itemize}
  
 \vskip0.3cm 
  
  \textbf{Potenciação}
  
  Na potenciação devemos aplicar o expoente ao coeficiente e à incógnita, obedecendo as propriedades de potência.
  
    \begin{itemize}
     \item $(2x)^2= 2^2 \cdot x^2= 2x^2$
     \item $(3x^2)^3= 3^3 \cdot x^{2\cdot 3}= 27x^6$
     \item $\left(\frac{3a^2}{4}\right)^2= \frac{3^2 a^{2 \cdot 2}}{4^2}= \frac{9a^4}{16}$
    \end{itemize}

  \vskip0.3cm 
  
  \textbf{Radiciação}

  Na radiciação devemos extrair a raiz do coeficiente e da incógnita. Observamos que extrair a raiz da incógnita é equivalente a dividir seu expoente pelo índice da raiz.
    \begin{itemize}
     \item $\sqrt{x}= x^{\frac{1}{2}}$
     \item $\sqrt{x^4}= (x^4)^{\frac{1}{2}}= x^{\frac{4}{2}}= x^2$
     \item $\sqrt[3]{8x^6}= \sqrt[3]{8} \cdot \sqrt[3]{x^6}= 2 x^{\frac{6}{3}}= 2x^2$
     \item $\sqrt{\frac{2x^2}{16}}= \frac{\sqrt{2x^2}}{\sqrt{16}}= \frac{\sqrt{2}x}{4}$
    \end{itemize}

 \vskip0.3cm 

  \textbf{Fatoração das expressões algébricas}
  
 \vskip0.3cm 
 
 A fatoração das expressões algébricas, é o que nos permite escrever a expressão como um produto de dois termos, ela é utilizada principalmente na resolução de equações, para acelerar o processo de resolução.
 
 Os seguintes casos de fatoração são os mais utilizados:
 \begin{itemize} 
  \item Fator em comum: $x^2 + x= x(x + 1)$; $4x^2 + 6= 2(2x^2 + 3)$
  \item Agrupamento: $ax + bx + ay + by= (a+b)x+(a+b)y= (a+b)(x+y)$
  \item Trinômio quadrado perfeito (+): $(a + b)^2= a^2 + 2ab + b^2$ 
  \item Trinômio quadrado perfeito (-): $(a - b)^2= a^2 - 2ab + b^2$
  \item Diferença de dois quadrados: $(a + b) \cdot (a - b)= a^2 - b^2$
  \item Cubo perfeito (+): $(a+b)^3= a^3 + 3a^2b + 3ab^2 + b^3$
  \item Cubo perfeito (-): $(a-b)^3= a^3 - 3a^2b + 3ab^2 - b^3$
 \end{itemize}
   
