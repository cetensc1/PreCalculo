\chapter{Só para garantir!}
 Caso não lembre fique a vontade para colar:
 \section{Tabuada}

 \begin{align*}
 & 1 \times 1= 1 & 2 \times 1= 2 & & 3 \times 1= 3 & & 4 \times 1= 4 & & 5 \times 1= 5 &\\
 & 1 \times 2= 2 & 2 \times 2= 4 & & 3 \times 2= 6 & & 4 \times 2= 8 & & 5 \times 2= 10 &\\
 & 1 \times 3= 3 & 2 \times 3= 6 & & 3 \times 3= 9 & & 4 \times 3= 12 & & 5 \times 3= 15 &\\
 & 1 \times 4= 4 & 2 \times 4= 8 & & 3 \times 4= 12 & & 4 \times 4= 16 & & 5 \times 4= 20 &\\
 & 1 \times 5= 5 & 2 \times 5= 10 & & 3 \times 5= 15 & & 4 \times 5= 20 & & 5 \times 5= 25 &\\
 & 1 \times 6= 6 & 2 \times 6= 12 & & 3 \times 6= 18 & & 4 \times 6= 24 & & 5 \times 6= 30 &\\
 & 1 \times 7= 7 & 2 \times 7= 14 & & 3 \times 7= 21 & & 4 \times 7= 28 & & 5 \times 7= 35 &\\
 & 1 \times 8= 8 & 2 \times 8= 16 & & 3 \times 8= 24 & & 4 \times 8= 32 & & 5 \times 8= 40 &\\
 & 1 \times 9= 9 & 2 \times 9= 18 & & 3 \times 9= 27 & & 4 \times 9= 36 & & 5 \times 9= 45 &\\
 & 1 \times 10= 10 & 2 \times 10= 20 & & 3 \times 10= 30 & & 4 \times 10= 40 & & 5 \times 10= 50 &
 \end{align*}

 \begin{align*}
 & 6 \times 1= 6 & 7 \times 1= 7 & & 8 \times 1= 8 & & 9 \times 1= 9 & & 10 \times 1= 10 &\\
 & 6 \times 2= 12 & 7 \times 2= 14 & & 8 \times 2= 16 & & 9 \times 2= 18 & & 10 \times 2= 20 &\\
 & 6 \times 3= 18 & 7 \times 3= 21 & & 8 \times 3= 24 & & 9 \times 3= 27 & & 10 \times 3= 30 &\\
 & 6 \times 4= 24 & 7 \times 4= 28 & & 8 \times 4= 32 & & 9 \times 4= 36 & & 10 \times 4= 40 &\\
 & 6 \times 5= 30 & 7 \times 5= 35 & & 8 \times 5= 40 & & 9 \times 5= 45 & & 10 \times 5= 50 &\\
 & 6 \times 6= 36 & 7 \times 6= 42 & & 8 \times 6= 48 & & 9 \times 6= 54 & & 10 \times 6= 60 &\\
 & 6 \times 7= 42 & 7 \times 7= 49 & & 8 \times 7= 56 & & 9 \times 7= 63 & & 10 \times 7= 70 &\\
 & 6 \times 8= 48 & 7 \times 8= 56 & & 8 \times 8= 64 & & 9 \times 8= 72 & & 10 \times 8= 80 &\\
 & 6 \times 9= 54 & 7 \times 9= 63 & & 8 \times 9= 72 & & 9 \times 9= 81 & & 10 \times 9= 90 &\\
 & 6 \times 10= 60 & 7 \times 10= 70 & & 8 \times 10= 80 & & 9 \times 10= 90 & & 10 \times 10= 100 &
 \end{align*}

 \newpage
 \section{Regras de Divisibilidade}
 \begin{itemize}
  \item \textbf{Divisibilidade por $1$}

 Todo número é divisível por $1$.

 \item \textbf{Divisibilidade por $2$}

 Um número é divisível por $2$ (ou seja, par) quando o seu dígito das unidades é igual a $0$, $2$, $4$, $6$ ou $8$.

 \item \textbf{Divisibilidade por $3$}

 Um número é divisível por $3$ se a soma de seus dígitos é um número múltiplo de $3$.

 \item \textbf{Divisibilidade por $4$}

 Um número é divisível por $4$ quando é par e a metade do último dígito adicionado ao penúltimo é um número par ou termina com zero nas duas últimas casas.

 \item \textbf{Divisibilidade por $5$}

 Um número é divisível por $5$ quando termina em $0$ ou $5$.

 \item \textbf{Divisibilidade por $6$}

 Um número é divisível por $6$ quando é divisível por $2$ e $3$.

 \item \textbf{Divisibilidade por $7$}

 Um número é divisível por $7$ quando estabelecida a diferença entre o dobro do último e os demais dígitos, esta é um número divisível por $7$.

 \item \textbf{Divisibilidade por $8$}

 Um número é divisível por $8$ quando termina em $000$ ou o número formado por seus três últimos dígitos é divisível por $8$.

 \item \textbf{Divisibilidade por $9$}

 É divisível por $9$ todo número em que a soma de seus dígitos constitui um número múltiplo de $9$.

 \item \textbf{Divisibilidade por $10$}

 Um número é divisível por $10$ quando terminar em $0$.

 \item \textbf{Divisibilidade por $12$}

 Um número é divisível por $12$ quando é divisível por $3$ e $4$.

 \item \textbf{Divisibilidade por $15$}

 Um número é divisível por $15$ quando é divisível por $3$ e $5$.
 \end{itemize}

  \newpage
 \section{Símbolos matemáticos}

  \begin{table}[H]
 \centering
 \begin{tabular}{|c|c|} \hline
 \rowcolor{cinza}
 Símbolos & Significado \\\hline
 $\forall$ & Para todo \\\hline
 $\exists$ & Existe \\\hline
 $\nexists$ & Não existe \\\hline
 $\in$ & Pertence \\\hline
 $\notin$ & Não pertence \\\hline
 $\infty$ & Infinito \\\hline
 $\emptyset$ & Vazio \\\hline
 $=$ & Igual \\\hline
 $\neq$ & Diferente \\\hline
 $<$ & Menor \\\hline
 $\leq$ & Menor ou igual \\\hline
 $>$ & Maior \\\hline
 $\geq$ & Maior ou igual \\\hline
 $\subset$ & Contido \\\hline
 $\supset$ & Contém \\\hline
 $\subseteq$ & Contido e pode ser igual \\\hline
 $\supseteq$ & Contém e pode ser igual \\\hline
 $\cup$ & União \\\hline
 $\cap$ & Interseção \\\hline
 $\N$ & Números Naturais \\\hline
 $\Z$ & Números Inteiros \\\hline
 $\Q$ & Números Racionais \\\hline
 $\R$ & Números Reais \\\hline
 $\C$ & Números Complexos \\\hline

 \end{tabular}
 \end{table}
