\chapter{Razão e proporção}

\section{Razão}

A razão é utilizada para comparar duas grandezas, esta comparação é feita através do quociente entre as grandezas. Dadas duas grandezas $a$ e $b$ quaisquer, com $b \neq 0$, o quociente entre $a$ e $b$ é o resultado da divisão de $a$ por $b$, que pode ser representado pela fração $(a/b)$, ou pela divisão $a:b$. Normalmente é denotada por razão de $a$ para $b$.

O próximo exemplo mostra como podemos utilizar a teoria de razão para melhor avaliar como iremos investir nosso dinheiro.

\begin{exem}
 Maria Clara vendeu seu apartamento e aplicou R\$ 8000,00 numa caderneta de poupança, ao final de um ano este valor rendeu R\$ 960,00. No mesmo período, ela aplicou R\$ 5000,00 num fundo de investimento que rendeu R\$ 800,00. Qual das duas aplicações teve maior rentabilidade?

 \underline{Resolução:}

 Em termos absolutos, o rendimento da caderneta foi maior.

 Em termos relativos:

 a rentabilidade da caderneta foi de $\frac{960}{8000}= \frac{12}{100}= 12 \%$,

 e a do fundo foi de $\frac{800}{5000}=\frac{16}{100}= 16 \%$.

 Portanto a rentabilidade do fundo foi maior.

 \fim
\end{exem}

A seguir temos um exemplo de como a teoria de razão poder ser cobrada em concurso, e de como ela aparece em seu dia-a-dia.

\begin{exem}[Comperve - 2018]
 Um idoso foi a uma farmácia com a prescrição de um medicamento da marca $X$ cuja caixa com $30$ comprimidos custa $R\$ 60,00$. O farmacêutico, então, lhe apresentou a opção de um medicamento similar da marca $A$ cuja caixa com $20$ comprimidos custa $R\$ 35,00$. Havia também um medicamento da marca $B$, com mesmo princípio ativo, no valor de $R\$ 25,00$ e cuja caixa contém $15$ comprimidos. Em relação à essas opções de compra, conclui-se que

 \begin{enumerate}[a)]
  \item a caixa do medicamento da marca B é a que apresenta o menor valor por comprimido.
  \item a caixa do medicamento da marca A é a que apresenta maior valor por comprimido.
  \item o valor de cada comprimido é o mesmo independente da escolha da marca.
  \item cada comprimido do medicamento da marca A custa o dobro do comprimido da marca B.
 \end{enumerate}

\underline{Resolução:}

  Note que o valor de cada comprimido da caixa $(V_c)$ é dado pela razão:
  \[V_c= \frac{\text{valor da caixa}}{\text{quantidade de comprimidos}} ,\]
  consideremos então as seguintes razões:

  para a marca $X$ temos:
  \[V_{c_x}=\frac{\text{valor da caixa da marca X}}{\text{quantidade de comprimidos da marca X}}= \frac{60}{30}= 2 ,\]

  para a marca $A$ temos:
   \[V_{c_A}=\frac{\text{valor da caixa da marca A}}{\text{quantidade de comprimidos da marca A}}= \frac{35}{20}= \frac{7}{4}= 1,75 ,\]

  para a marca $B$ temos:
  \[V_{c_B}=\frac{\text{valor da caixa da marca B}}{\text{quantidade de comprimidos da marca B}} = \frac{25}{15}= \frac{5}{3}= 1,66 \cdots .\]
  Portanto, cada comprimido da marca $X$ custa $R\$ 2,00$, cada comprimido da marca $A$ custa $R\$ 1,75$ e cada comprimido da marca $B$ custa $R\$ 1,67$, donde concluímos que a resposta da questão é o item a).

  \fim
\end{exem}



\subsection{Algumas razões especiais}

\textbf{Escala:} Quando construímos mapas ou plantas baixas de lugares sempre representamos as medidas das distâncias em \textit{escala} menor que a real. A escala é por definição dada pela seguinte razão:

\[\text{Escala}= \frac{\text{Medida no mapa}}{\text{Medida real}} \ \ \ \text{(ambos na mesma unidade de medida)}\]

\textbf{Velocidade Média:} É a razão entre a distância percorrida e o tempo total de percurso. A velocidade média será sempre acompanhada de uma unidade, que depende das unidades escolhidas para calcular distância e tempo. Por convenção, no sistema internacional de medidas, para o cálculo da velocidade sempre trabalhamos com distância em quilômetros $(km)$ ou metros $(m)$ e com tempo em horas $(h)$ ou segundos $(s)$, assim as unidades de medida para a velocidade média são $km/h$ ou $m/s$. A velocidade média é dada pela razão:

\[\text{Velocidade média}= \frac{\text{Distância percorrida}}{\text{Tempo total de percurso}}\]

\textbf{Densidade de um corpo:} A densidade de um corpo é a razão entre a sua massa e o seu volume. A densidade também será sempre acompanhada de uma unidade, que depende das unidades escolhidas para medir a massa e o volume. Alguns exemplos de unidades para a densidades são $g/cm^3$, $kg/m^3$ etc.

\[\text{Densidade de um corpo}= \frac{\text{Massa do corpo}}{\text{Volume do corpo}}\]

\textbf{Densidade demográfica:} A densidade demográfica é a razão entre o número de habitantes de uma região e sua área.

\[\text{Densidade demográfica}= \frac{\text{Número de habitantes}}{\text{Área}}\]

Por exemplo, a densidade demográfica do Brasil segundo a Wikipédia é de 23,8 habitantes por quilômetro quadrado $(km^2)$.

\textbf{Relação candidato/vaga:} Está presente em todos os concursos e vestibulares e dá aos candidatos uma ideia do quão concorrido está o concurso em questão, esta relação que norteia os candidatos, dando a eles uma ideia do quanto precisam estudar é dada pela seguinte razão:

\[\text{Relação candidato/vaga}= \frac{\text{Número de candidatos}}{\text{Número de vagas}}\]

\begin{exem}
 Após uma pesquisa rápida no Google sobre os concursos mais concorridos me deparei com o concurso de 2012 do Departamento da Polícia Federal para Agente da Polícia Federal, o qual possuía 500 vagas e contou com 107799 inscritos tendo portanto a seguinte relação candidato/vaga:
 \[\text{Relação candidato/vaga}= \frac{\text{Número de candidatos}}{\text{Número de vagas}}= \frac{107799}{500}= 215,598\]
 ou seja, tinham aproximadamente $215,6$ pessoas concorrendo a uma vaga neste concurso.

 \fim
\end{exem}


\begin{exem}[FGV - 2014]
Certa rua da cidade de João Pessoa tem $450 m$ de comprimento e aparece em um mapa com comprimento de $3 cm$. A escala desse mapa é:
\begin{multicols}{5}
\begin{enumerate}[a)]
 \item $1:15$
 \item $1:150$
 \item $1:1500$
 \item $1:15000$
 \item $1:150000$
\end{enumerate}
\end{multicols}
\underline{Resolução:}

O primeiro passo da resolução desta questão é transformar os $450m$ em centímetros, lembramos que um metro possui 100cm, logo $450 m=450*100 cm= 45000 cm$.

Como $450 m$ de comprimento é representado em um mapa com comprimento de $3 cm$, temos que neste mapa a escala é de:
\[\frac{3}{45000}= \frac{1}{15000} \Rightarrow 1:15000 .\]

 \fim
\end{exem}


\section{Proporção}

A proporção é uma igualdade entre duas razões (ou equivalência entre razões). Ou seja, se dissermos que as razões
\[\frac{a}{b}= \frac{c}{d}\]
então estamos dizendo que elas são proporcionais, ou formam uma proporção.

\vskip0.3cm

\colorbox{azul}{
 \begin{minipage}{0.9\linewidth}
 \begin{center}
 \textbf{Propriedade fundamental da proporção}

   O produto dos meios é igual ao produto dos extremos. Assim,
 \[\frac{a}{b}= \frac{c}{d} \Leftrightarrow ad= bc\]
 pois neste caso, $a$ e $d$ são os extremos da proporção e $b$ e $c$ são os meios da proporção.
 \end{center}
 \end{minipage}}

 \vskip0.3cm

É importante ressaltar que para usar a proporção as grandezas $a$ e $b$ precisam estar dadas na mesma unidade de medida, assim como as grandezas $c$ e $d$.

\begin{exem}[VUNESP - 2012]
 Francisco nasceu com 48 cm e agora está com 1,92 m. Seguindo a mesma proporção, caso ele tivesse nascido com 52 cm, hoje ele teria, em metros,
 \begin{multicols}{5}
 \begin{enumerate}[a)]
  \item 2,00
  \item 2,04
  \item 2,08
  \item 2,10
  \item 2,12
 \end{enumerate}
 \end{multicols}

\underline{Resolução:}

A razão entre as medidas de nascimento em centímetros é $\frac{48}{52}$, nesta ordem a razão entre as medidas atuais em metros é $\frac{1,92}{x}$, como estamos trabalhando com a mesma proporção temos que:
\[\frac{48}{52} = \frac{1,92}{x}\]
assim aplicando a propriedade fundamental da proporção obtemos que $48 \cdot x= 52 \cdot 1,92 \Rightarrow x= 2,08 m$. Portanto, a resposta á a letra c).

\fim
\end{exem}


\chapter{Regra de três}

A regra de três é uma importante ferramenta na resolução de problemas envolvendo grandezas diretamente proporcionais e/ou inversamente proporcionais, um exemplo de aplicação da regra de três é como já vimos anteriormente a resolução de problemas envolvendo a porcentagem.

Antes de entendermos como funciona a regra de três, vejamos algumas definições importantes:

\vskip0.3cm

\colorbox{azul}{
 \begin{minipage}{0.9\linewidth}
 \begin{center}
 \textbf{Grandezas Diretamente Proporcionais}

   Duas grandezas são diretamente proporcionais quando, o aumento de uma implica no aumento da outra na mesma proporção.
 \end{center}
 \end{minipage}}

 \vskip0.3cm

\colorbox{azul}{
 \begin{minipage}{0.9\linewidth}
 \begin{center}
 \textbf{Grandezas Inversamente Proporcionais}

   Duas grandezas são inversamente proporcionais quando, o aumento de uma implica na redução da outra na mesma proporção.
 \end{center}
 \end{minipage}}

 \vskip0.3cm


\section{Regra de 3 simples}

A regra de três simples é uma ferramenta utilizada na resolução de problemas envolvendo duas grandezas proporcionais.

A regra de três simples pode ser de duas maneiras: direta, quando as grandezas envolvidas são diretamente proporcionais; inversa, quando as grandezas envolvidas são inversamente proporcionais.

Para resolver problemas envolvendo regra de três, costumamos separar os dados do problema em colunas, sendo uma coluna para cada grandeza envolvida no problema. Portanto, no caso da regra de três simples, teremos então duas colunas uma para cada grandeza, ou equivalentemente, uma para cada razão. Para auxiliar na interpretação ao lado de cada coluna colocamos flechas que indicam se a grandeza esta aumentando (flecha para cima) ou diminuindo (flecha para baixo), as duas flechas para o mesmo lado indicam que as grandezas são diretamente proporcionais e, flechas para lados contrários indicam que as grandezas são inversamente proporcionais.

\subsection{Regra de 3 simples e direta}

 A regra de 3 simples e direta é como visto acima uma regra de três simples na qual as grandezas envolvidas são diretamente proporcionais (as duas flechas estão para o mesmo lado). Vejamos um exemplo de aplicação da regra de três simples e direta.

\begin{exem}
\begin{enumerate}
 Com a cana-de-açúcar é possível obter diversos produtos, entre eles o álcool combustível. Para obter $75$L de álcool combustível são necessários, aproximadamente, $1250$Kg de cana-de-açúcar.
 \begin{enumerate}
  \item Quantos litros de álcool combustível, no máximo, podem ser produzidos com $2000$Kg de cana-de-açúcar?

 \underline{Resolução:}

 \begin{eqnarray*}
  \text{Kg de cana} & & \text{Litros de combustível} \\
  \uparrow 1250 & = & 75 \uparrow \\
  2000 & = & x
 \end{eqnarray*}

 $\frac{1250}{2000}= \frac{75}{x} \Rightarrow 1250 x = 75*2000 \Rightarrow 1250 x = 150000$

 $\Rightarrow x = \frac{150000}{1250} \Rightarrow x = 120 \text{ L}$
 \fim


  \item No mínimo, quantos quilogramas de cana-de-açúcar são necessários para produzir $45$L de álcool combustível?

  \underline{Resolução:}
  \begin{eqnarray*}
  \text{Kg de cana} & & \text{Litros de combustível} \\
  \downarrow 1250 & = & 75 \downarrow \\
  x & = & 45
 \end{eqnarray*}

 $\frac{1250}{x}=\frac{75}{45} \Rightarrow 75 x = 1250*45 \Rightarrow 75 x = 56250 \Rightarrow $

 $x = \frac{56250}{75} \Rightarrow x = 750 \text{ Kg de cana-de-açúcar}$
 \fim

 \end{enumerate}
 \end{enumerate}
\end{exem}

\begin{obs}
 Porcentagem, como vimos anteriormente, é um exemplo de regra de três simples, já que as grandezas envolvidas são diretamente proporcionais.
\end{obs}


\subsection{Regra de 3 simples e inversa}

A regra de 3 simples e inversa é como já vimos uma regra de três simples na qual as grandezas envolvidas são inversamente proporcionais (as duas flechas estão para lados contrários). Vejamos um problema no qual precisamos usar regra de três simples e inversa para resolver.

\begin{exem}
\begin{enumerate}
  Marilda recebeu uma encomenda para preparar $1850$ salgados. Trabalhando ela e mais $3$ funcionários, foi possível terminar a encomenda em $3$ dias.
  \begin{enumerate}
  \item Para que Marilda terminasse a encomenda em $1$ dia, quantos funcionários a mais ela deveria contratar?

  \underline{Resolução:}
  \begin{eqnarray*}
   \text{Funcionários} & & \text{Dias trabalhados} \\
   \uparrow 4 & = & 3 \downarrow \\
   4 + x & = & 1
  \end{eqnarray*}
  $\frac{4}{4+x}= \frac{1}{3} \Rightarrow 4*3=1*(4 + x) \Rightarrow 4 + x = 12 \Rightarrow x = 12 - 4 \Rightarrow x= 8$
  \fim

  \item Se Marilda contratasse mais $2$ funcionários, em quantos dias ficaria pronta a encomenda?

  \underline{Resolução:}
  \begin{eqnarray*}
   \text{Funcionários} & & \text{Dias trabalhados} \\
   \uparrow 4 & = & 3 \downarrow \\
   4 + 2 & = & x
  \end{eqnarray*}
  $\frac{6}{4}= \frac{3}{x} \Rightarrow 4*3=6*x \Rightarrow 12 = 6x \Rightarrow x=\frac{12}{6} \Rightarrow x = 2$
  \fim

  \end{enumerate}
\end{enumerate}
\end{exem}

\begin{exem}
 (UEM - 2018) Luiz corre a uma velocidade de 6km/h e percorre certa distância em 5 minutos. Se ele corresse a 10km/h, a mesma distância seria percorrida em quanto tempo?
 \begin{enumerate}[a)]
  \item 30 min
  \item 2,5 min
  \item 3,6 min
  \item 3 min (*)
  \item 0,3 min
 \end{enumerate}

 \underline{Resolução:}
 Passo 1: transformar 5 min em horas.
  \begin{eqnarray*}
   \text{h} & & \text{min} \\
    \downarrow 1 & = & 60 \downarrow \\
   x & = & 5
  \end{eqnarray*}
  \[60x=5 \Rightarrow x=\frac{5}{60} \Rightarrow x= 0,083 h\]

 Passo 2: usar o tempo em horas para resolver o exercício.
   \begin{eqnarray*}
   \text{km/h} & & \text{horas} \\
   \uparrow 6 & = & 0,083 \downarrow \\
   10 & = & x
  \end{eqnarray*}
  \[\frac{6}{10}=\frac{x}{0,083}\]
  \[10x=6*0,083 \Rightarrow x=\frac{0,498}{10} \Rightarrow x= 0,0498 h\]

 Passo 3: transformar a resposta encontrada em horas para minutos.
  \begin{eqnarray*}
   \text{h} & & \text{min} \\
    \downarrow 1 & = & 60 \downarrow \\
   0,0498 & = & x
  \end{eqnarray*}
  \[x= 60* 0,0498 \Rightarrow x= 2,98 min \approx 3 min\]
  Portanto a resposta é a letra d).

\end{exem}


\section{Regra de 3 composta}

A regra de três composta é uma ferramenta utilizada na resolução de problemas envolvendo mais de duas grandezas direta ou inversamente proporcionais.

\begin{exem}
  (Fepese-2018) Se três pessoas fazem 72 peças de sushi a cada 2 horas, quantas pessoas são necessárias para fazer 252 peças de sushi a cada 1 hora e meia?
  \begin{enumerate}[a)]
  \item 12
  \item 13
  \item 14 *
  \item 15
  \item 18
  \end{enumerate}

  \underline{Resolução:}
  Primeiramente vamos construir uma ``tabela'' com uma coluna para cada grandeza, e em cada linha colocamos as grandezas que se relacionam, assim obtemos:
  \begin{eqnarray*}
   \text{Pessoas} & \text{Peças de sushi} & \text{Horas} \\
   3  & 72 & 2  \\
   x  & 252 & 1,5
  \end{eqnarray*}
  Agora vamos comparar cada uma das colunas com a coluna das pessoas que é onde temos o $x$, assim ficamos com:

  Se aumentamos o número de peças de sushi que queremos produzir em um tempo fixo, precisamos aumentar o número de pessoas que irão trabalhar, logo as duas setas são para cima:
  \begin{eqnarray*}
  \text{Pessoas} & & \text{Peças de sushi}  \\
   \uparrow 3 &  & 72 \uparrow  \\
   x &  & 252
  \end{eqnarray*}

  Agora fixando que o número de pessoas irá aumentar, precisamos de menos tempo para produzir uma quantidade pré-fixada de peças de sushi, logo a quantidade de horas diminui,
  \begin{eqnarray*}
   \text{Pessoas} & & \text{Horas} \\
   \uparrow3 & & 2 \downarrow  \\
   x & & 1,5
  \end{eqnarray*}

  Agora para montar a proporção precisamos fixar a posição da proporção das pessoas, que é a que tem o $x$, e igualar ao produto das outras duas, colocando-as de forma que as setas fiquem para cima, por isso precisamos inverter a proporção das horas, chegando a seguinte proporção:
  \begin{align*}
   \frac{3}{x}= \frac{72}{252} \cdot \frac{1,5}{2}
   & \Rightarrow \frac{3}{x}=\frac{72\cdot 1,5}{252 \cdot 2} \Rightarrow \frac{3}{x}=\frac{108}{504} \Rightarrow 108 x= 3 \cdot 504 \\
   & \Rightarrow x=\frac{1512}{108} \Rightarrow x=14.
  \end{align*}

  Portanto a resposta é a letra c, 14 pessoas.
  \fim
\end{exem}

Todas as situações problemas envolvendo regra de três composta seguem os mesmos passos deste exemplo.


\chapter{Porcentagem}

Percentagem (português europeu) ou Porcentagem (português brasileiro) (do latim \textit{per centum}, significando ``por cento'', ``a cada centena'') é uma medida de razão com base $100$ (cem), também conhecida como \textit{razão centesimal}. É um modo de expressar uma proporção ou uma relação entre $2$ valores (um é a parte e o outro é o inteiro) a partir de uma fração cujo denominador $100$ (cem) representa o todo, ou seja, é dividir um número por $100$ (cem). A porcentagem é representada pelo símbolo $\%$ (lê-se: por cento).

Por exemplo, $5$ por cento, é representado por:
\[ 5\%= \frac{5}{100}= 0,05 .\]

As razões centesimais podem ser representadas de três formas, como mostra a tabela a seguir:

\begin{table}[h]
\centering
 \begin{tabular}{|c|c|c|} \hline
  {\textbf{Forma percentual}} & {\textbf{Razão/ Fração}} & \textbf{Forma decimal} \\ \hline
 $30\%$ & $\frac{30}{100}$ & $0,3$ \\\hline
 $50\%$ & $\frac{50}{100} = \frac{1}{2}$ & $0,5$ \\\hline
 $5\%$ & $\frac{5}{100}=\frac{1}{20}$ & $0,05$ \\\hline
 \end{tabular}
\end{table}

As razões centesimais são úteis para calcular uma determinada porcentagem dada de um todo já conhecido, como no seguinte exemplo:
\begin{exem}
 Qual é o valor de $45 \%$ de R\$ $1022,00$?

\[45 \% \text{ de } 1022,00= \frac{45}{100} \cdot 1022= \frac{45\cdot 1022}{100}= 459,90\]
\fim
\end{exem}


Mas dado uma grandeza $b$, que represente o todo, o cálculo do valor $a$ equivalente a $x$ por cento de $b$ é feito utilizando proporção. Neste caso, temos a seguinte equivalência de razões:
\[\frac{a}{b}= \frac{x}{100}\]
na qual aplicamos a propriedade fundamental da proporção, ou equivalentemente, a regra de três simples direta, como no seguinte exemplo:


\begin{exem}
 A professora Sandra possui $40$ alunos. Uma enquete apontou que $30$ destes alunos gostam de esportes. Qual é a porcentagem de alunos que gostam de esportes?

 \underline{Resolução:}
 Observe que neste exercício o todo é equivalente a $40$ alunos, portanto neste caso $100 \%= 40$ alunos. Assim temos que as seguintes razões são equivalentes:
  \[\frac{30}{40}= \frac{x}{100}\]
  e portanto aplicando a propriedade fundamental da proporção obtemos:
  \[40 x = 30*100 \Rightarrow 40 x = 3000 \Rightarrow x = \frac{3000}{40} \Rightarrow x=  75 \% \text{ dos alunos}.\]

 Este exercício pode ser resolvido usando regra de três, e neste caso a resolução será a seguinte:

  \begin{eqnarray*}
  \text{Alunos} & & \text{Porcentagem} \\
  \downarrow 40 & = & 100 \downarrow \\
  30 & = & x
 \end{eqnarray*}
 $40 x = 30*100 \Rightarrow 40 x = 3000 \Rightarrow x = \frac{3000}{40} \Rightarrow x=  75 \% \text{ dos alunos}$

 Perceba que as duas resoluções são iguais, logo a regra de três simples e direta é apenas uma aplicação da propriedade fundamental da proporção.
 \fim

\end{exem}

Vejamos mais um exemplo de uso da porcentagem.

\begin{exem}
 Em uma partida de basquete, Oscar acertou $80\%$ dos $50$ arremessos que realizou. Quantos arremessos ele acertou?

 \underline{Resolução:}
 Observe que neste exercício o todo é equivalente a $50$ arremessos, portanto neste caso $100 \%= 50$ arremessos.
  \begin{eqnarray*}
  \text{Arremessos} & & \text{Porcentagem} \\
  \downarrow 50 & = & 100 \downarrow \\
  x & = & 80
 \end{eqnarray*}
 $100 x = 50*80 \Rightarrow 100 x = 4000 \Rightarrow x = \frac{4000}{100} \Rightarrow x= 40 \text{ arremessos}$
 \fim

\end{exem}

Vale observar que a porcentagem está presente na matemática financeira (área da matemática dedicada ao mundo das finanças), na qual a porcentagem é utilizada com mais frequência para o cálculo de descontos, juros e lucros, estando assim presente em todas as transações financeiras de nosso dia-a-dia, e por isso é tão importante.


\newpage
