\section{Questões}

 \begin{enumerate}
  \item (FGV - 2017) Manoel cria coelhos e seus animais ou são brancos ou são marrons. Do total dos 120 coelhos que possui, 63 são fêmeas, 50 são marrons e, dos machos, 32 são brancos. O número de fêmeas marrons é:
  \begin{multicols}{5}
  \begin{enumerate}[a)]
  \item 25;
  \item 27;
  \item 29;
  \item 31;
  \item 33.
  \end{enumerate}
  \end{multicols}
  
  \item (IESES - 2017)  Sabendo que temos os conjuntos $A$ e $B$ é \textbf{INCORRETO} afirmar sobre a teoria dos conjuntos que:
  \begin{enumerate}[a)]
  \item A e B são disjuntos quando não possuírem intersecção.
  \item União de A com B é o conjunto formado por elementos que pertencem a A ou B.
  \item Se B é um subconjunto de A, a complementar de A em B é igual a B – A.
  \item Intersecção de A com B é o conjunto dos elementos em comum a A e B.
  \end{enumerate}

  \item (CESGRANRIO-2012) Se $P$, $M$ e $N$ são conjuntos e $x$ é tal que $x \notin P \cup M \cup N$ , então:
  \begin{enumerate}[a)]
  \item $x \notin P$  e $x  \notin M$  e $x \notin N$;
  \item $x \notin P$ ou $x \notin M$ ou $x \notin N$;
  \item $x \notin P$ ou $x \notin M \cup N$;
  \item $x \notin P \cap M$ e $x \notin N$;
  \item $x \notin P \cup M$ ou $x \notin N$.
  \end{enumerate}

  \item (Quadrix - 2014)  Considere os conjuntos:
      \[F = \{2, 5, 6, 9,10 \}\]
      \[G = \{3, 4, 7, 8, 12\}\]
 Sabe-se que o conjunto $H$ é formado por uma operação realizada entre os conjuntos $F$ e $G$. Assinale a alternativa que  contém a operação realizada entre os conjuntos $F$ e $G$, que tem como resultado o conjunto $H$, sendo que $H = \varnothing$.
 \begin{enumerate}[a)]
  \item $H= (F \cup G)$;
  \item $H= (F \cap G)$;
  \item $H= (F - G)$;
  \item $H= (G - F)$;
  \item $H= (G \cup F)$.
 \end{enumerate}

 \newpage
 \item (FAFIPA - 2014) Sendo os conjuntos $A = \{ x,y,z \}$ e $B =  \{a, x,y,t\}$. Assinale a alternativa que apresenta o conjunto $A - B$:
 \begin{multicols}{5}
 \begin{enumerate}[a)]
  \item $\{ z \}$
  \item $\{ x, y \}$
  \item $\{a, t\}$
  \item  $\{x\}$
 \end{enumerate}
 \end{multicols}

 \item (QUADRIX - 2012) Considere os conjuntos:
  \[ A = \{ x \in \R | - 3 < x < 4 \} \ \ \text{ e } \ \ B = \{ x \in \R | - 2 < x < 5 \}.\]
 O conjunto $B - A$ possui quantos números inteiros?
 \begin{multicols}{5}
 \begin{enumerate}[a)]
  \item 2
  \item 0
  \item 4
  \item 3
  \item 1
 \end{enumerate}
 \end{multicols}

 \item (CONSULPLAN - 2014) Sejam os conjuntos $A = \{0, \{1\}, \{2\}, \{3, 4\}\}$ e $B = \{ø, 2, \{3\}, \{0, 3\}\}$. Diante das informações, analise.
 \begin{multicols}{2}
 \begin{enumerate}[I)]
  \item $3 \in B$
  \item $\{3, 4\} \in A$
  \item $\varnothing \nsubset A$
  \item $\varnothing \in B$
 \end{enumerate}
 \end{multicols}
 Estão corretas apenas as alternativas
 \begin{multicols}{2}
 \begin{enumerate}[a)]
 \item I e III.
 \item II e IV.
 \item III e IV.
 \item II, III e IV.
 \end{enumerate}
 \end{multicols}

 \item (SOCIESC - Téc. Enfermagem) Dentre as várias opções para compor o prato de almoço do restaurante do Arnaldo, está o feijão e o arroz. De todas as 328 pessoas que almoçaram lá, num determinado período, apenas 20 não optaram por, pelo menos, um dos ingredientes, contra a grande maioria que optou ou em comer somente o arroz, somente o feijão ou, ainda, em misturá-los. Assinale a alternativa que descreve a quantidade total de pessoas que optaram em comer feijão com arroz, se o total de pessoas que NÃO comeram arroz é de 100 pessoas e sabendo que 110 pessoas NÃO comeram feijão.
  \begin{enumerate}
  \item 90
  \item 80
  \item 218
  \item 228
  \item 138
 \end{enumerate}

 \newpage
 \item (Lógica - Fundatec - 2018) Em um condomínio de 245 condôminos, sabe-se que 125 usam o salão de jogos, 96 usam o salão de jogos e a piscina. Mas 74 não usam o salão de jogos nem a piscina. Quantos condôminos usam a piscina e não usam o salão de jogos?
\begin{multicols}{5}
\begin{enumerate}[a)]
\item 29
\item 46
\item 75
\item 120
\item 149
\end{enumerate}
\end{multicols}

\item (Lógica - Fundatec - 2018) Todos os funcionários proficientes em espanhol são também proficientes em italiano, mas nenhum funcionário proficiente em italiano é proficiente em francês. Então deduzimos que:
\begin{enumerate}[a)]
\item Algum funcionário é proficiente em espanhol e francês.
\item Todos os funcionários são proficientes em espanhol e francês.
\item Todos os funcionários não são proficientes em francês.
\item Todos os funcionários são proficientes em italiano.
\item Nenhum funcionário é proficiente em francês e espanhol.
\end{enumerate}

\item (Lógica COVEST - UFPE - 2014)  Uma pesquisa entre todos os funcionários de um escritório revelou que: 14 funcionários tomam refrigerante da marca C, 8 tomam refrigerante da marca G, 5 tomam refrigerantes das duas marcas, e 3 não tomam refrigerante. Quantos funcionários tomam precisamente uma marca de refrigerante?
\begin{enumerate}[a)]
\item 9
\item 10
\item 11
\item 12
\item 13
\end{enumerate}

 \end{enumerate}

 Gabarito:
 1 a); 2 c); 3 a); 4 b); 5 a); 6 e); 7 c); 8 e); 9 b); 10 e); 11 d).
