\section{Questões}

 \begin{enumerate}
  \item (FEPESE - 2017) Uma empresa aluga containers para guarda de bens. Se o custo de alugar $\frac{1}{4}$ de um container é R\$ 1.400,00 mensais, quanto custa alugar $\frac{4}{5}$ deste container?
  \begin{enumerate}[a)]
  \item Mais do que R\$ 4550,00.
  \item Mais do que R\$ 4500,00 e menos que R\$ 4550,00.
  \item Mais do que R\$ 4450,00 e menos que R\$ 4500,00.
  \item Mais do que R\$ 4400,00 e menos que R\$ 4450,00.
  \item Menos do que R\$ 4400,00.
  \end{enumerate}
  
  \item (FEPESE - 2016) Uma pessoa abastece 30 litros em seu carro e observa que o ponteiro do marcador do combustível, que indicava $\frac{1}{4}$ antes do abastecimento, passou para $\frac{2}{3}$. Portanto a capacidade total do tanque, em litros, é:
  \begin{enumerate}[a)]
  \item Maior que 73 litros.
  \item Maior que 71 e menor que 73 litros.
  \item Maior que 69 e menor que 71 litros.
  \item Maior que 66 e menor que 69 litros.
  \item Menor que 66 litros.
  \end{enumerate}

  \item (FEPESE - 2016) Em um concurso, a razão entre aprovados e reprovados é $\frac{2}{11}$. Se o total de pessoas que prestaram esse concurso é de 143, então o número de aprovados é igual a:
  \begin{enumerate}[a)]
  \item 18.
  \item 19.
  \item 20.
  \item 21.
  \item 22.
  \end{enumerate}

  \item (Inst. AOCP - 2017) Com base no conhecimento sobre operações entre números Racionais na forma de frações irredutíveis, é correto afirmar que o resultado da operação $\frac{3}{5} - \frac{2}{5} + \frac{8}{5}$  é
  \begin{enumerate}[a)]
  \item $\frac{13}{15}$.
  \item $\frac{13}{5}$.
  \item $\frac{-7}{5}$.
  \item $\frac{9}{15}$.
  \item $\frac{9}{5}$.
  \end{enumerate}

  \item (Inst. AOCP - 2017) Dadas as frações $A= \frac{1}{3}$, $B= \frac{3}{11}$ e $C= \frac{10}{33}$ assinale a alternativa que representa a verdadeira desigualdade.
  \begin{enumerate}[a)]
  \item $A < B < C$.
  \item $B < C < A$.
  \item $C < A < B$.
  \item $B < A < C$.
  \item $C < B < A$.
  \end{enumerate}

  \item (UEM - 2017) Fábio possui $48$ figurinhas e deu $\frac{1}{4}$ delas para Artur, $\frac{1}{8}$ para Bianca e $\frac{5}{12}$ para Carlos. Com quantas figurinhas Fábio ficou?
  \begin{enumerate}[a)]
  \item 8
  \item 10
  \item 12
  \item 24
  \item 26
  \end{enumerate}

  \item (UFSCAR - 2017) Qual o valor da expressão $(\frac{3}{2})^3 - (\frac{1}{2})^2 \times (\frac{2}{15})^{-1}$ ?
  \begin{enumerate}[a)]
  \item $\frac{3}{2}$
  \item $\frac{-3}{4}$
  \item $\frac{3}{8}$
  \item $\frac{401}{120}$
  \item $\frac{1}{2}$
  \end{enumerate}

  \item (NC-UFPR - 2017) Assinale a alternativa que apresenta o valor da expressão $\frac{(2^{-2} \times 16)^{\frac{1}{2}}}{2^{-1}}$.
   \begin{enumerate}[a)]
  \item 1
  \item 2
  \item 4
  \item 8
  \item 16
  \end{enumerate}

  \item (UniRV-Goais - 2017) Qual o resultado de $16^{\frac{1}{4}} + 4^{\frac{-1}{2}}$?
  \begin{enumerate}[a)]
  \item 3
  \item $\frac{3}{2}$
  \item 5
  \item $\frac{5}{2}$
  \end{enumerate}

  \item (FCC - 2017) A expressão numérica $(3,4)^{-1} \cdot 6,8 - \left[ \left(\frac{3}{2} \right)^{-1} - \left(\frac{-3}{2} \right)^{-1} \right]$ é igual a:
  \begin{enumerate}[a)]
  \item 0
  \item $\frac{-1}{4}$
  \item 1,5
  \item $\frac{-1}{2}$
  \item $\frac{2}{3}$
  \end{enumerate}

  \item (UFSCAR - 2017) Certa receita para $18$ porções de um cookie requer $3/4$ de xícara de açúcar mascavo, $1/2$ xícara de margarina e $180$g de chocolate em pó. Para produzir $24$ porções do mesmo cookie, devemos ajustar a receita utilizando:
   \begin{enumerate}[a)]
  \item $1$ xícara de açúcar mascavo, $2/3$ de xícara de margarina e $240$g de chocolate em pó.
  \item $2$ xícaras de açúcar mascavo, $2/3$ de xícara de margarina e $260$g de chocolate em pó.
  \item $2$ xícaras de açúcar mascavo, $4/3$ de xícaras de margarina e $240$g de chocolate em pó.
  \item $1$ xícara de açúcar mascavo, $2/3$ de xícara de margarina e $260$g de chocolate em pó.
  \item $2$ xícaras de açúcar mascavo, $2/3$ de xícara de margarina e $240$g de chocolate em pó.
  \end{enumerate}


  \item (FGV - 2017) Odete comprou um saco contendo 8 dúzias de balas. A seguir, ela fez saquinhos menores com 7 balas cada um. Tendo feito o maior número possível de saquinhos, o número de balas que sobrou foi
  \begin{enumerate}[a)]
  \item 1
  \item 2
  \item 3
  \item 4
  \item 5
  \end{enumerate}

  \item (FGV - 2017) Considere as frações $a= \frac{2}{5}$, $b=\frac{7}{20}$ e $c=\frac{7}{20}$. A ordem crescente dessas frações é:
  \begin{enumerate}[a)]
  \item a, b, c.
  \item b, a, c.
  \item c, a, b.
  \item b, c, a.
  \item c, b, a.
  \end{enumerate}

  \item (FGV - 2017) Uma fábrica de bebidas vai engarrafar todo a cerveja contida em 8 barris. Cada barril contém 150 litros de cerveja e cada garrafa tem a capacidade de 750 mililitros. Assinale a opção que indica o número de garrafas usado pela fábrica.
  \begin{enumerate}[a)]
  \item 1200
  \item 1300
  \item 1400
  \item 1500
  \item 1600
  \end{enumerate}

  \item (FGV - 2017) Raul tem 96 anos. Teotônio tem um terço da idade de Raul e Sara tem 9 anos a mais do que Teotônio. Assinale a opção que indica a idade de Sara.
  \begin{enumerate}[a)]
  \item 23 anos
  \item 29 anos
  \item 32 anos
  \item 39 anos
  \item 41 anos
  \end{enumerate}

  \item (FGV - 2017) Em certo concurso inscreveram-se 192 pessoas, sendo a terça parte, homens. Desses, apenas a quarta parte passou. O número de homens que passaram no concurso foi:
  \begin{enumerate}[a)]
  \item 12
  \item 15
  \item 16
  \item 18
  \item 20
  \end{enumerate}

   \item (Sociesc - 2009) Um documento da prefeitura possui 300 páginas. Com a formatação atual do texto, cada página possui 35 linhas sendo que cada linha é composta por 60 letras. Se a formatação por modificada para 30 linhas de 50 letras cada, o documento terá:
 \begin{enumerate}
  \item 258 páginas
  \item 250 páginas
  \item 420 páginas
  \item 500 páginas
 \end{enumerate}

 \item (Sociesc - 2009) A direção de uma escola resolveu presentear a vitória de seus alunos em uma olimpíada esportiva, com os fichários que os alunos usariam no próximo ano. Foram 32 alunos que ganharam o
 prêmio. O funcionário responsável pela compra e distribuição dos fichários, comprou 9 pacotes com 12 fichários cada. Cada aluno utiliza 4 fichários durante o ano.

 Pode-se afirmar que a quantidade planejada pelo funcionário é inadequada, pois:
 \begin{enumerate}
  \item faltarão 20 fichários para que cada aluno receba seus 4 fichários.
  \item faltarão 8 fichários para que cada aluno receba seus 4 fichários.
  \item cada aluno receberá seus 4 fichários e ainda sobrarão 6 fichários.
  \item cada aluno receberá seus 4 fichários e ainda sobrarão 12 fichários.
 \end{enumerate}

 \item (Sociesc - 2010) Determine o número racional $\frac{x}{y}$ sendo:
  $x=\frac{0,3 + \frac{1}{4}}{0,01}$ e $y=\frac{2-\frac{1}{3}}{0,1+6}$
  \begin{enumerate}
  \item 2013
  \item 204,2
  \item 201,3
  \item 2018
  \item 207,3
 \end{enumerate}

 \item (Sociesc - 2010) Em um determinado congestionamento de 6 km, cada carro ocupa 4,5 metros, em média, incluindo o pequeno espaço que separa um carro do outro. Quantos carros há neste congestionamento?
  \begin{enumerate}
  \item 1333 carros
  \item 1330 carros
  \item 1352 carros
  \item 1335 carros
  \item 1323 carros
 \end{enumerate}

 \item (Sociesc - 2008) Uma empresa tem dois reservatórios cúbicos de água: reservatório A, com lado igual a 1 metro e o reservatório B, com lado igual a 2 metros. Duas torneiras, torneira X e torneira Y, enchem de água os reservatórios cúbicos A e B, respectivamente. Sabendo-se que a torneira X tem vazão de 1 litro por hora, podemos afirmar que a vazão da torneira Y, para encher o recipiente B na metade do tempo que a torneira X enche o recipiente A, deve ser de:
  \begin{enumerate}
  \item 10 l/h
  \item 16 l/h
  \item 14 l/h
  \item 12 l/h
 \end{enumerate}

 \item (Sociesc - 2008) O número total de refrigerantes vendidos por uma cadeia de lanchonetes está aumentando exponencialmente. Se 4 bilhões de refrigerantes foram vendidos em 2000 e 12 bilhões foram vendidos no ano 2005, podemos afirmar que no ano de 2010, serão vendidos:
  \begin{enumerate}
  \item 26 bilhões
  \item 36 bilhões
  \item 42 bilhões
  \item 30 bilhões
 \end{enumerate}

 \item (Lógica - Fundatec - 2018) O funcionário de uma empresa constatou que, no mês de dezembro, gastou $\frac{1}{4}$ do seu salário em alimentação, $\frac{1}{5}$ do seu salário em transporte e $\frac{1}{3}$ do seu salário em moradia. Portanto, podemos conlcuir que:
\begin{enumerate}[a)]
\item Esse funcionário gastou todo o seu salário exclusivamente em alimentação, transporte e moradia.
\item Se o funcionário gastou R\$ 800,00 em alimentação então gastou R\$ 1.000,00 em moradia.
\item Se o funcionário gastou R\$ 1.200,00 em transporte então gastou R\$ 2.000,00 em moradia.
\item Se o funcionário gastou R\$ 400,00 em alimentação então gastou R\$ 500,00 em moradia.
\item Se o salário do funcionário é de R\$ 1.600,00 então ele gastou R\$ 360,00 em transporte.
\end{enumerate}

 \item (UFPE/Covest - 2015) Joana gastou $4/7$ do que havia em sua poupança comprando móveis; com o restante comprou um aparelho de TV. Se o aparelho de TV custou R\$ 1.200,00, quanto custaram, em reais, os móveis?
 \begin{enumerate}
 \item R\$ 1.300,00
 \item R\$ 1.400,00
 \item R\$ 1.500,00
 \item R\$ 1.600,00
 \item R\$ 1.700,00
 \end{enumerate}

 \item (UFPE/Covest - 2015) Usando 42 linhas por página e 78 caracteres (ou espaços) em cada linha, um texto ocupa 54 páginas. Para melhorar a legibilidade do texto, diminuiu-se para 26 o número de linhas por página e para 63 o número de caracteres (ou espaços) por linha. Qual será o número de páginas ocupadas pelo texto na nova configuração?
 \begin{enumerate}
 \item 100
 \item 102
 \item 104
 \item 106
 \item 108
 \end{enumerate}

 \end{enumerate}

 Gabarito: 1 c); 2 b); 3 e); 4 e); 5 b); 6 b); 7 a); 8 c); 9 d); 10 e); 11 a); 12 e); 13 d); 14 e); 15 e); 16 c); 17 c); 18 a); 19 c); 20 a); 21 b); 22 b); 23 c); 24 d); 25 e).
