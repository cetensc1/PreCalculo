\section{Questões}

 \begin{enumerate}
  \item (UFES) Uma fábrica de papel e celulose possui uma plantação de $100000$ pés de eucalipto em sua área de plantio comercial.
  A fábrica pretende explorar essa área, derrubando $2000$ pés de eucalipto por dia e, ao mesmo tempo, fazendo o plantio de $m$ pés
  de eucalipto por dia. Dessa forma, a fábrica espera contar com pelo menos $110000$ pés de eucalipto no prazo de $360$ dias.
  Para atingir esta meta, o valor mínimo de $m$ deverá ser:
  \begin{enumerate}
  \item $2025$
  \item $2028$
  \item $2026$
  \item $2029$
  \item $2027$
  \end{enumerate}

  \item Uma bolsa de valores tinha um preço de R\$ $42,00$ quando sofreu uma queda de R\$$2,50$ por dia, durante $5$ dias seguidos.
  \begin{enumerate}
  \item Qual é a função que representa a queda do valor dessa ação em função do dia?
  \item Represente, no plano cartesiano, os pontos correspondentes a esses $5$ dias e o segmento de reta que passa por esses pontos.
  \end{enumerate}

  \item Um táxi, realizando uma corrida, cobra uma taxa fixa denominada bandeira de R\$$3,50$ e R\$$0,80$ por quilômetro rodado.
  Com base nesses dados, determine:
  \begin{enumerate}
  \item A função que representa o valor pago por uma corrida de $x$ quilômetros.
  \item Quantos quilômetros foram rodados se a conta foi de R\$ $17,10$.
  \end{enumerate}

  \item Para cercar um terreno, tem-se duas opções:
  1ª) Taxa de entrega no local R\$ $100,00$ e R\$$12,00$ o metro linear de cerca.
  2ª) Taxa de entrega no local R\$ $80,00$ e R\$ $15,00$ o metro linear de cerca.
  \begin{enumerate}
  \item Represente o custo de cada opção para $x$ metros de cerca.
  \item Qual das duas opções é mais vantajosa para $140$m de perímetro.
  \end{enumerate}

  \item No Brasil, o sistema de numeração de sapatos ou tênis é baseado na fórmula $N(p)= \frac{5p + 28}{4}$, que indica o valor aproximado do
  número do calçado $N$ em função do comprimento $p$, em centímetros do pé da pessoa. Determine o número do sapato ou tênis que uma pessoa deve
  comprar se, ao medir o comprimento de seu pé obteve:
  \begin{enumerate}
  \item $22,8$ cm
  \item $24$ cm
  \item $26,4$ cm
  \end{enumerate}

  \item (CONSULPLAN - 2010) Sejam os conjuntos $A = \{- 3, - 1, 1, 3, 5, 7\}$ e $B = \{- 4, -2, 0, 2, 4\}$. É correto afirmar:
  \begin{enumerate}
  \item $f(x) = x + 1$ é uma função de A em B.
  \item $f(x) = 2x + 5$ é uma função de B em A.
  \item $f(x) = 2x - 8$ é uma função de A em B.
  \item $f(x) = x + 3$ é uma função de B em A.
  \item $f(x) = x^2 + 2x - 3$ é uma função de B em A.
 \end{enumerate}

 \item (FEPESE - 2017) Uma empresa farmacêutica testou um novo remédio em um grupo de pessoas. Todas tomaram o remédio no mesmo dia, no mesmo momento. Duas pessoas apresentaram reação alérgica no mesmo dia da ingestão do remédio.

 De fato, a empresa verifica que o número de pacientes que apresentaram reação alérgica ao remédio é dado pela função $r(t) = a t + b$, onde $t$ é o tempo em dias a partir da ingestão do remédio, e $a= 1$ e $b$ são números reais.

 Se após $3$ dias cinco pessoas apresentaram reação alérgica, quantas pessoas apresentaram reação alérgica após $6$ dias?
 \begin{enumerate}
  \item $14$.
  \item $12$.
  \item $10$.
  \item $8$.
  \item $6$.
 \end{enumerate}

 \item (FEPESE - 2017) Em um shopping, o estacionamento é gratuito pela primeira hora. A segunda hora (ou fração desta) custa R\$ 2,50. A terceira hora (ou fração) custa R\$ 2,75. A quarta hora (ou fração) custa R\$ 3,00 e assim sucessivamente.

 O custo de deixar um carro neste estacionamento por 14 horas e meia é:
 \begin{enumerate}
  \item Mais do que R\$ 90,00.
  \item Mais do que R\$ 85,00 e menos que R\$ 90,00.
  \item Mais do que R\$ 80,00 e menos que R\$ 85,00.
  \item Mais do que R\$ 75,00 e menos que R\$ 80,00.
  \item Menos que R\$ 75,00.
 \end{enumerate}

 \item (FEPESE - 2017) Uma função $f$ definida nos números reais é dita injetiva se $x \neq y$, então $f(x) \neq f(y)$.

Considere as afirmativas abaixo:
\begin{enumerate}[1.]
 \item A função $f$:  dada por $f(x) = x^2$ é injetiva.

 \item Se $f$ é uma função tal que $f(x) = f(y)$ implica que $x = y$, então, $f$ é injetiva.

 \item A função $f$:  dada por $f(x) = -2x + 5$ é injetiva.
 \end{enumerate}
 Assinale a alternativa que indica todas as afirmativas corretas.
 \begin{enumerate}
 \item É correta apenas a afirmativa 3.
 \item São corretas apenas as afirmativas 1 e 2.
 \item São corretas apenas as afirmativas 1 e 3.
 \item São corretas apenas as afirmativas 2 e 3.
 \item São corretas as afirmativas 1, 2 e 3.
 \end{enumerate}

 \item (FEPESE - 2017) Denotamos por $\R$ o conjunto dos números reais. Convencionamos nesta questão que uma função $f : \R \rightarrow \R$ é crescente se $x < y$ implica que $f(x) < f(y)$.

Considere as afirmativas abaixo:
\begin{enumerate}[1.]
\item A função $f:\R \rightarrow \R$, dada por $f(x) = -2x + 25$ é crescente.

\item Se $f$ é tal que $f(x) \geqslant f(y)$ implica $x \geqslant y$ então f é crescente.

\item A função $f: \R \rightarrow \R$ dada por $f(x) = x^2$  é crescente.
\end{enumerate}

Assinale a alternativa que indica todas as afirmativas corretas.

\begin{enumerate}
 \item É correta apenas a afirmativa 1.
 \item É correta apenas a afirmativa 2.
 \item É correta apenas a afirmativa 3.
 \item São corretas apenas as afirmativas 1 e 2.
 \item São corretas apenas as afirmativas 2 e 3.
\end{enumerate}

\item (FEPESE - 2010) Considere a função $f: \R \rightarrow \R$, definida por
\[f(x)= \frac{1}{\sqrt{x^2 - 1}}\]
Assinale a alternativa que indica \textbf{corretamente} o domínio dessa função.
\begin{enumerate}
\item $\{x \in \R \mid x \neq 1\}$;
\item $\{x \in \R \mid x < -1\} \cup \{x \in \R \mid x > 1\}$;
\item $\{x \in \R \mid x \leqslant -1\} \cup \{x \in \R \mid x \geqslant 1\}$;
\item $\{x \in \R \mid x > -1\} \cup \{x \in \R \mid x < 1\}$;
\item $\{x \in \R \mid x \geqslant -1\} \cup \{x \in \R \mid x \leqslant 1\}$;
\end{enumerate}

\item (CESGRANRIO - 2017) Qual o maior valor de $k$ na equação $log(kx) = 2log(x+3)$ para que ela tenha exatamente uma raiz?
\begin{enumerate}
\item 0;
\item 3;
\item 6;
\item 9;
\item 12.
\end{enumerate}

\item (CESGRANRIO - 2017) Quantos valores reais de $x$ fazem com que a expressão $(x^2 - 5x + 5)^{(x^2 + 4x - 60)}$ assuma valor numérico igual a $1$?
\begin{enumerate}
\item 2;
\item 3;
\item 4;
\item 5;
\item 6.
\end{enumerate}

\item (CONED - 2016) Qual a soma das raízes ou zeros da função exponencial abaixo?
\[f(x)= 2^{2x-3} - 3 \cdot 2^{x-1} + 4\]

\begin{enumerate}
\item 5;
\item 4;
\item 6;
\item 8;
\item -6.
\end{enumerate}

\item (IBFC - 2016) O valor da função $f(x) = 2^{3x-1} + 1$ para $x = 2$ é:
\begin{enumerate}
\item 63;
\item 32;
\item 33;
\item 17.
\end{enumerate}

 \item (Sociesc - 2009) Uma empresa de telefonia celular lançou o seguinte plano de pagamento: uma tarifa mensal fixa de R\$ 25,00 e R\$ 0,50 por minuto usado em ligações locais. Paulo, pensando em fazer este
 plano de pagamento, perguntou ao vendedor quanto pagaria em um mês que ele usasse o celular durante 2,5 horas com ligações locais. A
 resposta que o vendedor deu foi:
 \begin{enumerate}
  \item R\$ 100,00
  \item R\$ 87,50
  \item R\$ 69,00
  \item R\$ 26,25
 \end{enumerate}

 \item (Sociesc - 2009) A turma de enfermagem resolveu guardar dinheiro para o pagamento da sua festa de formatura. Para isso, foi feita uma aplicação financeira que obedece a seguinte equação:
  \[T = 4.000 + 150.x ,\]
  onde $T$ é o saldo total e $x$ é o tempo em meses. Sendo assim, o tempo para que o saldo total seja R\$ 10.000,00 será:
  \begin{enumerate}
  \item 32 meses
  \item 50 meses
  \item 60 meses
  \item 40 meses
 \end{enumerate}

 \item (Sociesc - 2010) O valor total cobrado por um eletricista consiste em uma taxa fixa, que é de R\$ 35,00, mais uma quantia que depende da quantidade de fio utilizada por ele no serviço executado. Observando a tabela abaixo, que mostra alguns orçamentos feitos por este eletricista, pode-se afirmar que para um serviço que necessita de 21 metros de fio, o valor total do serviço executado por ele será de:

  \begin{table}[H]
  \centering
 \begin{tabular}{|c|c|} \hline
  \multicolumn{1}{|c|}{\textbf{Quantidade de fio
(metros)}} & \multicolumn{1}{|c|}{\textbf{Valor total do serviço
(R\$)}} \\ \hline
 6 & 53,00 \\ \hline
 8 & 59,00 \\ \hline
 12 & 71,00 \\ \hline
 \end{tabular}
\end{table}

  \begin{enumerate}
  \item R\$ 91,00
  \item R\$ 90,00
  \item R\$ 98,00
  \item R\$ 94,00
  \item R\$ 93,00
 \end{enumerate}

 \item (Sociesc - 2010) Uma empresa que faz fotocopia criou um plano para o pagamento das cópias. O plano consiste em uma taxa fixa (mensal) de R\$ 50,00 e mais R\$ 0,07 por cópia.

  O número mínimo de cópias que devem ser tiradas em um mês, por um só cliente, para que o preço total ultrapasse R\$ 99,00, é:
  \begin{enumerate}
  \item 701
  \item 700
  \item 710
  \item 670
  \item 680
 \end{enumerate}

 \item (Sociesc - 2010) Uma empresa de componentes eletrônicos tem, para um de seus componentes, um custo diário de produção dado por $C(x) = x^2 – 120x + 2500$, onde $C(x)$ é o custo em reais e $x$ é o número de unidades fabricadas. O número de componentes eletrônicos que devem ser produzidos diariamente para que o custo seja mínimo é:
 \begin{enumerate}
  \item 54
  \item 122
  \item 50
  \item 60
  \item 141
 \end{enumerate}

 \item (Sociesc - 2010) Uma corrida de táxi inclui uma taxa fixa, denominada bandeirada, e uma parcela que depende da distância percorrida. A bandeirada custa R\$ 5,30 e cada quilômetro rodado custa R\$ 1,20. A distância aproximada percorrida por um passageiro que pagou R\$ 25,30 foi de?
  \begin{enumerate}
  \item 20 km
  \item 15 km
  \item 16,6 km
  \item 17,6 km
  \item 20,6 km
 \end{enumerate}

  \item (Sociesc - 2008) Um reservatório esta sendo esvaziado para limpeza. O seu volume varia com o tempo de acordo com a função $V(t)= 120(20 - t)^2$, onde o volume é dado em litros e o tempo, em minutos. De acordo com esta lei, podemos afirmar que o tempo que o reservatório demora para ficar vazio e a quantidade de litros que são escoados nos primeiros 10 minutos são respectivamente:
  \begin{enumerate}
  \item 22 horas e 36.000 litros
  \item 20 horas e 36.000 litros
  \item 20 horas e 30.000 litros
  \item 22 horas e 30.000 litros
 \end{enumerate}

 \item (UDESC/Fepese - 2009) Considere a função $f: \R \to \R$, definida por:
 \[f(x)= \frac{1}{\sqrt(x^2 - 1)}\]
 Assinale a alternativa que indica \textbf{corretamente} o domínio dessa função.
 \begin{enumerate}
 \item $\{x \in \R \mid x \neq 1\}$
 \item $\{x \in \R \mid x < -1\} \cup \{x \in \R \mid x>1\}$
 \item $\{x \in \R \mid x \leq -1\} \cup \{x \in \R \mid x \geq 1\}$
 \item $\{x \in \R \mid x > -1\} \cup \{x \in \R \mid x>1\}$
 \item $\{x \in \R \mid x \geq -1\} \cup \{x \in \R \mid x\leq 1\}$

 \end{enumerate}

 \end{enumerate}

 Gabarito: 1 b); 2 a) $f(x)= 42 - 2,50 x$; 2 b) a cargo do leitor; 3 a) $f(x)= 3,50 + 0,80 x$; 3 b) $x= 17 Km$; 4 a) 1ª opção: $f(x)= 100 + 12x$, 2ª opção: $f(x)= 80+15x$; 4 b) 1ª opção; 5 a) $35,5$; 5 b) $37$; 5 c) $40$; 6 d); 7 d); 8 e); 9 d); 10 b); 11 b); 12 e); 13 a); 14 a); 15 c); 16 a); 17 d); 18 c); 19 a); 20 d); 21 c); 22 b); 23 b).

% \begin{enumerate}
% \item
% \item
% \item
% \item
% \item
% \end{enumerate}
