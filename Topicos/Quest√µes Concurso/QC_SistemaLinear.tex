\section{Questões}

\begin{enumerate}
 \item (VUNESP - 2017) Um cliente de uma doceria comprou três bolos do tipo A e dois bolos do tipo B e pagou por eles a quantia de $R\$ 300,00$. Outro cliente comprou dois bolos do tipo A e quatro bolos do tipo B e pagou por eles a quantia de $R\$ 400,00$. A diferença de preço entre o bolo mais caro e o bolo mais barato é de:
 \begin{enumerate}
 \item $R\$ 15,00$
 \item $R\$ 20,00$
 \item $R\$ 25,00$
 \item $R\$ 30,00$
 \item $R\$ 35,00$
 \end{enumerate}
 
 \item (FEPESE - 2010) Encontre o valor de $a$ para que o sistema linear
 \[ \begin{cases} 
 ax+y+z= 15 \\ 
 2y + 8z= 17 \\ 
 x + 4z= 19 
 \end{cases} \]
 
 Não tenha solução:
 \begin{enumerate}
\item $\frac{-3}{4}$
\item $\frac{3}{4}$
\item $\frac{-5}{4}$
\item $\frac{5}{4}$
\item $\frac{1}{4}$
\end{enumerate}
 
 \item (IBADE - 2017) Observe o sistema linear de incógnitas $a$, $b$ e $c$.
 \[ \begin{cases} 
 8a + 2b + 12c= 10 \\ 
 6a + 3b + 9c= 2 \\ 
 2a + b + 3c= 1 
 \end{cases} \]
 Dessa forma, pode-se afirmar que esse sistema:
\begin{enumerate}
\item possui uma única solução.
\item possui uma infinidade de soluções.
\item não possui solução.
\item possui exatamente duas soluções.
\item possui exatamente três soluções.
\end{enumerate}

\item (MPE - GO - 2017) Fernando teve três filhos em três anos seguidos. Quando ele fez 39 anos reparou que essa sua idade era igual à soma das idades dos seus três filhos. Nesse dia, o seu filho mais velho tinha:  
\begin{enumerate}
\item 12 anos;
\item 13 anos;
\item 14 anos;
\item 15 anos;
\item 16 anos.
\end{enumerate}

\item (FGV - 2017) A soma de dois números é 47 e o dobro da diferença deles é 46. O menor desses dois números é 
\begin{enumerate}
\item 11;
\item 12;
\item 15;
\item 17;
\item 23.
\end{enumerate}

\item (UTFPR - 2017) A razão entre dois números é igual a 3. Sabendo que a soma deles é 60, o maior número é igual a:
\begin{enumerate}
\item 10;
\item 40;
\item 35;
\item 45;
\item 50.
\end{enumerate}

\item (VUNESP - 2017) Tadeu verificou a capacidade total de uma jarra, de uma garrafa e de um copo, e estabeleceu as seguintes relações comparativas entre as respectivas capacidades:
\begin{itemize}
\item uma jarra equivale a três garrafas;
\item uma jarra mais uma garrafa equivalem a oito copos.
\end{itemize}
Pode-se concluir, então, que uma jarra equivale a
\begin{enumerate}[a)]
\item 3 copos;
\item 4 copos;
\item 5 copos;
\item 6 copos;
\item 7 copos.
\end{enumerate}

\item (Sociesc - 2009) Duas torneiras A e B enchem juntas um reservatório em 4 horas. Se a torneira B for fechada, o reservatório fica completamente cheio em 5 horas. Se apenas a torneira B estiver aberta, o reservatório ficará completamente cheio em:
  \begin{enumerate}
  \item 1 hora
  \item 9 horas
  \item 20 horas
  \item 10 horas
 \end{enumerate}
 
 \item (Sociesc - 2009) No sistema
 $\begin{cases}
  \frac{3x}{5} - \frac{y}{3} = \frac{28}{15} \\
  \frac{2y}{3} - \frac{x}{4} = \frac{-11}{6}
 \end{cases}$, o valor de $x$ é:
  \begin{enumerate}
  \item Igual ao valor de $y$.
  \item Dobro do valor de $y$.
  \item Igual ao valor de $-y$.
  \item Metade do valor de $y$.
 \end{enumerate}
 
 \item (Sociesc - 2009) Uma turma da graduação está organizando sua formatura. Se cada um deles contribuir com R\$ 1.300,00 para as despesas do baile, sobrará R\$ 4.200,00. Mas se cada um contribuir com R\$ 800,00, faltará R\$ 6.300. O valor exato para as despesas com o baile é:
  \begin{enumerate}
  \item R\$ 22.100,00
  \item R\$ 25.000,00
  \item R\$ 23.000,00
  \item R\$ 23.100,00
 \end{enumerate}
 
 \item (Sociesc - 2009) A razão entre as idades de Pedro e André (representada por P/A, onde P é a idade de Pedro e A é a idade de André) é igual a 5/2. A diferença entre a idade de Pedro e a de André é 27. A idade de André é:
 \begin{enumerate}
  \item 20 anos
  \item 17 anos
  \item 18 anos
  \item 23 anos
 \end{enumerate}
 
 \item (Sociesc - 2010) Resolva o sistema 
  $\begin{cases}
    x+y= 3 \\
    x^2+ y^2 -3x=2
   \end{cases}$
  \begin{enumerate}
  \item $\{(\frac{7}{2}, -\frac{1}{2}), (1,2)\}$
  \item $\{(7, -\frac{1}{2}), (2,3)\}$
  \item $\{(\frac{7}{2}, 2), (-2,-2)\}$
  \item $\{(\frac{5}{2}, -2), (-1,3)\}$
  \item $\{(\frac{5}{2}, -\frac{1}{2}), (1,2)\}$
 \end{enumerate}
 
 \item (Sociesc - 2008) Em determinado vestibular, cada um dos 60 candidatos obteve, na avaliação dissertativa, nota 6 ou nota 9. A média aritmética dessas notas foi 7. Sendo assim, podemos afirmar que o número de alunos que obteve nota 6 e o número de alunos que obteve nota 9 foi, respectivamente:
  \begin{enumerate}
  \item 25 e 35
  \item 37 e 23
  \item 40 e 20
  \item 42 e 28
 \end{enumerate}
 
 \item (Sociesc - 2008) A média aritmética de três números a, b e c é 4. A média ponderada entre eles, considerando pesos de 3, 3 e 4, respectivamente, para a, b e c é 4,2. Sabendo que a é igual a 2, y e z valem:
  \begin{enumerate}
  \item 4 e 5
  \item 3 e 5
  \item 3 e 6
  \item 4 e 6
 \end{enumerate}
 
 \item (Sociesc - 2010) João Paulo possui 200 metros de tela e pretende construir um cercado retangular com área de $2100 m^2$. Quais deverão ser as dimensãos do retângulo?
  \begin{enumerate}
  \item 70 e 30
  \item 60 e 25
  \item 35 e 15
  \item 43 e 21
 \end{enumerate}
 
 \item (Sociesc - 2010) A ordenada $(x, y, z)$ de números reais é solução do sistema:
  $\begin{cases}
    x+y-z=0 \\
    x-y+z=2 \\
    2x+y-3z=1
   \end{cases}$ então a soma $(x, y, z)$ é igual a:
  \begin{enumerate}
  \item 1
  \item 2
  \item 3
  \item 0
 \end{enumerate}
 
 \item (UDESC/Fepese - 2009) Encontre o valor de \textbf{a} para o sistema linear 
 $\begin{cases}
    ax+y+z=15 \\
    2y+8z=17 \\
    4x+4z=19
   \end{cases}$
 \textbf{não} tenha solução:
  \begin{enumerate}
  \item $\frac{-3}{4}$
  \item $\frac{3}{4}$
  \item $\frac{-5}{4}$
  \item $\frac{5}{4}$
  \item $\frac{1}{4}$
 \end{enumerate}
 
 \item (UFPE/Covest - 2015) Em uma loja, quatro calças, quatro camisas e dois pares de meias custam R\$ 330,00 e nove calças, nove camisas e seis pares de meias custam R\$ 750,00. Quanto custa um par de meias nessa loja? Observação: Nessa loja, todas as camisas têm o mesmo preço, todas as calças têm o mesmo preço e todos os pares de meias têm o mesmo preço. 
 \begin{enumerate}
  \item R\$ 4,20
  \item R\$ 4,40
  \item R\$ 4,60
  \item R\$ 4,80
  \item R\$ 5,00
 \end{enumerate}

\end{enumerate}

Gabarito: 1 c); 2 a); 3 c); 4 c); 5 b); 6 d); 7 d); 8 c); 9 c); 10 d); 11 c); 12 a); 13 c); 14 d); 15 a); 16 d); 17 a); 18 e).

