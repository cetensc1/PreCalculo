\section{Questões}
\begin{enumerate}
 \item (UFPE/Covest - 2015) A média das alturas de um grupo de funcionários de um escritório é 1,69 m. Quando dois novos funcionários se juntarem ao grupo, a média das alturas permaneceu inalterada. Qual das alternativas a seguir contém possíveis alturas dos dois novos funcionários?
 \begin{enumerate}
 \item 1,65 m e 1,63 m
 \item 1,70 m e 1,68 m
 \item 1,69 m e 1,68 m
 \item 1,79 m e 1,68 m
 \item 1,78 m e 1,68 m
 \end{enumerate}
 
 \item (UFPE/Covest - 2015) Em um exame, a média dos estudantes de uma turma foi 6,5. A média dos estudantes com nota inferior a 6,5 foi 5,5, e a média dos estudantes com nota igual ou superior a 6,5 foi 7,0. Se a turma era composta por 60 estudantes, quantos estudantes tiveram nota inferior a 6,5?
 \begin{enumerate}
 \item 18
 \item 19
 \item 20
 \item 21
 \item 22
 \end{enumerate}
 
 \item (UFPE/Covest - 2015) Para ser aprovado em determinada disciplina, um aluno precisa alcançar média maior ou igual a 7,0 nos três exames do semestre, que têm pesos diferentes. Se ele obteve notas respectivas 5,0 e 6,5 nos dois primeiros exames, que têm pesos respectivos 1 e 2, qual a menor nota que ele pode tirar no terceiro exame, que tem peso 3, para ser aprovado na disciplina?
 \begin{enumerate}
 \item 7,5
 \item 8,0
 \item 8,5
 \item 9,0
 \item 9,5
 \end{enumerate}
 
 
\end{enumerate}

Gabarito: 
