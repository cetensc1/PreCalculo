\section{Questões}

 \begin{enumerate}
  \item (FEPESE - 2013) Um ônibus está circulando com 40 passageiros e tem 4 paradas pela frente até o destino final. Sabe-se que na primeira parada entram 7 e saem 9 passageiros do ônibus. Ainda, a cada parada subsequente o número de passageiros que entram é o dobro do número de passageiros que entraram na parada anterior e o mesmo vale para o número de passageiros que saem. Logo, ao chegar ao destino final, o número de passageiros no ônibus é de:
 \begin{enumerate}[a)]
 \item 5
 \item 10
 \item 15
 \item 20
 \item 30
 \end{enumerate}


 \item (FEPESE - 2013) Uma pessoa deve colocar 312 lâmpadas brancas e 845 lâmpadas amarelas em caixas, de maneira que todas as caixas tenham o mesmo número de lâmpadas, o número de lâmpadas amarelas em cada caixa seja igual e não sobrem lâmpadas fora das caixas. Logo, o número de caixas necessárias para fazer tal divisão é:
  \begin{enumerate}[a)]
  \item 8
  \item 9
  \item 10
  \item 11
  \item 13
  \end{enumerate}


  \item (FEPESE - 2016) João trabalha 5 dias e folga 1, enquanto Maria trabalha 3 dias e folga 1. Se João e Maria folgam no mesmo dia, então quantos dias, no mínimo, passarão para que eles folguem no mesmo dia novamente?
  \begin{enumerate}[a)]
  \item 8
  \item 10
  \item 12
  \item 15
  \item 24
  \end{enumerate}


  \item (FEPESE - 2016) Em uma excursão participam 120 homens e 160 mulheres. Em determinado momento é preciso dividir os participantes em grupos formados somente por homens ou somente por mulheres, de maneira que os grupos tenham o mesmo número de integrantes. Neste caso, o número máximo de integrantes em um grupo é:
  \begin{enumerate}[a)]
  \item 10
  \item 15
  \item 20
  \item 30
  \item 40
  \end{enumerate}


  \item (VUNESP - 2017) Uma papelaria precisa organizar seu estoque de cadernos e, para isso, irá utilizar caixas de papelão, colocando em cada uma delas o mesmo número de cadernos. Se forem colocados 30 cadernos em cada caixa, todas as caixas serão utilizadas e 20 cadernos ficarão de fora, mas, se forem colocados 35 cadernos em cada caixa, todos os cadernos serão encaixotados e 2 caixas não serão utilizadas. Se essa papelaria decidir colocar 40 cadernos em cada caixa, todos os cadernos também serão encaixotados, e o número de caixas necessárias será
  \begin{enumerate}[a)]
  \item 12
  \item 14
  \item 16
  \item 18
  \item 20
  \end{enumerate}


   \item (FEPESE - 2013) Em uma fábrica, a entrega do insumo A ocorre a cada 18 dias e do insumo B, a cada 24 dias. Se ambos os insumos são entregues no dia de hoje, em quantos dias ambos serão novamente entregues simultaneamente?
  \begin{enumerate}[a)]
  \item 32
  \item 44
  \item 60
  \item 72
  \item 144
  \end{enumerate}


  \item (VUNESP - 2017) Se, numa divisão, o divisor e o quociente são iguais, e o resto é 10, sendo esse resto o maior possível, então o dividendo é
  \begin{enumerate}[a)]
  \item 131
  \item 121
  \item 120
  \item 110
  \item 101
  \end{enumerate}


   \item (FGV - 2017) Uma corda de 7 metros e 20 centímetros de comprimento foi dividida em três partes iguais. O comprimento de cada parte é:

  \begin{enumerate}[a)]
  \item 2 metros e 40 centímetros;
  \item 2 metros e 50 centímetros;
  \item 2 metros e 60 centímetros;
  \item 2 metros e 70 centímetros;
  \item 2 metros e 80 centímetros.
  \end{enumerate}

 \item (TJ/SC - 2018) Vanda foi ao consultório médico em uma segunda-feira. O médico disse que ela deveria tomar um comprimido de certo remédio todos os dias, durante 180 dias. Vanda começou a tomar o remédio no mesmo dia da consulta e cumpriu exatamente o que disse o médico.
  O primeiro dia que Vanda NÃO precisou tomar o remédio foi:
  \begin{enumerate}
  \item Uma quarta-feira;
  \item Uma quinta-feira;
  \item Uma sexta-feira;
  \item Um sábado;
  \item Um domingo.
 \end{enumerate}
 \end{enumerate}

 Gabarito: 1 b); 2 e); 3 c); 4 e); 5 b); 6 d); 7 d); 8 a); 9 d).
