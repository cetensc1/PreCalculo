\chapter{Sequências Numéricas}

 %\vskip0.3cm

\colorbox{azul}{
 \begin{minipage}{0.9\linewidth}
 \begin{center}
  Uma \textbf{sequência} ou \textbf{sucessão} de elementos/números é uma lista ordenada de elementos/números.
 \end{center}
 \end{minipage}}
 \vskip0.3cm


A ideia de sequência esta muito presente em nosso dia-a-dia, aparecendo por exemplo nas seguintes situações:
\begin{exem}
 \begin{itemize}
  Exemplos de sequências:
  \item a sequência dos dias das semana:
\[
\{\text{domingo}, \text{segunda}, \text{terça}, \text{quarta}, \text{quinta}, \text{sexta}, \text{sábado}.\}
\]
  \item a sequência dos meses do ano:
\begin{multline*}
\{\text{janeiro}, \text{fevereiro}, \text{março}, \text{abril}, \text{maio}, \text{junho}, \\
\text{julho}, \text{agosto}, \text{setembro}, \text{outubro}, \text{novembro}, \text{dezembro}.\}
\end{multline*}
  \item a sequência dos anos nos quais a Copa do Mundo de Futebol é realizada:
\begin{multline*}
\{1930, 1934, 1938, 1950, 1954, 1958, 1962, 1966, 1970, \\
1974, 1978, 1982, 1986, 1990, 1994, 1998, 2002, 2006, 2010, 2014, 2018.\}
\end{multline*}
  \item a sequência dos anos nos quais ocorreram eleições para presidente do Brasil:
\begin{multline*}
\{1891, 1894, 1898, 1902, 1906, 1910, 1914, 1918, 1919, 1922, 1926,\\ 1930, 1934, 1938, 1945, 1950, 1955, 1960, 1964, 1965, 1966, 1969,\\ 1974, 1978, 1985, 1989, 1994, 1998, 2002, 2006, 2010, 2014, 2018\}
\end{multline*}
  \item a sequência de Fibonacci:
\[
\{0, 1, 1, 2, 3, 5, 8, 13, 21, 34, 55, \ldots \}
\]
  \item a sequência dos números primos:
\[
\{2, 3, 5, 7, 11, 13, 17, 19, 23, \ldots \}
\]
  \item a sequência das potências de 2:
\[
\{1, 2, 4, 8, 16, 32, 64, 128, \ldots \}
\]
  \item a sequência das potências de 10:
\[
\{1, 10, 100, 1000, 10000, 100000, \ldots \}
\]
  \end{itemize}

\end{exem}

 Note que em todos os exemplos acima existe uma ordem nos elementos da sequência. Esses elementos são também chamados de \textbf{termos da sequência}, e são indicados pela letra $a_i$, onde $i$ indica a posição do termo na sequência. Por exemplo, na sequência dos dias da semana, temos: \\
 1º termo: $a_1=$ domingo\\
 2º termo: $a_2=$ segunda\\
 3º termo: $a_3=$ terça\\
 4º termo: $a_4=$ quarta\\
 5º termo: $a_5=$ quinta\\
 6º termo: $a_6=$ sexta \\
 7º termo: $a_7=$ sábado.

 Pelos exemplos percebemos também que algumas sequências são finitas, isto é, possuem um número finito de elementos, e outras são infinitas, isto é, possuem um número infinito de elementos, ao listar os elementos das sequências infinitas costumamos colocar alguns elementos seguidos de $\cdots$ para indicar que a sequência possui mais elementos.

 Algumas sequências numéricas são dadas por regras ou leis de formação, que possibilitam explicitar todos os termos da sequência. Por exemplo:
 \begin{exem}
  A sequência $a_n= 2n-1$, $n \in \N^{\ast}$, é dada por:
  \begin{itemize}
   \item para $n=1 \Rightarrow a_1=2\cdot 1 - 1= 1$,
   \item para $n=2 \Rightarrow a_2=2\cdot 2 - 1= 3$,
   \item para $n=3 \Rightarrow a_3=2\cdot 3 - 1= 5$,
   \item para $n=4 \Rightarrow a_4=2\cdot 4 - 1= 7$, etc.
  \end{itemize}
 Portanto a sequência é $\{1, 3, 5, 7, \ldots \}$, sequência dos números ímpares.

 Neste caso $a_n= 2n-1$, é chamado, \textbf{termo geral da sequência}.
 \end{exem}

 Nas próximas seções vamos estudar dois tipos especiais de sequências chamadas Progressão Aritmética (PA) e Progressão Geométrica (PG), que estão presentes em muitas questões de concurso.

\section{Progressão Aritmética (PA)}

 \colorbox{azul}{
 \begin{minipage}{0.9\linewidth}
 \begin{center}
  \textbf{Progressão Aritmética (PA)} é uma sequência na qual cada termo é igual à soma do termo anterior com uma constante $(r)$ denominada ``razão da PA''. Ou seja,
  \[a_{n}= a_{n-1} + r, \ \ \forall n \in \N^{\ast}, n > 1\]
 \end{center}
 \end{minipage}}
 \vskip0.3cm

 Desta definição obtemos que dados o primeiro termo ($a_1$) e a razão $r$ de uma PA para obter os próximos termos basta fazer:
 \[a_2= a_1 + r\]
 \[a_3= a_2 + r= a_1 + r + r= a_1 + 2 \cdot r\]
 \[a_4= a_3 + r= a_1 + 2 \cdot r + r= a_1 + 3 \cdot r\]
 e assim por diante, desta forma o termo de ordem $n$ da sequência, representado por $a_n$, e denominado termo geral da PA, é dado por:
 \[\destaque{a_n= a_1 + (n-1) \cdot r} .\]
 Na qual: $a_n=$ termo geral, $a_1=$ 1º termo, $n=$ nº de termos (até $a_n$) e $r=$ razão da PA.

 Ainda da definição de PA é fácil temos que:
 \[a_{n}= a_{n-1} + r \Rightarrow
 \destaque{r= a_{n} - a_{n-1}} ,\]
 logo para calcular a razão $r$ de uma PA basta fazer a diferença entre dois números consecutivos. Percebemos então que a PA:
 \begin{itemize}
  \item é constante, $a_n= a_{n-1} \Leftrightarrow r= 0$,
  \item é crescente, $a_n > a_{n-1} \Leftrightarrow r > 0$,
  \item é decrescente, $a_n < a_{n-1} \Leftrightarrow r < 0$.
 \end{itemize}

 \begin{exem} Exemplos de progressões aritméticas:
  \begin{itemize}
   \item PA constante: $\{5, 5, 5, 5, \ldots \}$ com $a_1= 5$ e razão $r= 0$.
   \item PA crescente: $\{2, 5, 8, 11, 14, \ldots \}$ com $a_1= 2$ e razão $r= 3$.
   \item PA decrescente: $\{9, 5, 1, -3, -7, \ldots \}$ com $a_1= 9$ e razão $r= -4$.
  \end{itemize}
 \end{exem}

 Dada uma PA é comum nos perguntarmos qual a soma dos $n$ primeiros termos desta PA, a fórmula que nos permite fazer este cálculo é a seguinte:
 \[\destaque{S_n= \frac{(a_1 + a_n)\cdot n}{2}}\]
 Em que: $a_n=$ é o enésimo termo, $a_1=$ é o 1º termo, $n=$ nº de termos (até $a_n$) e $S_n=$ é a soma dos $n$ primeiros termos.



\section{Progressão Geométrica (PG)}

 \colorbox{azul}{
 \begin{minipage}{0.9\linewidth}
 \begin{center}
  \textbf{Progressão Geométrica (PG)} é uma sequência na qual cada termo é igual ao produto do termo anterior com uma constante $(q)$ denominada ``razão da PG''. Ou seja,
  \[a_{n}= a_{n-1} \cdot q, \ \ \forall n \in \N^{\ast}, n > 1\]
 \end{center}
 \end{minipage}}
 \vskip0.3cm

 Desta definição obtemos que dados o 1ª termo $(a_1)$ e a razão $q$ de uma PG para obter os próximos termos basta fazer:
 \[a_2= a_1 \cdot q\]
 \[a_3= a_2 \cdot q= a_1 \cdot q \cdot q= a_1 \cdot q^2\]
 \[a_4= a_3 \cdot q= a_1 \cdot q^2 \cdot q= a_1 \cdot q^3 \]
 e assim por diante, desta forma o termo de ordem $n$ da sequência, representado por $a_n$, e denominado termo geral da PG, é dado por:
 \[\destaque{a_n= a_1 \cdot q^{(n-1)}} .\]
 Na qual: $a_n=$ termo geral, $a_1=$ 1º termo, $n=$ nº de termos (até $a_n$) e $q=$ razão da PG.

 Ainda da definição de PG é fácil temos que:
 \[a_{n}= a_{n-1} \cdot q \Rightarrow
 \destaque{q= \frac{a_{n}}{ a_{n-1}}} ,\]
 logo para calcular a razão $q$ de uma PG basta fazer a razão entre dois números consecutivos. Percebemos então que a PG:
 \begin{itemize}
  \item é constante, $a_n= a_{n-1} \Leftrightarrow q= 1$,
  \item é crescente, $a_n > a_{n-1} \Leftrightarrow a_1 > 0$ e $q > 1$, ou, $a_1 < 0$ e $0 < q < 1$,
  \item é decrescente, $a_n < a_{n-1} \Leftrightarrow a_1 > 0$ e $0 < q < 1$, ou, $a_1 < 0$ e $q > 1$.
 \end{itemize}

 \begin{exem} Exemplos de progressões geométricas:
  \begin{itemize}
   \item PG constante: $\{5, 5, 5, 5, \ldots \}$ com $a_1= 5$ e razão $q= 1$.
   \item PG crescente:
   \begin{enumerate}
   \item $a_1 > 0$ e $q > 1$: $\{2, 4, 8, 16, \ldots \}$ com $a_1= 2$ e razão $q= 2$.
   \item $a_1 < 0$ e $0 < q < 1$: $\{-81, -27, -9, -3, \ldots \}$ com $a_1= -81$ e razão $q= \frac{1}{3}$.
   \end{enumerate}
   \item PG decrescente:
    \begin{enumerate}
   \item $a_1 > 0$ e $0 < q < 1$: $\{4096, 1024, 256, 64, \ldots \}$ com $a_1= 4096$ e razão $q= \frac{1}{4}$.
   \item $a_1 < 0$ e $q > 1$: $\{-5, -10, -20, -40, \ldots \}$ com $a_1= -5$ e razão $q= 2$.
   \end{enumerate}
  \end{itemize}
 \end{exem}

 Dada uma PG é comum nos perguntarmos qual a soma dos $n$ primeiros termos desta PG, a fórmula que nos permite fazer este cálculo para uma PG com razão $q \neq 1$ é a seguinte:
 \[\destaque{S_n= a_1 \cdot \frac{ 1 - q^n}{1 - q}}\]
 Em que: $a_1=$ é o 1º termo, $n=$ nº de termos (até $a_n$), $q=$ é a razão da PG e $S_n=$ é a soma dos $n$ primeiros termos.

\newpage
