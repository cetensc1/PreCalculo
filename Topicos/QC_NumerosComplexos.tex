\section{Questões}
 
 \begin{enumerate}
  \item (NC-UFPR - 2017) Considere a seguinte equação com números complexos: $\frac{i - 1}{-2i - 2}$. Assinale a alternativa que apresenta a expressão equivalente a essa equação. 
  \begin{enumerate}[a)]
  \item $-0,5 i$.
  \item $0,5$.
  \item $i - 1$.
  \item $2i - 1$.
  \item $i - 2$.
  \end{enumerate}
 
  \item (IBFC - 2017) Considerando o conjunto numérico que contém as raízes da equação $x^2+1=0$. Os elementos desse conjunto numérico tem a forma $a+bi$, onde $a$ e $b$ são números reais e a unidade imaginária $i$ tem a propriedade $i^2=-1$. As informações referem-se ao conjunto dos números:
  \begin{enumerate}[a)]
  \item Racionais.
  \item Inteiros.
  \item Irracionais.
  \item Complexos.
  \item Naturais.
  \end{enumerate}
  
  \item (Big Advice - 2017) As afirmações a seguir referem-se aos números complexos 
 
   \begin{enumerate}[I)]
  \item $i^2= -1$.
  \item $i^3= -i$.
  \item $i^5= -i$.
  \end{enumerate}
 A alternativa correta é:
  \begin{enumerate}[a)]
  \item Apenas a I.
  \item Apenas a II.
  \item Apenas a III.
  \item I e II.
  \item I e III.
  \end{enumerate}
  
  \item (IBFC - 2016) O número complexo que representa o \textbf{conjugado} da soma entre os números complexos $z_1= 3 - 2i$ e $z_2 = 4 + 7i$ é igual a: 
  \begin{enumerate}[a)]
  \item $7+5i$.
  \item $-7+5i$.
  \item $-7-5i$.
  \item $7-5i$.
  \end{enumerate}
  
  \item (IDECAN - 2016) Analise as afirmativas a seguir, marque V para as verdadeiras e F para as falsas. 
 
   \begin{enumerate}[( \ \ )]
  \item $z = (2p + 8) + 3i$ é imaginário puro para $p = -4$.
  \item $z = (k + 2) + (k2 - 4)i$ é real e não nulo se $k = -2$.
  \item Se $z = a + bi$, então $z + \overline{z}$ é sempre real. 
  \end{enumerate}
  A sequência está correta em
  \begin{enumerate}[a)]
  \item V, F, V.
  \item V, F, F.
  \item V, V, F.
  \item F, F, V.
  \end{enumerate}
  
   \item (IFB - 2017) A corrente de um circuito elétrico utiliza em seu cálculo o quociente entre dois números complexos, onde o numerador é a fonte de tensão de uma residência e o denominador é uma carga de impedância. Do resultado deste cálculo utilizam-se as informações do módulo e do argumento para tomar as decisões. Se $a$, $b$ e $c$ são números complexos, tais que $a = -\sqrt{3}+ i$ e $b = 2i$ e $c = \frac{a}{b}$ , o \textbf{módulo} e o \textbf{argumento} do número complexo “$c$” são, respectivamente:  
 
  \begin{enumerate}[a)]
  \item $2$ e $\frac{\pi}{6}$.
  \item $1$ e $\frac{\pi}{3}$.
  \item $2$ e $\frac{\pi}{4}$.
  \item $\frac{1}{2}$ e $\frac{\pi}{3}$.
  \item $1$ e $\frac{\pi}{4}$.
  \end{enumerate}
 \end{enumerate}
 
 Gabarito: 1 a); 2 d); 3 d); 4 d); 5 a); 6 b).
