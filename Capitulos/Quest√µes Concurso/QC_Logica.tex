\section{Questões}

\begin{enumerate}
 \item (FEPESE - 2017) Um pai paga mesada mensal para seus cinco filhos, a saber, Cláudia, Arthur, Joaquim, Danilo e Beatriz, nos valores de R\$ 100, R\$ 200, R\$ 400, R\$ 600 e R\$ 700, não necessariamente nessa ordem.

Sabe-se que:
\begin{enumerate}[1.]
\item A mesada de Cláudia é o dobro da mesada de Beatriz.
\item O valor que Danilo recebe em um mês é igual ao que Beatriz recebe em três meses.
\item A mesada de Joaquim é maior que a de Danilo.
\end{enumerate}
Portanto, a mesada mensal de Arthur, em reais, é:
\begin{multicols}{5}
\begin{enumerate}
\item 100;
\item 200;
\item 400;
\item 600;
\item 700.
\end{enumerate}
\end{multicols}

\item (FCC - 2018) Os assistentes de trânsito de um município foram divididos em três grupos (A, B e C) para otimizar sua atuação nas quatro regiões da cidade (Norte, Sul, Leste e Oeste). Cada grupo deverá atuar em duas ou três regiões e cada região deverá receber assistentes de exatamente dois grupos. Em relação à distribuição, ficou decidido que os assistentes do:
\begin{itemize}
 \item grupo A deverão atuar em apenas duas regiões;
 \item grupo B não deverão atuar na região Norte;
 \item grupo C não deverão atuar na região Leste.
\end{itemize}
Dessa forma, as regiões onde os assistentes do grupo A deverão atuar são:
\begin{multicols}{2}
\begin{enumerate}[a)]
\item Norte e Leste;
\item Norte e Oeste;
\item Sul e Leste;
\item Sul e Oeste;
\item Oeste e Leste.
\end{enumerate}
\end{multicols}

\item (FUNDEP - 2018) Observe os números a seguir.
\[28; 13; 7; 22; 18; 3; 9 \]
Somando esses números em pares, obtém-se 31 como resultado, exceto com um deles, que sobra. Qual é o número que sobra?
\begin{multicols}{4}
\begin{enumerate}[a)]
\item 7;
\item 9;
\item 13;
\item 18.
\end{enumerate}
\end{multicols}

\item (FCC - 2017) Cássio, Ernesto, Geraldo, Álvaro e Jair são suspeitos de um crime. A polícia sabe que apenas um deles cometeu o crime. No interrogatório, os suspeitos deram as seguintes declarações:


Cássio: Jair é o culpado do crime.

Ernesto: Geraldo é o culpado do crime.

Geraldo: Foi Cássio quem cometeu o crime.

Álvaro: Ernesto não cometeu o crime.

Jair: Eu não cometi o crime.


Sabe-se que o culpado do crime disse a verdade na sua declaração. Dentre os outros quatro suspeitos, exatamente três mentiram na declaração. Sendo assim, o único inocente que declarou a verdade foi
\begin{multicols}{2}
\begin{enumerate}[a)]
\item Cássio;
\item Ernesto;
\item Geraldo;
\item Álvaro;
\item Jair.
\end{enumerate}
\end{multicols}

\item (UFES - 2017) Lucas e Mateus foram a uma loja para, cada um, comprar uma bola de um mesmo tipo, cujo preço era um número inteiro de reais. Porém, faltavam R\$ 20,00 para Lucas e R\$ 25,00 para Mateus para que cada um tivesse a quantia necessária para a compra. Os dois resolveram juntar as quantias de dinheiro que tinham e comprar uma única bola. Mesmo assim, a quantia total não era o suficiente para comprar nem uma bola. É CORRETO afirmar que o maior valor possível para o preço da bola é um inteiro

\begin{enumerate}[a)]
\item maior do que 29 e menor do que 35.
\item maior do que 34 e menor do que 40.
\item maior do que 39 e menor do que 45.
\item maior do que 44 e menor do que 50.
\item maior do que 49.
\end{enumerate}

\item (UFES - 2017) Em um guarda-roupa há, exclusivamente, seis peças: uma blusa azul, uma blusa marrom, uma blusa preta, uma saia azul, uma saia marrom e uma saia preta. Paula, Luciana e Renata foram juntas a uma festa e escolheram, cada uma, uma blusa e uma saia, nesse guarda-roupa, para vestirem. A blusa de Renata era azul, mas sua saia não era azul. A blusa de Luciana não era preta. A blusa e a saia de Paula eram da mesma cor. As cores das saias de Luciana e Renata eram, respectivamente:
\begin{multicols}{2}
\begin{enumerate}[a)]
\item marrom e preta.
\item azul e marrom.
\item azul e preta.
\item preta e marrom.
\item preta e preta.
\end{enumerate}
\end{multicols}

\item (INSTITUTO AOCP - 2017) João e Marcos resolveram iniciar uma sociedade para fabricação e venda de cachorro quente. João iniciou com um capital de R\$ 30,00 e Marcos colaborou com R\$ 70,00. No primeiro final de semana de trabalho, a arrecadação foi de R\$ 240,00 bruto e ambos reinvestiram R\$ 100,00 do bruto na sociedade, restando a eles R\$ 140,00 de lucro. De acordo com o que cada um investiu inicialmente, qual é o valor que João e Marcos devem receber desse lucro, respectivamente?
\begin{multicols}{2}
\begin{enumerate}[a)]
\item 30 e 110 reais.
\item 40 e 100 reais.
\item 42 e 98 reais.
\item 50 e 90 reais.
\item 70 e 70 reais.
\end{enumerate}
\end{multicols}

\item (FGV - 2017) Dois conjuntos A e B têm a mesma quantidade de elementos. A união deles tem 2017 elementos e a interseção deles tem 1007 elementos. O número de elementos do conjunto A é:
\begin{multicols}{2}
\begin{enumerate}[a)]
\item 505;
\item 1010;
\item 1512;
\item 1515;
\item 3014.
\end{enumerate}
\end{multicols}

\item (VUNESP - 2017) Tadeu verificou a capacidade total de uma jarra, de uma garrafa e de um copo, e estabeleceu as seguintes relações comparativas entre as respectivas capacidades:
\begin{itemize}
\item uma jarra equivale a três garrafas;
\item uma jarra mais uma garrafa equivalem a oito copos.
\end{itemize}
Pode-se concluir, então, que uma jarra equivale a
\begin{multicols}{2}
\begin{enumerate}[a)]
\item 3 copos;
\item 4 copos;
\item 5 copos;
\item 6 copos;
\item 7 copos.
\end{enumerate}
\end{multicols}

\item (FGV - 2017) O apresentador de um programa de auditório mostra no palco três portas, numeradas com 1, 2 e 3, e diz que atrás de cada uma delas há um prêmio: uma bicicleta, uma geladeira e um computador, não necessariamente nessa ordem. O apresentador sorteará uma pessoa do auditório, que deve escolher uma das portas e levar o seu prêmio.

Entretanto, se com as informações recebidas do apresentador a pessoa puder deduzir que objeto há atrás de cada porta, ela ganhará todos os prêmios.

As informações do apresentador são:
\begin{itemize}
 \item A geladeira não está na porta 1.
 \item A bicicleta e a geladeira não estão em portas com números consecutivos.
\end{itemize}
Então, é correto afirmar que:
\begin{enumerate}[a)]
\item a geladeira está na porta 2;
\item o computador está na porta 1;
\item a bicicleta está na porta 3;
\item a bicicleta está na porta 2;
\item o computador está na porta 2.
\end{enumerate}

\item (FGV - 2017) Considere as seguintes afirmativas:
\begin{itemize}
\item Se X é líquido, então não é azul.
\item Se X não é líquido, então é vegetal.
\end{itemize}
Pode-se concluir logicamente que:
\begin{enumerate}[a)]
\item se X é azul, então é vegetal;
\item se X é vegetal, então é azul;
\item se X não é azul, então não é líquido;
\item se X não é vegetal, então é azul;
\item se X não é azul, então não é vegetal.
\end{enumerate}

\item (FCC - 2017) A afirmação que corresponde à negação lógica da frase ‘Vendedores falam muito e nenhum estudioso fala alto’ é
\begin{enumerate}[a)]
\item ‘Nenhum vendedor fala muito e todos os estudiosos falam alto’.
\item ‘Vendedores não falam muito e todos os estudiosos falam alto’.
\item ‘Se os vendedores não falam muito, então os estudiosos não falam alto’.
\item ‘Pelo menos um vendedor não fala muito ou todo estudioso fala alto’.
\item ‘Vendedores não falam muito ou pelo menos um estudioso fala alto’.
\end{enumerate}

\item (Quadrix - 2017) Em Lógica Matemática, chamamos de proposição toda sentença declarativa afirmativa que pode ser classificada como verdadeira ou falsa, mas não como verdadeira e falsa simultaneamente. Considere as sentenças abaixo, em que a sigla CRF-MT significa Conselho Regional de Farmácia do Mato Grosso.
\begin{enumerate}[I.]
\item A capital do Mato Grosso é Cuiabá.
\item O CRF-MT possui sede em Uberlândia.
\item Não deixe de resolver essa prova com a devida atenção.
\item Ele foi o primeiro colocado no concurso do CRF-MT em 2014.
\item O CRF-MT possui 76 funcionários concursados.
\end{enumerate}
Diante do exposto, é correto afirmar que, dentre as sentenças acima, aquelas que não podem ser consideradas proposições são somente as identificadas com os algarismos romanos:
\begin{multicols}{2}
\begin{enumerate}[a)]
\item I, II e III.
\item IV e V.
\item I, II e V.
\item III e IV.
\item III, IV e V.
\end{enumerate}
\end{multicols}

\item (Quadrix - 2017) Considere a afirmação: “Jorge não almoçou e foi estudar para o concurso”. A negação dessa afirmação é:
\begin{enumerate}[a)]
\item Jorge almoçou e não foi estudar para o concurso.
\item Jorge almoçou ou não foi estudar para o concurso.
\item Jorge não almoçou e não foi estudar para o concurso.
\item Jorge não almoçou ou não foi estudar para o concurso.
\item Jorge estudou para o concurso e não almoçou.
\end{enumerate}

\item (Lógica - Fundatec - 2018) A tabela-verdade da fórmula $\neg(P \lor Q) \rightarrow Q$:
\begin{enumerate}[a)]
\item Só é falsa quando $P$ e $Q$ são falsos.
\item É uma tautologia.
\item É uma contradição.
\item Só é falsa quando $P$ e $Q$ são verdadeiras.
\item Só é falsa quando $P$ é verdadeiro e $Q$ é falso.
\end{enumerate}

\item (Lógica - Fundatec - 2018) Se Ana e Beatriz são estagiárias então Carla é estagiária. Entretanto, Carla não é estagiária ou Daniela é estagiária. Da hipótese de Daniela não ser estagiária, deduzimos que:
\begin{enumerate}[a)]
\item Ana e Beatriz não são estagiárias.
\item Ana ou Beatriz não são estagiárias.
\item Ana é estagiária e Beatriz não é estagiária.
\item Ana não é estagiária e Beatriz é estagiária.
\item Carla é estagiária e Beatriz não é estagiária.
\end{enumerate}

\item (Lógica - Fundatec - 2018) A negação da sentença: Se o projeto de lei não foi analisado pela comissão então a votação ocorrerá após o recesso legislativo é:
\begin{enumerate}[a)]
\item Se o projeto não foi analisado pela comissão, então a votação não ocorrerá após o recesso legislativo.
\item Se o projeto de lei foi analisado pela comissão, então a votação não ocorrerá após o recesso legislativo.
\item O projeto de lei foi analisado pela comissão e a votação não ocorrerá após o recesso legislativo.
\item O projeto de lei não foi analisado pela comissão e a votação não ocorrerá após o recesso legislativo.
\item O projeto de lei foi analisado pela comissão e a votação ocorrerá após o recesso legislativo.
\end{enumerate}

  \item (UDESC/Fundatec - 2015) Considere a seguinte proposição composta.

  \textbf{Se Ana não está com a carteira de habilitação vencida então ela pode dirigir o carro.}

  Temos uma proposição composta falsa quando:

  \begin{enumerate}
  \item A sentença proposicional simples Ana está com a carteira de habilitação vencida é verdadeira, e a sentença proposicional simples Ana pode dirigir seu carro é verdadeira.
  \item A sentença proposicional simples Ana está com a carteira de habilitação vencida é falsa, e a sentença proposicional simples Ana pode dirigir seu carro é falsa.
  \item A sentença proposicional simples Ana está com a carteira de habilitação vencida é verdadeira, e a sentença proposicional simples Ana pode dirigir seu carro é falsa.
  \item A sentença proposicional simples Ana não está com a carteira de habilitação vencida é falsa, e a sentença proposicional simples Ana pode dirigir seu carro é falsa.
  \item A sentença proposicional simples Ana está com a carteira de habilitação vencida é falsa, e a sentença proposicional simples Ana pode dirigir seu carro é verdadeira.
 \end{enumerate}


  \item (UDESC/Fundatec - 2015) Considere a figura associada à tabela-verdade inicial da fórmula $P \rightarrow \sim R \lor S$, onde apresentamos as colunas iniciais das interpretações do valor-lógico dos símbolos proposicionais $P$, $R$ e $S$. A avaliação correta da última coluna da correspondente tabela-verdade, onde $\sim$ representa o conetivo da negação, $\rightarrow$ representa o conetivo do condicional.

  \begin{table}[H]
 \centering
 \begin{tabular}{|c|c|c|} \hline
 $P$ & $R$ & $S$ \\ \hline
 V & V & V  \\ \hline
 V & V & F  \\ \hline
 V & F & V  \\ \hline
 V & F & F  \\ \hline
 F & V & V  \\ \hline
 F & V & F  \\ \hline
 F & F & V  \\ \hline
 F & F & F  \\ \hline
 \end{tabular}
 \end{table}

  \begin{multicols}{5}
  \begin{enumerate}
  \item
  \begin{table}[H]
 \begin{tabular}{|c|} \hline
 V   \\ \hline
 F  \\ \hline
 V   \\ \hline
 V   \\ \hline
 V   \\ \hline
 V   \\ \hline
 V  \\ \hline
 V   \\ \hline
 \end{tabular}
 \end{table}

  \item
  \begin{table}[H]
 \begin{tabular}{|c|} \hline
 V   \\ \hline
 F  \\ \hline
 V   \\ \hline
 V   \\ \hline
 F   \\ \hline
 F   \\ \hline
 F  \\ \hline
 F   \\ \hline
 \end{tabular}
 \end{table}

  \item
  \begin{table}[H]
 \begin{tabular}{|c|} \hline
 V   \\ \hline
 F  \\ \hline
 V   \\ \hline
 V   \\ \hline
 V   \\ \hline
 V   \\ \hline
 F  \\ \hline
 F   \\ \hline
 \end{tabular}
 \end{table}

  \item
  \begin{table}[H]
 \begin{tabular}{|c|} \hline
 V   \\ \hline
 F  \\ \hline
 F   \\ \hline
 F   \\ \hline
 V   \\ \hline
 V   \\ \hline
 F  \\ \hline
 F   \\ \hline
 \end{tabular}
 \end{table}

  \item
  \begin{table}[H]
 \begin{tabular}{|c|} \hline
 V   \\ \hline
 F  \\ \hline
 V   \\ \hline
 V   \\ \hline
 F   \\ \hline
 F  \\ \hline
 V  \\ \hline
 V   \\ \hline
 \end{tabular}
 \end{table}

 \end{enumerate}
 \end{multicols}

  \item (UDESC/Fundatec - 2015) Sejam $A$, $B$ e $C$ proposições simples e $\sim A$, $\sim B$ e $\sim C$ as respectivas negações, os conectivos da conjunção, disjunção e condicional são, respectivamente, representados por: $\land$, $\lor$, $\rightarrow$. Assim, a fórmula proposição $\sim ( A \lor \sim B) \rightarrow \sim C$ é equivalente a:

  \begin{enumerate}
  \item $(\sim A \land \sim B) \lor C$
  \item $\sim A \lor B \lor \sim C$
  \item $A \lor \sim (B \lor C)$
  \item $A \lor \sim (B \land C)$
  \item $\sim (A \lor B \lor C)$
 \end{enumerate}

   \item (TJ/SC - 2018) Maria é mais nova que Roberta e Joana é mais velha que Sílvia, que tem a mesma idade de Roberta. É correto concluir que:
    \begin{enumerate}
  \item Maria é mais velha que Sílvia
  \item Roberta é mais jovem que Joana (*)
  \item Maria é mais velha que Joana
  \item Sílvia é mais jovem que Maria
  \item Maria e Joana tem a mesma idade.
 \end{enumerate}

  \item (TJ/SC - 2018) Em um escritório há pastas azuis, verdes e marrons. O chefe do escritório disse ao estagiário:

  Processos trabalhistas são colocados em pastas verdes.

  É correto concluir que:
  \begin{enumerate}
  \item processo não trabalhista não é colocado em pasta verde;
  \item dentro de uma pasta verde há sempre um processo trabalhista;
  \item dentro de uma pasta azul não há um processo trabalhista; (*)
  \item um processo penal é colocado em pasta marrom;
  \item pelo menos um processo penal está em pasta azul.
 \end{enumerate}

  \item (TJ/SC - 2018) Considere a afirmação: ``Nenhum médico é cego''. A negação dessa afirmação é:
  \begin{enumerate}
  \item Há, pelo menos, um médico cego; (*)
  \item Nenhum cego é médico;
  \item Todos os médicos são cegos;
  \item Todos os cegos são médicos;
  \item Todos os médicos não são cegos.
 \end{enumerate}

 \item (TJ/SC - 2018) Alberto disse: ``Se chego tarde em casa, não ligo o computador e, se não ligo o computador, vou cozinhar. Porém sempre ligo o computador, tomo café''.

  Certo dia, Alberto chegou em casa e não tomou café.

  É correto concluir que Alberto:
  \begin{enumerate}
  \item cozinhou; (*)
  \item chegou tarde;
  \item não cozinhou;
  \item chegou cedo;
  \item ligou o computador.
 \end{enumerate}

\end{enumerate}

Gabarito: 1 a), 2 a), 3 a), 4 e), 5 c), 6 b), 7 c), 8 c), 9 d), 10 e), 11 a), 12 e), 13 d), 14 b), 15 a), 16 a), 17 d), 18 b), 19 a), 20 d).

% \begin{enumerate}[a)]
% \item
% \item
% \item
% \item
% \item
% \end{enumerate}
