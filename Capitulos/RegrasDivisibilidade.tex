\chapter{Regras de Divisibilidade}
 \begin{itemize}
  \item \textbf{Divisibilidade por $1$}

 Todo número é divisível por $1$.

 \item \textbf{Divisibilidade por $2$}

 Um número é divisível por $2$ (ou seja, par) quando o seu dígito das unidades é igual a $0$, $2$, $4$, $6$ ou $8$.

 \item \textbf{Divisibilidade por $3$}

 Um número é divisível por $3$ se a soma de seus dígitos é um múltiplo de $3$.

 \item \textbf{Divisibilidade por $4$}

 Um número é divisível por $4$ quando o dobro do dígito das dezenas somado com o dígito das unidades é divisível por $4$.

 \item \textbf{Divisibilidade por $5$}

 Um número é divisível por $5$ quando termina em $0$ ou $5$.

 \item \textbf{Divisibilidade por $6$}

 Um número é divisível por $6$ quando é divisível por $2$ e por $3$.

 \item \textbf{Divisibilidade por $7$}

 Um número é divisível por $7$ quando, ao subtrair o dobro do último dígito do número formado pelos demais dígitos, o resultado é um número divisível por $7$.

 \item \textbf{Divisibilidade por $8$}

 Um número é divisível por $8$ quando o número formado por seus três últimos dígitos é divisível por $8$ (isto inclui o caso em que o número termina em $000$).

 \item \textbf{Divisibilidade por $9$}

 É divisível por $9$ todo número em que a soma de seus dígitos constitui um número múltiplo de $9$.

 \item \textbf{Divisibilidade por $10$}

 Um número é divisível por $10$ quando terminar em $0$.

 \item \textbf{Divisibilidade por $12$}

 Um número é divisível por $12$ quando é divisível por $3$ e por $4$.

 \item \textbf{Divisibilidade por $15$}

 Um número é divisível por $15$ quando é divisível por $3$ e por $5$.
 \end{itemize}