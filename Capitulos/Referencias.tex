 \phantomsection
\chapter*{Referências}
\addcontentsline{toc}{chapter}{Referências}

 \begin{enumerate}[1.]
 
 \item ADAMI, Adriana Miorelli; DORNELLES FILHO, Adalberto Ayjara; LORANDI, Magda Mantovani. \textbf{Pré-cálculo.} Porto Alegre: Bookman, 2015.
 
 \item BONETTO, Giácomo Augusto; MUROLO, Anfrânio Carlos. \textbf{Fundamentos de matemática para engenharia e tecnologias.} São Paulo: Cengarge Learning, 2016.
 
 \item COLLINGWOOD, David H.; PRINCE, K. David; CONROY, Matthew M.. \textbf{Precalculus.} S.i: S.i., 2011. Disponível em: \href{https://sites.math.washington.edu/~colling/HSMath120/TB201112.pdf}{Link} Acesso em: 11 jul. 2019.
 
 \item DANTE, Luiz Roberto. \textbf{Matemática: Ensino médio.} São Paulo: Editora ática, 2008.
 
 \item DOMINGUES, Hygino Hugueros. \textbf{Fundamentos da Aritmética.} 2. ed. Florianópolis: Editora da UFSC, 2017.

  \item  GOMES, Franscisco Magalhães. \textbf{Pré-cálculo:} operações, equações, funções e sequências. São Paulo: Cengage Learning, 2018.
 
  \item GUIDORIZZI, Hamilton Luiz. \textbf{Um curso de cálculo.} 5. ed. Rio de Janeiro: Ltc, 2013. v. 1.
  
   \item IEZZI, Gelson et al. \textbf{Fundamentos da Matemática Elementar:} Conjuntos e Funções. 6. ed. São Paulo: Atual, 1985. v. 1.
   
   \item IEZZI, Gelson et al. \textbf{Fundamentos da Matemática Elementar:} Logaritmos. 7. ed. São Paulo: Atual Editora, 1985. v. 2.
   
   \item IEZZI, Gelson et al. \textbf{Fundamentos da Matemática Elementar:} Trigonometria. 6. ed. São Paulo: Atual Editora, 1985. v. 3.
   
   \item IEZZI, Gelson et al. \textbf{Fundamentos da Matemática Elementar:} Complexos, polinômios, equações. 5. ed. São Paulo: Atual Editora, 1985. v. 6.
     
  \item KUHLKAMP, Nilo. \textbf{Cálculo 1.} 5. ed. Florianópolis: Editora da UFSC, 2015.
  
  \item LANG, Serge. \textbf{Estruturas Algébricas.} Rio de Janeiro: Ao Livro Técnico, 1972.
  
  \item LARSON, Ron. \textbf{Precalculus.} S.i: Cengage Learning, 2016.
  
  \item LIAL, Margaret L. et al. \textbf{Precalculus.} 6. ed. Boston: Pearson, 2017.
  
  \item SOARES, Marcio G. \textbf{Cálculo em uma variável complexa.} 5, ed. Rio de Janeiro: IMPA, 2009.
  
  \item SPIEGEL, Murray R.. \textbf{Álgebra.} 3. ed. Porto Alegre: Bookman, 2015. (Coleção Sachaum).
  
  \item STEWART, James; REDLIN, Lothar; WATSON, Saleem. \textbf{Precalculus:} Mathematics for calculus. 7. ed. S.i: Cengage Learning, 2014.
  
  \item SULLIVAN, Michael. \textbf{Precalculus.} 10. ed. Chicago State University: Pearson, .
  
  \item \textbf{Regras de divisibilidade.} Fonte: \href{http://mundoeducacao.bol.uol.com.br/matematica/regras-divisibilidade.htm}{Mundo Educação}. Acesso em: 18 de janeiro de 2018.

 

 
  
  % \item Martins, Márcia da Silva. \textbf{Lógica - Uma abordagem introdutória}. Rio de Janeiro: Editora Ciência Moderna Ltda., 2012.

 % \item Souza, João Nunes de. \textbf{Lógica para ciência da computação: uma introdução concisa}. 2 ed. Rio de Janeiro: Elsevier, 2008.

 % \item Alencar Filho, Edgard. \textbf{Iniciação à Lógica Matemática}. São Paulo: Nobel, 2002.

 % \item Dutra, Maurici José. \textbf{Matemática financeira: livro didático}. 8 ed. rev. e atual. Palhoça: UnisulVirtual, 2010.

  \end{enumerate}
