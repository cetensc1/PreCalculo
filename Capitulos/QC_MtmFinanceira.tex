\section{Exercícios}
 
 \begin{exer}
 
  (UFPE/Covest - 2015) Qual deve ser o prazo de aplicação de um capital, a uma taxa de juros simples e anuais de 15\%, para que os juros correspondam a três vezes o valor do capital?
  \begin{enumerate}[a)]
  \item 14 anos
  \item 16 anos
  \item 18 anos
  \item 20 anos
  \item 22 anos
  \end{enumerate}
  \end{exer}
  
  \begin{exer}
  (UFPE/Covest - 2015) Acrescido de juros simples pelo prazo de cinco meses, um capital aplicado resultou num montante de R\$ 18.900,00. O mesmo capital, acrescentado de juros simples pelo prazo de oito meses, aplicado à mesma taxa que o anterior, resultou num montante de R\$ 19.440,00. Qual a taxa anual de juros?
  \begin{enumerate}[a)]
  \item 1\%
  \item 4\%
  \item 6\%
  \item 8\%
  \item 12\%
  \end{enumerate}
  \end{exer}

  \begin{exer}
  (FGV - 2018) Suponha que um investidor tenha o objetivo de quadruplicar o seu capital em um investimento que remunere a taxa de juros de 1\% ao mês, sob o regime de juros simples. Assinale a opção que indica o tempo necessário para atingir esse objetivo.
  \begin{enumerate}[a)]
  \item 139 meses
  \item 11 anos e 7 meses
  \item 300 anos
  \item 25 anos
  \item 2 anos e meio
  \end{enumerate}
  \end{exer}

 \begin{exer}
  (FGV - 2018) Suponha que um consumidor entre em uma loja e verifique que o preço à vista de um ar condicionado é de R\$ 1.000,00. No entanto, ele opta pelo financiamento, que exige uma entrada de R\$ 200,00 e mais duas parcelas mensais iguais com taxa efetiva mensal de juros de 1\%, sob regime composto. O valor de cada parcela será igual a:
  \begin{enumerate}[a)]
  \item R\$ 808,00
  \item R\$ 406,01
  \item R\$ 404,25
  \item R\$ 402,50
  \item R\$ 507,51
  \end{enumerate}
  \end{exer}

  \begin{exer}
  (FCC - 2017) A aplicação de um capital, no valor de R\$ 900.000, em determinada instituição financeira, por um período de seis meses, foi resgatado pelo valor de R\$ 1.035.000. Considerando-se que o capital foi aplicado a juros simples, a taxa de juros ao mês foi de
  \begin{enumerate}[a)]
  \item 2,5\%
  \item 0,15\%
  \item 3,0\%
  \item 2,0\%
  \item 4,0\%
  \end{enumerate}
  \end{exer}

  \begin{exer}
  (FCC - 2015) Clóvis comparou os preços de um fogão em duas lojas. Na loja A o preço do fogão era R\$ 850,00 e a loja oferecia um desconto de 10\% para quem pagasse à vista. Na loja B o preço do fogão era R\$ 950,00 e a loja oferecia um desconto de 20\% para quem pagasse à vista. Clóvis comprou o fogão à vista na loja cujo preço, à vista, era mais vantajoso. Agindo dessa maneira, Clóvis economizou, em relação às condições de preço à vista da outra loja, a quantia de
  \begin{enumerate}[a)]
  \item R\$ 12,00
  \item R\$ 25,00
  \item R\$ 8,00
  \item R\$ 5,00
  \item R\$ 9,00
  \end{enumerate}
  \end{exer}

    \begin{exer}
  (COSEAC - 2015) Uma empresa fez um empréstimo de R\$ 100.000,00 com duração de dois anos à taxa de juros compostos de 4,2\% ao ano. O valor pago no fim do período foi de:
  \begin{enumerate}[a)]
  \item $R\$ 108.400,00$
  \item $R\$ 108.254,30$
  \item $R\$ 106.562,25$
  \item $R\$ 108.576,40$
  \item $R\$ 106.240,45$
  \end{enumerate}
 \end{exer}

 Gabarito: 1 d); 2 e); 3 d); 4 b); 5 a); 6 d); 7 d).
 

   \begin{exer}
  (Exatus-PR) Considere que uma caixa de bombom custava, em novembro, R\$ 8,60 e passou a custar, em dezembro, R\$ 10,75. O aumento no preço dessa caixa de bombom foi de:
  \begin{enumerate}[a)]
  \item $30\%$
  \item $25\%$
  \item $20\%$
  \item $15\%$
  \end{enumerate}
  \end{exer}

    \begin{exer}
  (UNA Concursos - 2015) Ao comprar um sofá que custava R\$ 3.800,00 Antônio recebeu um desconto de $11,5\%$. Sendo assim, quanto ele pagou pelo produto?
   \begin{enumerate}[a)]
  \item $R\$ 437,00$
  \item $R\$ 3.363,00$
  \item $R\$ 2.650,00$
  \item $R\$ 3.987,00$
  \end{enumerate}
  \end{exer}

 \begin{exer}
  (BIO-RIO - 2015) Com a inflação, mês passado um comerciante aumentou o preço de seus produtos em 20\%. Agora ele está arrependido porque as vendas caíram muito. Assim, ele resolveu baixar os preços atuais em 20\%. Dessa forma, o preço final a ser cobrado depois desse desconto, comparado com o preço inicial, de antes do aumento, será:
  \begin{enumerate}[a)]
  \item $4\%$ mais barato.
  \item $2\%$ mais barato.
  \item igual.
  \item $2\%$ mais caro.
  \item $4\%$ mais caro.
  \end{enumerate}
  \end{exer}

  \begin{exer}
  (FGV - 2015) Francisco compra e vende terrenos em um determinado condomínio. Certa semana ele comprou dois terrenos, um por 15.000 reais e outro por 25.000 reais, e, na semana seguinte, Francisco vendeu o primeiro com lucro de 40\% e o segundo com lucro de 8\%.

  Em toda a operação, o lucro de Francisco em relação ao capital investido foi de:
  \begin{enumerate}[a)]
  \item $20\%$
  \item $22\%$
  \item $24\%$
  \item $26\%$
  \item $28\%$
  \end{enumerate}
  \end{exer}

  \begin{exer}
  (UFPE/Covest - 2015) Um banco cobra juros compostos e mensais para dívidas não pagas no cheque especial. Se uma dívida não paga de R\$ 600,00, no cheque especial, se transforma em um débito de R\$ 1.884,00, em um período de um ano, qual a taxa mensal de juros do cheque especial? Dado: use a aproximação $3,14^{\frac{1}{12}} \approx 1,10$.
  \begin{enumerate}
  \item 8\%
  \item 9\%
  \item 10\%
  \item 11\%
  \item 12\%
  \end{enumerate}
  \end{exer}

  \begin{exer}
  (VUNESP - 2017) Um produto foi comprado em 2 parcelas, a primeira à vista e a segunda após 3 meses, de maneira que sobre o saldo devedor, incidiram juros simples de 2\% ao mês. Se o valor das 2 parcelas foi igual a R\$ 265,00, o preço desse produto à vista é:
  \begin{enumerate}[a)]
  \item $R\$ 530,00$
  \item $R\$ 515,00$
  \item $R\$ 500,00$
  \item $R\$ 485,00$
  \item $R\$ 460,00$
  \end{enumerate}
  \end{exer}

  \begin{exer}
  (PR-4 UFRJ - 2017) Uma dívida bancária de R\$ 950,00 foi quitada 2 quadrimestres depois de contraída. A taxa linear mensal praticada pela instituição financeira, que resultou na cobrança de juros de R\$ 433,20, foi de:
   \begin{enumerate}[a)]
  \item $5,7\%$
  \item $57\%$
  \item $4,7\%$
  \item $0,57\%$
  \item $47\%$
  \end{enumerate}
  \end{exer}

  \begin{exer}
  (VUNESP - 2017) Um capital de R\$ 1.500,00 aplicado a juro simples durante 9 meses rendeu juros de R\$ 81,00. A taxa anual de juros dessa aplicação foi:
  \begin{enumerate}[a)]
  \item $7,2\%$
  \item $6,8\%$
  \item $6,3\%$
  \item $5,5\%$
  \item $5,2\%$
  \end{enumerate}
  \end{exer}

  \begin{exer}
  (UFES - 2017) No regime de juros simples, os juros em cada período de tempo são calculados sobre o capital inicial. Um capital inicial $C_0$ foi aplicado a juros simples de $3\%$ ao mês. Se $C_n$ é o montante quando decorridos $n$ meses, o menor valor inteiro para $n$, tal que $C_n$ seja maior que o dobro de $C_0$, é
  \begin{enumerate}[a)]
  \item $30$
  \item $32$
  \item $34$
  \item $36$
  \item $38$
  \end{enumerate}
  \end{exer}

  \begin{exer}
  (VUNESP - 2017) Dois capitais distintos, $C_1$ e $C_2$, sendo $C_2$ maior que $C_1$, foram aplicados por prazos iguais, a uma mesma taxa de juros simples e geraram, ao final da aplicação, montantes iguais a $9/8$ dos respectivos capitais iniciais. Se a diferença entre os valores recebidos de juros pelas duas aplicações foi igual a R\$ $500,00$, então $C_2 – C_1$ é igual a
  \begin{enumerate}[a)]
  \item $R\$ 3.000,00$
  \item $R\$ 4.000,00$
  \item $R\$ 5.000,00$
  \item $R\$ 6.000,00$
  \item $R\$ 8.000,00$
  \end{enumerate}
  \end{exer}

 \begin{exer}
  (CRA-SC - 2017) O capital de \$ $1.000,00$ foi aplicado gerando o montante de \$ $1.600,00$ no regime dos juros simples. Sabendo-se que a taxa de juros empregada foi de $36\%$ ao ano, pergunta-se qual foi o prazo da aplicação?
  \begin{enumerate}[a)]
  \item $20$ meses.
  \item $24$ meses.
  \item $18$ meses.
  \item $22$ meses.
  \item $26$ meses.
  \end{enumerate}
  \end{exer}

  \begin{exer}
  (RBO - 2017) Um capital aplicado à taxa de juros simples de $3,5\%$ ao mês durante um ano produziu um montante de $R\$ 3.550,00$. O valor de juros produzido por esse capital é igual a
  \begin{enumerate}[a)]
  \item $R\$ 950,00$
  \item $R\$ 1.050,00$
  \item $R\$ 1.250,00$
  \item $R\$ 1.350,00$
  \item $R\$ 1.550,00$
  \end{enumerate}
  \end{exer}

  \begin{exer}
  (PR-4 UFRJ - 2017) Uma aplicação de R\$ 30.000,00 a uma taxa mensal de 4\% no regime de capitalização composta, ao final de um bimestre, gera um capital acumulado de:
  \begin{enumerate}[a)]
  \item $R\$ 32.400,00$
  \item $R\$ 31.827,00$
  \item $R\$ 32.448,00$
  \item $R\$ 33.120,26$
  \item $R\$ 33.200,14$
  \end{enumerate}
  \end{exer}

  \begin{exer}
  (FCM - 2017) A empresa Good Finance aplicou em uma renda fixa um capital de 100 mil reais, com taxa de juros compostos de 1,5\% ao mês, para resgate em 12 meses. O valor recebido de juros ao final do período foi de:
  \begin{enumerate}[a)]
  \item $R\$ 10.016,00$
  \item $R\$ 25.254,24$
  \item $R\$ 16.361,26$
  \item $R\$ 18.000,00$
  \item $R\$ 19.561,82$
  \end{enumerate}
  \end{exer}

  \begin{exer}
  (IESES - 2017) Uma aplicação financeira de \$ 1.500,00 é feita por 2 trimestres a taxa de juros compostos de 5\% ao mês. Qual será o valor do montante?
  \begin{enumerate}[a)]
  \item $\$ 1.840,64$
  \item $\$ 1.950,00$
  \item $\$ 2.010,14$
  \item $\$ 2.143,58$
  \end{enumerate}
  \end{exer}

  \begin{exer}
  (CONPASS - 2016) Considere uma aplicação de R\$ 3000,00 a uma taxa mensal de juros compostos de 1,5\%. Qual o tempo necessário para se obter um montante de R\$ 5000,00?
  \begin{enumerate}[a)]
  \item $40$ meses.
  \item $30$ meses.
  \item $24$ meses.
  \item $45$ meses.
  \item $35$ meses.
  \end{enumerate}
  \end{exer}

  \begin{exer}
  (UFES - 2016) Joana investiu um capital inicial a juros compostos de 10\% ao mês. Depois de 3 meses, o montante era igual a R\$ 5.324,00. O capital inicial era igual a
  \begin{enumerate}[a)]
  \item $R\$ 4.020,00$
  \item $R\$ 3.990,00$
  \item $R\$ 4.010,00$
  \item $R\$ 3.980,00$
  \item $R\$ 4.000,00$
  \end{enumerate}
  \end{exer}

  \begin{exer}
  (Quadrix - 2015) Um investidor aplica um capital de R\$ 12.000,00 em um fundo de investimento, durante um período de 5 meses, à taxa de 10\% ao mês. Qual é o valor do montante gerado, considerando-se juros compostos?
  \begin{enumerate}[a)]
  \item $R\$ 6.000,00$
  \item $R\$ 18.000,00$
  \item $R\$ 13.527,15$
  \item $R\$ 18.315,42$
  \item $R\$ 19.326,12$
  \end{enumerate}
  \end{exer}
 




