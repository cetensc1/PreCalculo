\chapter{Matemática Financeira}

\section{Porcentagem no mundo das finanças}

No contexto da matemática financeira, a porcentagem é utilizada por exemplo para o cálculo de descontos, lucros e juros (simples e compostos), estando portanto presente na maioria de nossas transações financeiras. Devido a sua importância, vamos revisitar o conceito de porcentagem, através de dois exemplos do mundo financeiro, para depois definir os conceitos específicos desta área da matemática.

\begin{exem}
Qual é o valor de $40 \%$ de R\$ $92,00$?

Podemos resolver isso de duas formas:

\textsf{1ª forma:}

\[40 \% \text{ de } 92,00= \frac{40}{100} \cdot 92= \frac{40\cdot 92}{100}= 36,80\]

\textsf{2ª forma:}

\begin{eqnarray*}
  \text{Reais} & & \text{Porcentagem} \\
  92 & = & 100 \\
  x  & = & 40  \\
  100x= 40 \cdot 92 &\Rightarrow& x= \frac{40\cdot 92}{100}= 36,80
 \end{eqnarray*}

\end{exem}

 O interessante desta 2ª forma é que ela nos permite resolver também o seguinte exemplo:
 \begin{exem}
  Se na compra de um produto no valor de R\$ $160,00$ durante uma promoção, pagamos apenas R\$ $120,00$ de quantos por cento foi o desconto?

   Note que $160 - 120= 40$ logo o desconto foi de R\$ $40,00$ reais, como,
   \begin{eqnarray*}
  \text{Reais} & & \text{Porcentagem} \\
  160 & = & 100 \\
  40  & = & x  \\
  160x= 40 \cdot 100 &\Rightarrow& x= \frac{40\cdot 100}{160}= 25
 \end{eqnarray*}
 decorre que o desconto foi de $25 \%$.
 \end{exem}

 \section{Matemática financeira}

 Entendido porcentagem vamos agora ao foco deste capítulo que é a matemática financeira e para começar definimos na tabela abaixo alguns termos próprios desta área:
 \begin{table}[h]
\centering
 \begin{tabular}{|c|p{9cm}|} \hline
 Capital (\textbf{C})&  Quantia de dinheiro disponível, valor que está sendo investido ou emprestado. \\ \hline
 Juros (\textbf{J})  &  Remuneração recebida pelo uso de um capital por um intervalo de tempo.
                        Compensação paga pelo tomador do empréstimo (ou receptor do investimento)
                        para ter o direito de usar o dinheiro até o dia do pagamento.
                        Pode ser expresso em valor monetário (\$) ou como uma taxa de juro (\%). \\\hline
 Prazo (\textbf{n})  &  Número de períodos que compõem o intervalo de tempo utilizado. \\\hline
 Montante (\textbf{M})  & Soma do capital aplicado ou emprestado mais juros. \[\destaque{M= C + J}\]  \\\hline
 Taxa de juros (\textbf{i})  & É o coeficiente resultante da razão entre o juro e o capital. A cada taxa, deverá vir anexado o período a que ela se refere. \[\destaque{i = \frac{J}{C}}\] \\\hline
 \end{tabular}
\end{table}

 \subsection{Taxa de juros \texorpdfstring{$i$}{i}}

 A taxa de juros $i$ indica qual remuneração será paga ao dinheiro emprestado, ou quanto o investidor irá receber pelo dinheiro investido, em um determinado período. Ela vem normalmente expressa da forma percentual, seguida da especificação do período de tempo a que se refere:

 $8 \% a.a.$ - (a.a. significa ao ano).

 $10 \% a.t.$ - (a.t. significa ao trimestre).

Outra forma de apresentação da taxa de juros é a unitária, que é igual a taxa percentual dividida por 100, sem o símbolo $\%$:

 $0,15 a.m.$ - (a.m. significa ao mês).

 $0,10 a.q.$ - (a.q. significa ao quadrimestre).

 \newpage

 \section{Regimes de formação de juros}

 Existem dois tipos de juros, são eles:
 \begin{itemize}
  \item Juros Simples;
  \item Juros compostos.
 \end{itemize}
 Em ambos os juros são calculados por períodos, a diferença entre eles é que no primeiro os juros são calculados sempre sobre o capital inicial (principal), enquanto que no segundo os juros gerados em um período são incorporados ao capital inicial, formando um novo capital sobre o qual serão calculados os juros do período seguinte.

 \begin{exem}
  Ao aplicarmos um capital de R\$$3.000,00$ por $4$ anos, a uma taxa de juros de $12\%$ a.a. no regime de juros simples ou compostos, obtemos os seguintes resultados:
\begin{table}[h]
\centering
\label{my-label}
\begin{tabular}{|c|c|c|c|c|} \hline
\multirow{2}{*}{Período} & \multicolumn{2}{|c|}{Juros Simples} & \multicolumn{2}{|c|}{Juros Compostos} \\
    & Juros  & Montante & Juros  & Montante \\\hline
  0 & -      & 3.000,00 & -      & 3.000,00 \\\hline
  1 & 360,00 & 3.360,00 & 360,00 & 3.360,00 \\\hline
  2 & 360,00 & 3.720,00 & 403,20 & 3.763,20 \\\hline
  3 & 360,00 & 4.080,00 & 451,58 & 4.214,78 \\\hline
  4 & 360,00 & 4.440,00 & 505,77 & 4.720,56 \\\hline
\end{tabular}
\end{table}
 \end{exem}

 Uma vez entendido a diferença entre os dois tipos de juros, vamos agora nos aprofundar em cada um deles.

 \subsection{Juros simples}

 Lembremos que no regime do juros simples:
 \begin{itemize}
  \item O juro é calculado apenas sobre o capital inicial.
  \item Os juros são iguais em todos os períodos, sendo por isso denominado constante.
 \end{itemize}

 Destas características do juros podemos concluir que:
  \begin{eqnarray*}
  J_1&=& C\cdot i \\
  J_2&=& C\cdot i + C\cdot i= C\cdot 2i \\
  J_3&=& C\cdot i + C\cdot i + C\cdot i= C\cdot 3i \\
  \vdots \\
  J_n&=& C \cdot n \cdot i
  \end{eqnarray*}

 Portanto, o juro simples após um determinado período $n$ pode ser calculado através da fórmula:
  \[\destaque{J= C \cdot n \cdot i}\]

 Assim o montante resultante após um período $n$ de capitalização sob juros simples é:
 \[M= C + J = C + C \cdot n \cdot i \Rightarrow \destaque{M=  C \cdot (1 + n \cdot i)}\]

 Da fórmula do montante, dada por, $M= C \cdot (1 + n \cdot i)$ decorrem as fórmulas para os cálculos do período, da taxa, e do capital.
 \begin{obs}
 \begin{itemize}
  \item Nos cálculos de juros é necessário que a taxa seja colocada na forma unitária, ou seja, $30 \%= 0,3$.
  \item A taxa de juros e o número de períodos $(n)$ devem estar sempre na mesma unidade de tempo.
 \end{itemize}
 \end{obs}



 \subsection{Juros compostos}

 Lembremos que no regime do juros composto:
 \begin{itemize}
  \item Os valores dos juros são incorporados ao capital inicial ao final de cada período, formando assim um novo capital sob o qual serão calculados os juros do período seguinte.
  \item Os valores dos juros são portanto diferentes em cada um dos períodos.
 \end{itemize}
 Lembremos também que $M= C+J$ e $J= C \cdot i$ assim no caso dos juros compostos temos:
 \begin{eqnarray*}
  M_1&=& C + C \cdot i = C(1+i) \\
  M_2&=& C(1+i) + C(1+i)\cdot i = C(1+i)(1+i)= C(1+i)^2\\
  M_3&=& C(1+i)^2 + C(1+i)^2\cdot i= C(1+i)^2 \cdot (1+i) = C(1+i)^3 \\
  \vdots & & \vdots\\
  M_n&=& C(1+i)^n
 \end{eqnarray*}
 portanto, o montante após $n$ períodos, a uma taxa de juros $i$ ao período passa a ser
 \[\destaque{M= C(1+i)^n.}\]

 Decorre portanto da fórmula do montante as seguintes fórmulas:
 \begin{itemize}
  \item Cálculo dos juros compostos:
  \[M=C+J \Rightarrow J= M-C \Rightarrow J= C(1+i)^n - C \Rightarrow \destaque{J= C((1+i)^n - 1)}.\]
  \item Cálculo do capital:
  \[M= C(1+i)^n \Rightarrow \destaque{C= \frac{M}{(1+i)^n}}.\]
  \item Cálculo da taxa:
  \[M= C(1+i)^n \Rightarrow (1+i)^n= \frac{M}{C} \Rightarrow 1+i= \left(\frac{M}{C}\right)^{1/n} \Rightarrow \destaque{i= \left(\frac{M}{C}\right)^{1/n} -1}\]
  \item Cálculo do período ou prazo:
  \begin{align*}
  M= C(1+i)^n & \Rightarrow (1+i)^n= \frac{M}{C} \Rightarrow ln(1+i)^n= ln\left(\frac{M}{C}\right) \Rightarrow n \cdot ln(1+i)= ln\left(\frac{M}{C}\right) \\
  & \Rightarrow \destaque{n= \frac{ln\left(\frac{M}{C}\right)}{ln(1+i)}}
  \end{align*}
 \end{itemize}

 \begin{obs}
 \begin{itemize}
  \item O fator $(1+i)^n$ é chamado fator de acumulação de capital.
  \item A taxa de juros e o número de períodos $(n)$ devem estar sempre na mesma unidade de tempo.
 \end{itemize}
 \end{obs}

 \newpage
