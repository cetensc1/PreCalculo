 \section{Exercícios}

\begin{exer}
(UFPE/Covest - 2015) Em uma loja de eletrodomésticos, no início de determinado mês, o número de aparelhos de TV estava para o número de computadores assim como 4 : 5. No final do mês, depois que 160 TVs e 220 computadores foram vendidos, os números de TVs e computadores remanescentes na loja ficaram iguais. Quantos eram os computadores na loja, no início do mês?
  \begin{enumerate}[a)]
  \item 300
  \item 310
  \item 320
  \item 330
  \item 340
  \end{enumerate}
 \end{exer}
 
 \begin{exer}
 (UFPE/Covest - 2015) Júnior divide seu salário mensal entre gastos domésticos e poupança, na proporção de 5 : 2. Um aumento no aluguel, que compõe os gastos domésticos, obrigou Júnior a transferir 1/4 da poupança mensal para os gastos domésticos. Qual a nova proporção entre os gastos domésticos e a poupança de Júnior?
  \begin{enumerate}[a)]
  \item 20:7
  \item 17:9
  \item 15:7
  \item 13:5
  \item 11:3
  \end{enumerate}
 \end{exer}
 
 \begin{exer}
 (Sociesc - 2009) Um livro tem 500 páginas. Maria já leu 310 destas páginas. Sendo assim, para terminar o livro, Maria ainda terá que ler:
  \begin{enumerate}[a)]
  \item 34\% do livro
  \item 42\% do livro
  \item 45\% do livro
  \item 38\% do livro
 \end{enumerate}
 \end{exer}
 
 \begin{exer}
 (Sociesc - 2009) João recebeu de adiantamento, R\$ 300,00, que correspondem a 60\% do seu salário mensal. Sendo assim, podemos afirmar que o salário mensal de João é:
 \begin{enumerate}[a)]
  \item R\$ 550,00
  \item R\$ 500,00
  \item R\$ 650,00
  \item R\$ 480,00
 \end{enumerate}
 \end{exer}
 
 \begin{exer}
 (Sociesc - 2009) Paula resolveu vender 2 vidros de perfume que havia comprado em uma viagem. Um deles ela vendeu por R\$ 88,00 e, vendendo a este preço, Paula teve um lucro de 10\%. O segundo vidro de perfume, ela vendeu para sua irmã, por R\$ 108,00, e, ao fazer este preço para a irmã, Paula teve 10\% de prejuízo.

  Somando as duas vendas, podemos afirmar que Paula obteve:
  \begin{enumerate}[a)]
  \item Um lucro de R\$ 4,00
  \item Um prejuízo de R\$ 20,00
  \item Um lucro de R\$ 20,00
  \item Um prejuízo de R\$ 4,00
 \end{enumerate}
 \end{exer}
 
 \begin{exer}
 (Sociesc - 2009) A empresa XY teve um acréscimo no valor total de suas vendas de R\$ 3.000,00. Isso representa um acréscimo de 20\% em relação ao valor total de suas vendas no mês anterior. De acordo com estas informações podemos afirmar que o valor total de suas vendas no mês anterior foi de:
  \begin{enumerate}[a)]
  \item R\$ 15.000
  \item R\$ 12.000
  \item R\$ 18.000
  \item R\$ 13.500
 \end{enumerate}
 \end{exer} 
 
 \begin{exer}
 (Sociesc - 2010) Uma pequena empresa de camisetas obteve, com a venda de 40 unidades, uma receita de R\$ 1.500,00. O preço de venda para que a receita superasse em 30\% a obtida deveria ser de:
  \begin{enumerate}[a)]
  \item R\$ 44,75
  \item R\$ 48,25
  \item R\$ 44,25
  \item R\$ 44,00
  \item R\$ 48,75
 \end{enumerate}
 \end{exer} 
 
 \begin{exer}
 (Sociesc - 2010) Na compra de um eletrodoméstico tive um desconto de 12\% sobre o preço da etiqueta. Qual era o preço da etiqueta, sabendo que o desconto foi de R\$ 5,40?
  \begin{enumerate}[a)]
  \item R\$ 45,00
  \item R\$ 44,50
  \item R\$ 35,40
  \item R\$ 42,00
  \item R\$ 37,50
 \end{enumerate}
 \end{exer} 
 
 \begin{exer}
 (Sociesc - 2008) Uma doceira faz bolos a um custo unitário de R\$ 11,00. A seguir, contrata os serviços de uma decoradora, que cobra R\$ 7,00 por 5 bolos decorados. Quanto esta doceira deverá cobrar por 10 bolos, se deseja obter um lucro de 30\% em cada bolo vendido?
  \begin{enumerate}[a)]
  \item R\$ 158,40
  \item R\$ 164,20
  \item R\$ 162,40
  \item R\$ 161,20
 \end{enumerate}
 \end{exer} 
 
 \begin{exer}
 (Sociesc - 2009) 40\% de 40\% de 40\% de uma quantidade representam a seguinte porcentagem dessa quantidade
  \begin{enumerate}[a)]
  \item 32\%
  \item 12,8\%
  \item 16\%
  \item 6,4\%
 \end{enumerate}
 \end{exer} 
 
 \begin{exer}
 (Sociesc - 2010) O preço de um liquidificador sofreu um reajuste de 12\%, aumentando para R\$ 60,48. Qual era o preço desse produto antes do reajuste?
  \begin{enumerate}[a)]
  \item R\$ 54,00
  \item R\$ 54,63
  \item R\$ 53,42
  \item R\$ 57,00
 \end{enumerate}
 \end{exer} 
 
 \begin{exer}
 (UDESC/Fepese - 2009) Um recenseador tem que entrevistar 540 pessoas. Sabendo que a cada 7 dias ele entrevista 63 pessoas, assinale a alternativa que indica a porcentagem de pessoas entrevistadas após 30 dias.
  \begin{enumerate}[a)]
  \item 25\% do total
  \item 40\% do total
  \item 50\% do total
  \item 60\% do total
  \item 75\% do total
 \end{enumerate}
 \end{exer} 
 
 \begin{exer}
 (UFPE/Covest - 2015) Segundo o Banco Central, a menor taxa anual do cheque especial cobrada por um banco no Brasil é de $43\%$, enquanto a maior taxa é de $321\%$. De qual percentual a maior taxa é superior à menor taxa? Indique o valor inteiro mais próximo do valor obtido.
 \begin{enumerate}[a)]
 \item 745\%
 \item 647\%
 \item 549\%
 \item 441\%
 \item 433\%
 \end{enumerate}
 \end{exer} 
 
 \begin{exer}
 (UFPE/Covest - 2015) Em uma empresa, 30\% dos funcionários têm menos de 30 anos, $1/20$ deles têm 30 anos, $1/10$ deles têm mais de 30 anos e menos de 35 anos, e os 77 funcionários restantes têm 35 ou mais anos. Quantos funcionários da empresa têm 30 anos?
 \begin{enumerate}[a)]
 \item 4
 \item 5
 \item 6
 \item 7
 \item 8
 \end{enumerate}
 \end{exer} 
 
 \begin{exer}
 (UFPE/Covest - 2015) Um grupo de 120 trabalhadores, todos com a mesma produtividade, constroem uma obra em 5 dias, trabalhando 6 horas por dia. Utilizando 100 trabalhadores, com uma produtividade 10\% menor que a dos anteriores, e trabalhando 8 horas por dia, em quantos dias a mesma obra seria concluída?
 \begin{enumerate}[a)]
 \item 5 dias
 \item 6 dias
 \item 7 dias
 \item 8 dias
 \item 9 dias
 \end{enumerate}
 \end{exer} 
 
 \begin{exer}
 (UFPE/Covest - 2015) No começo de determinado mês, a proporção entre o número de livros de arte e o número de livros de ciência em uma livraria era de $5:2$. No final do mesmo mês, observou-se que $20\%$ dos livros de arte e $20\%$ dos livros de ciência foram vendidos e restava um total de 2100 livros. Quantos livros de ciência foram vendidos no mês?
 \begin{enumerate}[a)]
 \item 130
 \item 140
 \item 150
 \item 160
 \item 170
 \end{enumerate}
 \end{exer} 
 
 \begin{exer}
 (UFPE/Covest - 2015) Um apartamento foi colocado à venda com desconto de 15\% sobre seu preço original. Depois de um mês, como o apartamento não foi vendido, o corretor decidiu dar um novo desconto de 8\% sobre o preço anterior, e o preço resultante do apartamento foi de R\$ 391.000,00. Qual era o preço original do apartamento?
 \begin{enumerate}
 \item R\$ 470.000,00
 \item R\$ 480.000,00
 \item R\$ 490.000,00
 \item R\$ 500.000,00
 \item R\$ 510.000,00
 \end{enumerate}
 \end{exer} 
 
 \begin{exer}
 (TJ/SC - 2018) Simone mora em Florianópolis e comprou móveis em uma fábrica em São Bento do Sul. O gerente da fábrica informou que o preço dos móveis seria acrescido de 20\% pelo transporte da fábrica até a casa de Simone.
  Ao receber os móveis em casa, Simone pagou um total de R\$5100,00. O preço pago apenas pelos móveis foi de:
  \begin{enumerate}[a)]
  \item R\$ 4080,00
  \item R\$ 4140,00
  \item R\$ 4150,00
  \item R\$ 4220,00
  \item R\$ 4250,00
 \end{enumerate}
 \end{exer} 
 
 \begin{exer}
 (Fepese - 2017) Em uma cidades são cometidos $45$ crimes a cada $30$ dias. Logo, em $360$ dias são cometidos:
   \begin{enumerate}[a)]
   \item $430$ crimes.
   \item $445$ crimes.
   \item $510$ crimes.
   \item $540$ crimes.
   \item $570$ crimes.
   \end{enumerate}
 \end{exer} 
 
 \begin{exer}
 (Fepese - 2016 ) Em uma empresa, os setores A e B fazem acordos diferentes relativos à carga horária semanal de trabalho. O setor A trabalha $38,5$ horas por semana, enquanto o setor B trabalha $44,25$ horas por semana. Quantas horas o setor B trabalha a mais por semana do que o setor A?
 \begin{enumerate}[a)]
   \item $300$ minutos.
   \item $315$ minutos.
   \item $330$ minutos.
   \item $345$ minutos.
   \item $360$ minutos.
   \end{enumerate}
 \end{exer} 
 
 \begin{exer}
 (Fepese - 2016) Em uma fábrica de refrigerante, uma máquina enche $600$ garrafas a cada $6$ dias, funcionando $12$ horas por dia. Logo, quantas máquinas, funcionando $8$ horas por dia, são necessárias para encher $16000$ garrafas a cada $30$ dias?
 \begin{enumerate}[a)]
   \item $4$.
   \item $6$.
   \item $8$.
   \item $12$.
   \item $80$.
   \end{enumerate}
 \end{exer} 
 
 \begin{exer}
 (Fepese - 2013) Se em uma fábrica $213$ funcionários montam $711$ telefones por dia, então quantos funcionários são necessário para montar $237$ telefones por dia?
 \begin{enumerate}[a)]
   \item $59$.
   \item $67$.
   \item $71$.
   \item $83$.
   \item $95$.
   \end{enumerate}
 \end{exer} 
 
 \begin{exer}
 (Fepese - 2013) Se $45$ trabalhadores constroem $36$ km de estradas por mês, então $52$ trabalhadores constroem quantos km de estradas por mês?
 \begin{enumerate}[a)]
   \item $40,8$ km.
   \item $40,9$ km.
   \item $41,4$ km.
   \item $41,6$ km.
   \item $42,2$ km.
   \end{enumerate}
 \end{exer} 
 
 \begin{exer}
 (Fepese - 2017) Uma empresa emprega $60$ homens e $70$ mulheres. $75\%$ dos homens falam Inglês, enquanto $40\%$ das mulheres não falam Inglês. Logo, o número de empregados desta empresa que são mulheres ou falam inglês é:
 \begin{enumerate}[a)]
   \item $97$.
   \item $115$.
   \item $127$.
   \item $130$.
   \item $157$.
   \end{enumerate}
 \end{exer} 
 
 \begin{exer}
 (Fepese - 2017) Uma pessoa compra um terreno por $R\$ 150.000,00$. Porém, nos $3$ anos subsequentes à compra, o terreno desvaloriza-se a taxa de $5\%$ ao ano. Qual o valor do terreno $3$ anos após a compra?
  \begin{enumerate}[a)]
   \item Maior que $R\$ 128.500,00$.
   \item Maior que $R\$ 128.250,00$ e menor ou igual a $R\$128.500,00$.
   \item Maior que $R\$ 128.000,00$ e menor ou igual a $R\$128.250,00$.
   \item Maior que $R\$ 127.500,00$ e menor ou igual a $R\$128.00,00$.
   \item Menor ou igual a $R\$ 127.500,00$.
   \end{enumerate}
 \end{exer} 
 
 \begin{exer}
 (Fepese - 2017) Um aeroporto é servido por somente duas empresas aéreas, digamos A e B. Dentre os passageiros que passam pelo aeroporto, $81\%$ fazem uso da empresa A e $50\%$ fazem uso de ambas as empresas. Portanto, a porcentagem de passageiros que utiliza a empresa B é:

  \begin{enumerate}[a)]
   \item Menor que $55\%$.
   \item Maior que $55\%$ e menor que $60\%$.
   \item Maior que $60\%$ e menor que $65\%$.
   \item Maior que $65\%$ e menor que $70\%$.
   \item Maior que $70\%$.
   \end{enumerate}
 \end{exer} 
 
 \begin{exer}
 (Fepese - 2017) Em um supermercado, o preço do litro de leite sobe $R\$ 0,30$ e passa a custar $R\$ 2,10$. Portanto, o porcentual do aumento foi:
    \begin{enumerate}[a)]
    \item Maior que $17\%$.
    \item Maior que $16\%$ e menor que $17\%$.
    \item Maior que $15\%$ e menor que $16\%$.
    \item Maior que $14\%$ e menor que $15\%$.
    \item Menor que $14\%$.
    \end{enumerate}
 \end{exer} 
 
 \begin{exer}
 (Fepese - 2016) Em uma empresa, a razão entre o número de homens e mulheres é $\frac{11}{12}$. Portanto a porcentagen de mulheres nessa empresa é:
    \begin{enumerate}[a)]
    \item Maior que $52,25\%$.
    \item Maior que $52\%$ e menor que $52,25\%$.
    \item Maior que $51,75\%$ e menor que $52\%$.
    \item Maior que $51,5\%$ e menor que $51,75\%$.
    \item Menor que $51,5\%$.
    \end{enumerate}
 \end{exer} 
 
    \begin{exer} (Fepese - 2013) Uma pessoa tem um rendimento mensal fixo e compromete $25\%$ dessa renda com o pagamento do aluguel do mês. Se o custo do aluguel é de $R\$850,00$ mensais, então a renda anual desta pessoa é:
    \begin{enumerate}[a)]
    \item $R\$ 40.000,00$.
    \item $R\$ 40.200,00$.
    \item $R\$ 40.400,00$.
    \item $R\$ 40.600,00$.
    \item $R\$ 40.800,00$.
    \end{enumerate}
    \end{exer}

    \begin{exer} (Fepese - 2013)  Em uma cidade, no mês de janeiro foram feitas $678$ ligações de eletricidade. Deste total, $114$ foram religações; o restante foram ligações novas. Portanto, a porcentagem de ligações novas feitas em janeiro, em relação ao total de ligações efetuadas, é;
    \begin{enumerate}[a)]
    \item Menor do que $80\%$.
    \item Maior do que $80\%$ e menor do que $81\%$.
    \item Maior do que $81\%$ e menor do que $82\%$.
    \item Maior do que $82\%$ e menor do que $83\%$.
    \item Maior do que $83\%$.
    \end{enumerate}
    \end{exer}


    \begin{exer} (Fepese - 2010) Um recenseador tem que entrevistar $540$ pessoas. Sabendo que a cada $7$ dias ele entrevista $63$ pessoas, assinale a alternativa que indica a porcentagem de pessoas entrevistadas após $30$ dias.
    \begin{enumerate}[a)]
    \item $25\%$ do total.
    \item $40\%$ do total.
    \item $50\%$ do total.
    \item $60\%$ do total.
    \item $75\%$ do total.
    \end{enumerate}
    \end{exer}

    \begin{exer} (UEM - 2017) Léo pagou $R\$ 60,00$ por uma camisa. Se a camisa custava $R\$ 75,00$, quantos por cento de desconto ele obteve?
    \begin{enumerate}[a)]
    \item $15 \%$
    \item $18 \%$
    \item $20 \%$
    \item $25 \%$
    \item $30 \%$
    \end{enumerate}
    \end{exer}

    \begin{exer} (FGV - 2017) Uma árvore é 4 m mais alta do que outra árvore. As alturas das duas árvores estão na razão $\frac{2}{3}$. A árvore mais alta mede
    \begin{enumerate}[a)]
    \item 6 m
    \item 8 m
    \item 9 m
    \item 12 m
    \item 15 m
    \end{enumerate}
    \end{exer}


    \begin{exer} (Sociesc - 2009) Num clássico entre Sport e Náutico, o atacante do Náutico abre o placar aos 15 minutos do primeiro tempo com um gol de cabeça. Sabendo-se que uma partida de futebol é composta de 2 tempos regulamentares de 45 minutos e supondo que neste jogo não haverá acréscimos além do tempo regulamentar, podemos afirmar que o tempo que resta, imediatamente após o gol de cabeça do atacante do Náutico, para o Sport tentar mudar este placar é:
  \begin{enumerate}[a)]
  \item 1/6 do tempo regulamentar
  \item 5/8 do tempo regulamentar
  \item 1/3 do tempo regulamentar
  \item 5/6 do tempo regulamentar
 \end{enumerate}
 \end{exer}

\begin{exer} (Sociesc - 2009) A razão entre as idades de Pedro e Carlos é P/C. Se acrescentarmos 6 unidades na idade de Pedro e 9 unidades na idade de Carlos, a razão entre a idade dos dois permanece a mesma. O valor dessa razão é:
  \begin{enumerate}[a)]
  \item 2/9
  \item 2/5
  \item 2/3
  \item 3/5
 \end{enumerate}
 \end{exer}

  \begin{exer} (Sociesc - 2009) Na preparação do solo para determinado plantio, foram utilizados 10 litros de um produto químico para uma área de 500 metros quadrados. Seguindo a mesma lógica o técnico que trabalha
 com este tratamento precisará agora preparar outro solo, para o mesmo tipo de plantio, em uma área de 700 metros quadrados. A quantidade de produto químico que ele deverá utilizar é:
 \begin{enumerate}[a)]
  \item 12 litros
  \item 14 litros
  \item 10 litros
  \item 8 litros
 \end{enumerate}
 \end{exer}

 \begin{exer} (Sociesc - 2009) Em uma obra, 6 pedreiros fazem o acabamento de determinada casa em 24 dias. Com três pedreiros a mais, para executar a mesma tarefa, nas mesmas condições, levarão:
  \begin{enumerate}[a)]
  \item 18 dias
  \item 25 dias
  \item 16 dias
  \item 36 dias
 \end{enumerate}
 \end{exer}

 \begin{exer} (Sociesc - 2010) Com certa quantidade de cobre fabricam-se 1600 metros de fio com secção de $12 mm^2$. Se a secção for de $8mm^2$, quantos metros de fio poderão ser obtidos?
  \begin{enumerate}[a)]
  \item 2350 metros
  \item 2200 metros
  \item 2400 metros
  \item 1980 metros
 \end{enumerate}
 \end{exer}

 \begin{exer} (SOCIESC - Téc. Enfermagem) A razão entre os números $(x-1)$ e 6 é a mesma entre $(x+2)$ e 4, então $x$ vale:
  \begin{enumerate}[a)]
  \item 16
  \item 8
  \item -16
  \item -8
  \item 4
 \end{enumerate}
 \end{exer}

 \begin{exer} (SOCIESC - Téc. Enfermagem) Um caminhão é projetado para levar 30 caixas de frutas ou 25 refrigeradores. Se já há 20 refrigeradores no caminhão, quantas caixas de frutas ainda podem entrar?
 \begin{enumerate}[a)]
  \item 5
  \item 7
  \item 4
  \item 6
  \item 8
 \end{enumerate}
 \end{exer}

 \begin{exer} (TJ/SC - 2018) Dois técnicos analisam 10 processos em 30 dias. Com a mesma eficiência, quatro técnicos analisarão 20 processos em:
    \begin{enumerate}
    \item 15 dias
    \item 30 dias
    \item 60 dias
    \item 90 dias
    \item 120 dias
    \end{enumerate}
\end{exer}

 \textbf{Gabarito:} 1 a); 2 e); 3 d); 4 b); 5 d); 6 a); 7 e); 8 a); 9 d); 10 d); 11 a); 12 c); 13 b); 14 d); 15 a); 16 c); 17 d); 18 e); 19 d); 20 d); 21 c); 22 c); 23 d); 24 b); 25 a); 26 d); 27 b); 28 b); 29 e); 30 e); 31 c); 32 c); 33 e); 34 d); 35 c); 36 b); 37 c); 38 c); 39 d); 40 d); 41 b).
