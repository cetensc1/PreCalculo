%Este trabalho está licenciado sob a Licença Creative Commons Atribuição-CompartilhaIgual 4.0 Internacional. Para ver uma cópia desta licença, visite https://creativecommons.org/licenses/by-sa/4.0/ ou envie uma carta para Creative Commons, PO Box 1866, Mountain View, CA 94042, USA.

%%%%%%%%%%%%%%%%%%%%%%%%%%%%%%%%%%%%%%%%%
% ATENÇÃO
%
% POR SEGURANÇA, NÃO EDITE ESTE ARQUIVO.
%
%%%%%%%%%%%%%%%%%%%%%%%%%%%%%%%%%%%%%%%%

%%%%%%%%%%%%%%%%%%%%%%%%%%%%%%%%%
%   Predefinicoes
%%%%%%%%%%%%%%%%%%%%%%%%%%%%%%%%%

\newif\ifisbook         % O layout será book?
\newif\ifishtml        % O layout será html?

\def\tfn{config.knd}     % Arquivo que guarda as definições do tipo de saída
\def \tdata{}          % Definições do tipo de saída: book, slide ou html.

\openin1=\tfn\relax    % Leitura das definições de saída
\read1 to \tdata
\closein1

\tdata                 % Definições de saída

%%%%%%%%%%%%%%%%%%%%%%%%%%%%%%%%%
%   Opcões de Linguagem
%%%%%%%%%%%%%%%%%%%%%%%%%%%%%%%%%

\usepackage[utf8]{inputenc}
\usepackage[T1]{fontenc}
\usepackage[english,brazil]{babel} %fazemos com que o compilador traduza expressões como “table of contents”
\usepackage[fixlanguage]{babelbib}
\selectbiblanguage{brazil}
\usepackage{graphicx} % Possibilita o uso de figuras e gráficos (suporta os formatos EPS, PDF, JPG e PNG)
\usepackage{color}    % Possibilita o uso de cores no documento
\usepackage{pdfpages} % Possibilita a inclusão da ficha catalográfica
\usepackage{listings}

% Alguns pacotes adicionais (ADICIONE/REMOVA DE ACORDO COM A SUA NECESSIDADE)
\usepackage{microtype} 			% para melhorias de justificação
\usepackage{amsmath}
\usepackage{amsfonts}
\usepackage{amssymb}
\usepackage{amsthm}
\usepackage{textcomp}
\usepackage{gensymb} %usar comando \degree
\usepackage{venndiagram}
\usepackage{tikz}               % para criar desenhos legais
\usetikzlibrary{fit,shapes.geometric,arrows} %pra fazer os diagramas de funções
\usepackage{verbatim}
\usepackage[normalem]{ulem}     %para riscar palavra usando \sout
\usepackage{float}
\usepackage{multirow}           % para tabelas, basicamente essencial
\usepackage{multicol}
\usepackage{cite}               % usado para citações
\usepackage{enumerate}          % para fazer listas com tipos diferentes de itens (a, i, 1,...)
\usepackage{calrsfs}
\usepackage[all]{xy}
\usepackage{graphicx,txfonts}   %test for heart
\usepackage[pdftex]{hyperref}   %usado para incluir links no texto
\usepackage{subfigure}           %para colocar figuras lado a lado
\usepackage{xcolor}
\usepackage{colortbl}
\usepackage{makeidx}            % índice remissivo
\usepackage{fancyhdr}
\usepackage{shadethm}
\usepackage{indentfirst}
\usepackage{bm}

\hypersetup{
    colorlinks = true,
    allcolors = {blue}
}



\newcommand{\fim}{\hfill {$\Box$}}
\newcommand{\destaque}[1]{\colorbox{rosa}{$\displaystyle #1$}}
\newcommand{\iu}{{i\mkern1mu}} %unidade imaginária
\newcommand{\todo}[1]{{\color{red}\textbf{Observação}: Em uma versão futura deste material, também será explicado sobre: #1}}
\newcommand{\sen}{\operatorname{sen}}
\newcommand{\cotan}{\operatorname{cot}}
\newcommand{\cosec}{\operatorname{csc}}
\newcommand{\arcsen}{\operatorname{arcsen}}
\newcommand*\abs[1]{\left|#1\right|}

\newcommand{\exemautorefname}{Exemplo}
\newcommand*{\propautorefname}{Proposição}

\newcommand{\N}{\mathbb{N}}
\newcommand{\Z}{\mathbb{Z}}
\newcommand{\Q}{\mathbb{Q}}
\newcommand{\I}{\mathbb{I}}
\newcommand{\R}{\mathbb{R}}
\newcommand{\C}{\mathbb{C}}

\newcommand{\paralela}{\hspace{3pt}/\hspace{-2pt}/\hspace{3pt}}
\newcommand{\heart}{\ensuremath\varheartsuit} %test coração










%%%%%%%%%%%%%%%%%%%%%%%%%%%%%%%%%%%%%%%%%%%%%%%%%%


%%%%%%%%%%%%%%%%%%%%%%%%%%%%%%%%%%%%%%%%%%%%%%%%%%
% + INTRUCOES PARA O FORMATO LIVRO
%%%%%%%%%%%%%%%%%%%%%%%%%%%%%%%%%%%%%%%%%%%%%%%%%%
\ifisbook
\input preambulo_book.tex
\fi
%%%%%%%%%%%%%%%%%%%%%%%%%%%%%%%%%%%%%%%%%%%%%%%%%%


%%%%%%%%%%%%%%%%%%%%%%%%%%%%%%%%%%%%%%%%%%%%%%%%%%
% + INTRUCOES PARA O FORMATO HTML
%%%%%%%%%%%%%%%%%%%%%%%%%%%%%%%%%%%%%%%%%%%%%%%%%%
\ifishtml
\input preambulo_html.tex
\fi
%%%%%%%%%%%%%%%%%%%%%%%%%%%%%%%%%%%%%%%%%%%%%%%%%%
