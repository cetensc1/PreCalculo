%Este trabalho está licenciado sob a Licença Creative Commons Atribuição-CompartilhaIgual 4.0 Não Adaptada. Para ver uma cópia desta licença, visite https://creativecommons.org/licenses/by-sa/4.0/ ou envie uma carta para Creative Commons, PO Box 1866, Mountain View, CA 94042, USA.

\chapter*{Prefácio}
\addcontentsline{toc}{chapter}{Prefácio}

Este texto começou a ser escrito em 2018, como material de apoio para um curso preparatório para concursos públicos que eu estava ministrando naquele ano. Naquele momento para atender a necessidade dos estudantes tentei escrever de forma simples, clara e objetiva o conteúdo, e sempre ilustrando com exemplos os conteúdos abordados.

Em 2019 ao retornar para a Universidade Federal de Santa Catarina (UFSC) para fazer o doutorado, em conversa com alguns professores verifiquei a necessidade de uma material com todo o conteúdo da disciplina de Pré-cálculo ofertada pela UFSC, como já tinha parte deste conteúdo escrito, trabalho que havia realizado em 2018, me dispus a complementar o texto com o objetivo de atender as necessidades da referida disciplina, mas mantendo a ideia da simplicidade e do grande número de exemplos bem detalhados. 

E porquê resolvi fazer isso?

Quando iniciei minha graduação em Matemática na UFPR no ano 2006, eu tive muita dificuldade com essa disciplina, pois minha formação básica foi toda em escola pública e eu também não sabia a importância dos estudos antes de entrar da graduação, logo comecei o curso com a sensação de nunca ter estudado matemática na vida, assim como muitos de vocês podem estar se sentindo. 

Na época além de assistir as aulas eu precisava estudar muito e buscar os conteúdos em livros para complementar meus estudos, acontece que era muito difícil encontrar livros com as contas básicas, exemplo resolução de equação do 1º grau, abertas com todos os detalhes, então sempre tinha uma parte que eu não entendia, além disso é difícil encontrar todo o conteúdo de Pré-cálculo em um único livro, e quando eu estava começando a graduação eu não sabia estudar usando vários livros, eu sempre me perdia, penso que isso deve acontecer com mais pessoas.

Por estes motivos meu objetivo com este material é ajudar os estudantes que estão começando sua vida acadêmica a entender matemática e conseguir bom desempenho na disciplina de Pré-cálculo, por isso a apostila conta com vários exemplos, com o máximo de detalhes possíveis em cada um deles, mesmo que muitas vezes seja repetitivo, algumas aplicações de conteúdos, em alguns momentos motivações para estudar determinados conteúdos, além de exercícios de concursos públicos que cobram certos conteúdos. 

Nos Apêndices vocês encontram alguns temas que não estão na ementa de Pré-cálculo mas que podem ser úteis para vocês em outras disciplinas, ou eventualmente considerados como de conhecimento de todos em alguns exercícios mesmo durante este curso.

Como a apostila está em fase de construção caso vocês encontrem alguma imprecisão, por favor me avisem pelo e-mail frantriches@gmail.com, para que eu possa corrigir, logo em seguida vocês receberão uma nova versão.

Bons estudos!